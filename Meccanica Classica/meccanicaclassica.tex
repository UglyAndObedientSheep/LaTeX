\documentclass[12pt]{report}
\usepackage[pagebackref=true]{hyperref}
\usepackage[italian]{babel}
\usepackage[utf8]{inputenc}
\usepackage[T1]{fontenc}
\usepackage{latexsym}
\usepackage{graphicx}
\usepackage{epstopdf}
\usepackage{amsmath}
\usepackage{amssymb}
\usepackage{amsfonts}
\usepackage{amsthm}
\usepackage{lipsum}
\theoremstyle{plain}
\newtheorem{thm}{Teorema}[section]
\newtheorem{cor}{Corollario}[thm]
\newtheorem{lem}{Lemma}[section]
\newtheorem{prop}{Proposizione}[section]
\newtheorem{rem}{Nota}[section]
\theoremstyle{definition}
\newtheorem{defn}{Definizione}[section]
\newtheorem{exm}{Esempio}[section]
\theoremstyle{remark}
\newtheorem{oss}{Osservazione}[section]
\usepackage[left=2cm,right=2cm,top=2cm,bottom=2cm]{geometry}
\newcommand{\ham}{\mathcal{H}}
\newcommand{\lag}{\mathcal{L}}
\newcommand{\pdev}[3][]{\frac{\partial^{#1}#2}{\partial #3^{#1}}}
\newcommand{\dev}[3][]{\frac{\mathrm{d}^{#1}#2}{\mathrm{d}#3^{#1}}}
\numberwithin{equation}{section}
\newcommand{\Div}{\mathrm{div}}
\newcommand{\grad}{\mathrm{grad}}
\newcommand{\diff}[1][]{\mathrm{d}#1}
\newcommand{\bra}{\langle}
\newcommand{\ket}{\rangle}

\begin{document}
\begin{titlepage}
\centering
{\Huge Meccanica Classica}\\
\vspace*{0.5cm}
{\small Appunti (non rivisti) delle lezioni del professor d'Emilio}
\vspace*{\stretch{0.5}} \\
\includegraphics[width=250pt,keepaspectratio=true]{Addons/eigenLibrichiaro}
\begin{center}
un progetto di
\end{center}
\includegraphics[width=250pt,keepaspectratio=true]{Addons/eigenlabinvertito2.png} \\
\url{www.eigenlab.org}
\vspace*{\stretch{1}} \\
{\small a cura di}\\
\vspace*{0.5cm}
{\normalsize Francesco Cicciarella\par}
\end{titlepage}
\pagebreak

\section*{Note legali}
\begin{center}
\begin{figure}[htbp]
\centering
\includegraphics[scale=1]{Addons/88x31.png}
\end{figure}
\vspace{0.5cm}
Copyright \copyright \; 2011-2012 di Francesco Cicciarella \\
\textit{Appunti di Meccanica Classica} \\	
è rilasciato sotto i termini della licenza \\
Creative Commons Attribuzione - Non commerciale - Condividi allo stesso modo 3.0 Italia. \\
Per visionare una copia completa della licenza, visita \\
\url{http://creativecommons.org/licenses/by-nc-sa/3.0/it/legalcode}
\end{center}
\section*{Liberatoria, mantenimento e segnalazione errori}
Questo documento viene pubblicato, in formato elettronico, senza alcuna garanzia di correttezza del suo contenuto. Il testo, nella sua interezza, è opera di \\

\vspace{0.3cm}
\begin{flushleft}
\texttt{Francesco Cicciarella}\\
\texttt{<f[DOT]cicciarella[AT]inventati[DOT]org>}
\end{flushleft}
\vspace{0.3cm}
e viene mantenuto dallo stesso, a cui possono essere inviate eventuali segnalazioni di errori.
\vspace{1cm}
\begin{flushright}
Pisa, 10 Ottobre 2012
\end{flushright}
\pagebreak

\tableofcontents
\pagebreak

\chapter{Meccanica Analitica}
\section{Richiami}
\subsection*{Meccanica del punto}
Dato un punto materiale di massa $m$ e coordinata $\mathbf{r}$, funzione del tempo, $\mathbf{r}\equiv\mathbf{r}(t)$, definiamo la velocità del punto, la sua quantità di moto e la forza agente sul punto rispettivamente come
\begin{equation}
\mathbf{v}=\dev{\mathbf{r}}{t},\quad \mathbf{p}=m\mathbf{v},\quad \mathbf{F}=\dev{\mathbf{p}}{t}\;.
\end{equation}
Trovare la traiettoria di un punto, vuol dire risolvere l'equazione differenziale del secondo ordine:
\begin{equation}
\begin{cases}
\mathbf{F}(\mathbf{r})=m\dfrac{\diff^2{r}}{\diff{t}^2}=m\mathbf{a} \\
\\
\mathbf{r}(0)=\mathbf{r}_0,\quad \mathbf{v}(0)=\mathbf{v}_0
\end{cases}\;.
\end{equation}
Da questa notiamo che $\mathbf{F}\equiv 0$ implica $\mathbf{p}$ costante. Dato un punto materiale dotato di impulso $\mathbf{p}$, definiamo il momento angolare rispetto a un polo $O$:
\begin{equation}
\mathbf{L}=\mathbf{r}\times \mathbf{p}\;,
\end{equation}
dove $\mathbf{r}$ è la distanza tra il polo e il punto materiale. Osserviamo adesso:
\begin{equation}
\mathbf{r}\times \dot{\mathbf{p}}=\frac{\mathrm{d}}{\mathrm{d}t}[\mathbf{r}\times\mathbf{p}]=\dev{\mathbf{L}}{t}=\mathbf{M}\;.
\end{equation}
Definiamo dunque il momento della forza come la derivata rispetto al tempo del momento angolare. Il lavoro compiuto da una forza su una particella è dato da:
\begin{align}
W_{12}&=\int_1^2 \mathbf{F}\cdot \diff{\mathbf{s}}=\int_1^2 \mathbf{F}\cdot \dev{\mathbf{s}}{t}\mathrm{d}t \notag \\
&= \int_1^2 m\dot{\mathbf{v}}\cdot\mathbf{v}\mathrm{d}t=\int_1^2 \frac{\mathrm{d}}{\mathrm{d}t}\left[\frac{1}{2}m\mathbf{v}^2\right]\mathrm{d}t
\notag \\
&= \frac{1}{2}m\mathbf{v}_2^2-\frac{1}{2}m\mathbf{v}_1^2=T_2-T_1\;,
\end{align}
dove abbiamo introdotto l'energia cinetica $T=m\mathbf{v}^2/2$. Se la forza è tale che $\mathbf{F}=-\mathbf{\nabla}U(\mathbf{r})$, cioè è scrivibile come gradiente di un potenziale dipendente solo dalla distanza, allora:
\begin{equation}
W_{12}=\int_1^2 -\mathbf{\nabla}U(\mathbf{r})d\mathbf{s}=U_1-U_2\;,
\end{equation}
dove abbiamo introdotto l'energia potenziale $U$. Eguagliando le due espressioni del lavoro così trovate, otteniamo il teorema di conservazione dell'energia:
\begin{equation}
T_1+U_1=T_2+U_2\;.
\end{equation}
\subsection*{Sistemi di punti}
Per un sistema formato da $N$ punti materiali, le equazioni assumono la forma:
\begin{equation}
\dev{\mathbf{p}_{\alpha}}{t}=\mathbf{F}_{\alpha}^{ext}+\sum_{\alpha\ne\beta}\mathbf{F}_{\alpha\beta}\;, 
\end{equation}
(che deve essere sommata sull'indice $\alpha=1,\ldots,N$). Ma per il terzo principio di Newton, $\mathbf{F}_{\alpha\beta}=-\mathbf{F}_{\beta\alpha}$, allora:
\begin{align}
\sum_{\alpha}\dev{\mathbf{p}_{\alpha}}{t}&=\frac{\mathrm{d}}{\mathrm{d}t} \left[\sum_{\alpha}m_{\alpha}\mathbf{v}_{\alpha}\right]=
\sum_{\alpha}\mathbf{F}_{\alpha}^{ext}+\frac{1}{2}\sum_{\alpha,\beta}(\mathbf{F}_{\alpha\beta}-\mathbf{F}_{\beta\alpha})= \sum_{\alpha} \mathbf{F}_{\alpha}^{ext} \label{sec1_npoints}
\end{align}
Chiamiamo $M=\sum_{\alpha} m_{\alpha}$ e $\mathbf{P}=\sum_{\alpha} m_{\alpha}\mathbf{v}_{\alpha}$. L'equazione \eqref{sec1_npoints} diventa pertanto:
\begin{equation}
\frac{\mathrm{d}}{\mathrm{d}t}\mathbf{P}\equiv M\ddot{\mathbf{R}}=\sum_{\alpha} \mathbf{F}_{\alpha}^{ext}\;,
\end{equation}
dove $M\mathbf{R}=\sum_{\alpha}m_{\alpha}\mathbf{r}_{\alpha}$ è la coordinata del baricentro del sistema. Osserviamo che se $\sum_{\alpha} \mathbf{F}_{\alpha}^{ext}=0$, allora $\mathbf{P}$ si conserva. Definiamo il momento angolare totale di un sistema di punti materiali come:
\begin{equation}
\mathbf{L}=\sum_{\alpha}(\mathbf{r}_{\alpha}\times \mathbf{p}_{\alpha})\;.
\end{equation}
Derivando rispetto al tempo:
\begin{align}
\dev{\mathbf{L}}{t}&=\sum_{\alpha} \mathbf{r}_{\alpha}\times \dot{\mathbf{p}}_{\alpha}=\sum_{\alpha} \mathbf{r}_{\alpha}\times \mathbf{F}_{\alpha}^{ext}+\frac{1}{2}\sum_{\alpha,\beta}(\mathbf{r}_{\alpha}\times \mathbf{F}_{\alpha\beta}+\mathbf{r}_{\beta}\mathbf{F}_{\beta\alpha}) \notag \\
&= \sum_{\alpha} \mathbf{r}_{\alpha}\times \mathbf{F}_{\alpha}^{ext}+\frac{1}{2}\sum_{\alpha,\beta} (\mathbf{r}_{\alpha}\times\mathbf{F}_{\alpha\beta}-\mathbf{r}_{\beta}\mathbf{F}_{\alpha\beta}) \notag \\
&=\sum_{\alpha} \mathbf{r}_{\alpha}\times \mathbf{F}_{\alpha}^{ext}+\frac{1}{2}\sum_{\alpha,\beta} [( \mathbf{r}_{\alpha}-\mathbf{r}_{\beta})\times \mathbf{F}_{\alpha\beta} \notag \\
&= \sum_{\alpha} \mathbf{r}_{\alpha}\times \mathbf{F}_{\alpha}^{ext}\;,
\end{align}
dove abbiamo usato il fatto che $\mathbf{F}_{\alpha\beta}$ è diretta lungo la congiungente tra i due punti. Definiamo dunque il momento delle forze totale:
\begin{equation}
\mathbf{N}=\dev{\mathbf{L}}{t}=\sum_{\alpha} \mathbf{r}_{\alpha}\times \mathbf{F}_{\alpha}^{ext}\;.
\end{equation}
\pagebreak
\section{Formalismo Lagrangiano}
Consideriamo un certo sistema ad un dato istante $t$ fisso. Vogliamo su di esso eseguire degli \textit{spostamenti virtuali compatibili} $\delta\mathbf{r}_{\alpha}$, cioè tali che siano piccoli e coerenti con i vincoli del sistema e $\delta\dot{\mathbf{r}}_{\alpha}=0$. Allora possiamo esprimere i $\delta\mathbf{r}_{\alpha}$ in termini delle coordinate lagrangiane $q_k$:
\begin{equation}
\delta\mathbf{r}_{\alpha}=\sum_k \pdev{\mathbf{r}_{\alpha}}{q_k}\delta q_k\;,
\end{equation}
da cui:
\begin{equation}
\sum_{\alpha} (\dot{\mathbf{P}}_{\alpha}-\mathbf{F}_{\alpha}^{ext}-\mathbf{F}_{\alpha}^{reaz})\cdot \delta\mathbf{r}_{\alpha}=0\;,
\end{equation}
ma $\mathbf{F}_{\alpha}^{reaz}=0$ perché i vincoli non compiono lavoro. Otteniamo dunque il principio di D'Alembert dei lavori virtuali:
\begin{equation}
\sum_{\alpha}(\dot{\mathbf{P}}_{\alpha}-\mathbf{F}_{\alpha}^{ext})\cdot\delta\mathbf{r}_{\alpha}=0\;.
\end{equation}
I $\delta\mathbf{r}_{\alpha}$ sono linearmente dipendenti. Cerco di esprimerli in funzione delle coordinate lagrangiane $q_k$:
\begin{align}
-\mathbf{F}_{\alpha}^{ext}\cdot\delta\mathbf{r}_{\alpha} &= -\left(\mathbf{F}_{\alpha}^{ext}\cdot\pdev{\mathbf{r}_{\alpha}}{q_k}\right)\delta q_k=\nabla_{\mathbf{r}_{\alpha}}U\pdev{\mathbf{r}_{\alpha}}{q_k}\delta q_k \notag \\
&=\frac{\partial U}{\partial r^i_{\alpha}}\cdot \frac{\partial r^i_{\alpha}}{\partial q_k}\delta q_k=\pdev{U}{q_k}\delta q_k\;. \\
\dot{\mathbf{P}}_{\alpha}\cdot\delta\mathbf{r}_{\alpha} &= m_{\alpha}\dot{\mathbf{v}}_{\alpha}\cdot\delta\mathbf{r}_{\alpha}=m_{\alpha}\dot{\mathbf{v}}_{\alpha}\cdot\frac{\partial\mathbf{r}_{\alpha}}{\partial q_k}\delta q_k \notag \\
&=\left[\frac{\diff}{\diff{t}}\left(m_{\alpha}\mathbf{v}_{\alpha}\cdot\frac{\partial\mathbf{r}_{\alpha}}{\partial q_k}\right)-m_{\alpha}\mathbf{v}_{\alpha}\cdot\frac{\partial\mathbf{v}_{\alpha}}{\partial q_k}\right]\delta q_k\;.
\end{align}
Poiché $\partial\mathbf{r}_{\alpha}/\partial q_k=\partial\mathbf{v}_{\alpha}/\partial\dot{q}_k$, si ha:
\begin{align}
\dot{\mathbf{P}}_{\alpha}\cdot\delta\mathbf{r}_{\alpha}&=\left[\frac{\mathrm{d}}{\mathrm{d}t}\left(m_{\alpha}\mathbf{v}_{\alpha} 
\cdot \pdev{\mathbf{v}_{\alpha}}{\dot{q}_k}\right)-m_{\alpha}\mathbf{v}_{\alpha}\cdot \pdev{\mathbf{v}_{\alpha}}{q_k}\right]\delta q_k \notag \\
&= \left[\frac{\mathrm{d}}{\mathrm{d}t}\frac{\partial}{\partial\dot{q}_k}\left(\frac{1}{2}m_{\alpha}\mathbf{v}_{\alpha}^2\right)-\frac{\partial}{\partial q_k}\left(\frac{1}{2} m_{\alpha}\mathbf{v}_{\alpha}^2\right)\right]\delta q_k \notag \\
&= \left[\frac{\mathrm{d}}{\mathrm{d}t}\pdev{T}{\dot{q}_k}-\pdev{T}{q_k}\right]\delta q_k\;.
\end{align}
Quindi il principio di D'Alembert in termini delle coordinate lagrangiane diventa:
\begin{equation}
\delta q_k \left[\left(\frac{\mathrm{d}}{\mathrm{d}t}\frac{\partial}{\partial \dot{q}_k}- \frac{\partial}{\partial q_k}\right)T+\frac{\partial}{\partial q_k}U\right]=\delta q_k\left[\frac{\mathrm{d}}{\mathrm{d}t}\frac{\partial}{\partial \dot{q}_k}(T-U)-\frac{\partial}{\partial q_k}(T-U)\right]=0, \quad k=1,\ldots, N\;.
\end{equation}
Definiamo a questo punto la funzione \textit{Lagrangiana} $L=T-U$. Avremo a questo punto le equazioni di Eulero-Lagrange per il sistema:
\begin{equation}
\frac{\diff}{\diff{t}}\pdev{L}{\dot{q}_k}-\pdev{L}{q_k}=0,\qquad k=1,\ldots, N\;,
\end{equation}
che restituiscono direttamente le equazioni del moto.
\subsection{Proprietà della funzione Lagrangiana}
Consideriamo due Lagrangiane $L(q,\dot{q})$ e $L'(q,\dot{q})$ che differiscono per la derivata totale rispetto al tempo di una funzione $F\equiv F(q,t)$:
\begin{equation}
L' = L+\dev{F}{t}=L+\pdev{F}{q_{\alpha}}\dot{q}_{\alpha}+\pdev{F}{t}\;.
\end{equation}
Definiamo il \textit{momento canonico generalizzato} $p_{\alpha}\equiv\partial L/\partial\dot{q}_{\alpha}$. Allora si ha:
\begin{equation}
\pdev{L'}{\dot{q}_{\alpha}}=p_{\alpha}'=p_{\alpha}+\pdev{F}{q_{\alpha}}\;.
\end{equation}
Derivando entrambi i membri rispetto al tempo:
\begin{equation}
\frac{\mathrm{d}}{\mathrm{d}t}\pdev{L'}{\dot{q}_{\alpha}}=\dot{p}_{\alpha}+\frac{\mathrm{d}}{\mathrm{d}t}\pdev{F}{q_{\alpha}}= \frac{\diff}{\diff{t}}\pdev{L}{\dot{q}_{\alpha}}+\frac{\diff}{\diff{t}}\pdev{F}{q_{\alpha}}\;.
\end{equation}
Ma:
\begin{align}
\frac{\diff}{\diff{t}}\pdev{L}{\dot{q}_{\alpha}}&=\pdev{L}{q_{\alpha}}\;, \\
\frac{\diff}{\diff{t}}\pdev{F}{q_{\alpha}} &= \frac{\partial}{\partial q_{\alpha}}\dev{F}{t}\;.
\end{align}
Allora:
\begin{equation}
\frac{\diff}{\diff{t}}\pdev{L'}{\dot{q}_{\alpha}}=\pdev{L}{q_{\alpha}}+\frac{\partial}{\partial q_{\alpha}}\dev{F}{t}=\frac{\partial}{\partial q_{\alpha}}\left(L+\dev{F}{t}\right)=\pdev{L'}{q_{\alpha}}\;.
\end{equation}
Dunque concludiamo che le equazioni del moto generate da $L'$ sono le stesse di quelle generate da $L$. In generale, \textit{due Lagrangiane che differiscono per la derivata totale di una certa funzione generano le stesse equazioni del moto}. \\

Se la Lagrangiana non dipende da una certa coordinata $q_{\alpha}$, detta \textit{coordinata ciclica}, (può tuttavia dipendere dalle sue derivate), allora:
\begin{equation}
\pdev{L}{q_{\alpha}}=0\;.
\end{equation}
Dalle equazioni di Eulero-Lagrange segue che il momento canonico $p_{\alpha}$ si conserva, infatti:
\begin{equation}
\dot{p}_{\alpha}=\frac{\mathrm{d}}{\mathrm{d}t}p_{\alpha}=\frac{\mathrm{d}}{\mathrm{d}t}\pdev{L}{\dot{q}_{\alpha}}=\pdev{L}{q_{\alpha}}=0\;.
\end{equation}
\pagebreak
\section{Leggi di conservazione}
Durante il moto di un sistema meccanico, le $2n$ quantità $q_i$ e $\dot{q}_i$ $(i=1,\ldots,n)$ che caratterizzano lo stato del sistema variano nel tempo. Tuttavia, esistono funzioni di queste quantità che rimangono costanti durante il moto, e dipendono solamente dalle condizioni iniziali. Tali funzioni sono chiamate \textit{integrali primi del moto}. Il numero di integrali primi indipendenti per un sistema con $n$ gradi di libertà è $2n-1$.
\subsection{Energia}
La prima legge di conservazione discende dall'omogeneità del tempo. In virtù di tale omogeneità, la Lagrangiana di un sistema chiuso non dipende esplicitamente dal tempo. La derivata totale rispetto al tempo della Lagrangiana può essere quindi scritta come:
\begin{equation}
\dev{L}{t}=\sum_i\pdev{L}{q_i}\dot{q}_i+\sum_i\pdev{L}{\dot{q}_i}\ddot{q}_i\;.
\end{equation}
Sostituendo $\partial L/\partial q_i$ in accordo con le equazioni di Eulero-Lagrange, otteniamo:
\begin{equation}
\dev{L}{t} =\sum_i\frac{\diff}{\diff{t}}\pdev{L}{\dot{q}_i}+\sum_i\pdev{L}{\dot{q}_i}\ddot{q}_i=\sum_i\frac{\diff}{\diff{t}}\left(\dot{q}_i\pdev{L}{\dot{q}_i}\right)\;,
\end{equation}
ovvero, per linearità dell'operatore derivata:
\begin{equation}
\frac{\diff}{\diff{t}}\left(\sum_i\dot{q}_i\pdev{L}{\dot{q}_i}-L\right)=0\;.
\end{equation}
Notiamo che la quantità:
\begin{equation}
E\equiv \sum_i \dot{q}_i\pdev{L}{\dot{q}_i}-L \label{sec3_energy}
\end{equation}
rimane costante durante il moto, cioè è un integrale primo del moto; essa è detta \textit{energia} del sistema. Sappiamo che in un sistema chiuso la Lagrangiana è della forma $L=T(q,\dot{q})-U(q)$, con $T$ funzione quadratica delle velocità, data da:
\begin{align}
T &= \sum_{\alpha}\frac{1}{2}m_{\alpha}\dot{r}_{\alpha}^2=\frac{1}{2}\sum_{\alpha,k,l}m_{\alpha}\pdev{r_{\alpha}}{q_l}\dot{q}_k\dot{q}_l=\sum_{k,l}F_{kl}(q_1,\ldots,q_N)\dot{q}_k\dot{q}_l\;.
\end{align}
Allora:
\begin{align}
\sum_i\pdev{L}{\dot{q}_i}\dot{q}_i =\sum_i\pdev{T}{\dot{q}_i}\dot{q}_i &= \sum_{k,l}\sum_i\dot{q}_i\frac{\partial}{\partial \dot{q}_i}(F_{kl}\dot{q}_k\dot{q}_l)=\sum_{k,l}F_{kl}\sum_i\dot{q}_i\frac{\partial}{\partial \dot{q}_i}(\dot{q}_k\dot{q}_l)\notag \\
&=\sum_{k,l}F_{kl}\sum_i\left(\delta_{ik}\dot{q}_i\dot{q}_l+\delta_{il}\dot{q}_i\dot{q}_k\right)=\sum_{k,l}(F_{kl}\dot{q}_k\dot{q}_l+F_{kl}\dot{q}_k\dot{q}_l) \notag \\
&=T+T=2T\;.
\end{align}
Da ciò concludiamo che $T$ è omogenea di grado 1. Dunque, sostituendo quanto trovato nella \eqref{sec3_energy}:
\begin{equation}
E=\sum_i \pdev{L}{\dot{q}_i}\dot{q}_i-L=\sum_i \pdev{L}{\dot{q}_i}\dot{q}_i-L=2T-(T-U)=T(q,\dot{q})+U(q)\;,
\end{equation}
che coincide con la nota espressione dell'energia meccanica di un sistema.
\subsection{Momento}
Una seconda legge di conservazione deriva dall'omoegeneità dello spazio. In virtù di tale omigeneità, le proprietà meccaniche di un sistema chiuso rimangono invariante per ogni spostamento congruo dell'intero sistema. Consideriamo uno spostamento infinitesimo $\boldsymbol{\epsilon}$ e ricaviamo la condizione per cui la Lagrangiana rimane invariata. Uno spostamento congruo è una trasformazione in cui ogni punto del sistema è spostato della stessa quantità e nella stessa direzione, e quindi il raggio vettore $\mathbf{r}$ diventa $\mathbf{r}+\boldsymbol{\epsilon}$. La variazione della Lagrangiana per una variazione infinitesima delle coordinate, le velocità delle particelle rimanenti fisse, è
\begin{equation}
\delta L=\sum_{\alpha} \pdev{L}{\mathbf{r}_{\alpha}}\cdot\delta\mathbf{r}_{\alpha}=\boldsymbol{\epsilon}\cdot \sum_{\alpha}\pdev{L}{\mathbf{r}_{\alpha}}\;.
\end{equation}
Dato che $\boldsymbol{\epsilon}$ è arbitrario, la condizione $\delta L=0$ è equivalente a:
\begin{equation}
\sum_{\alpha}\pdev{L}{\mathbf{r}_{\alpha}}=0\;. \label{sec3_dldra}
\end{equation}
Dalle equazioni di Eulero-Lagrange segue che:
\begin{equation}
\sum_{\alpha}\frac{\diff}{\diff[t]}\pdev{L}{\mathbf{v}_{\alpha}}=\frac{\diff}{\diff[t]}\sum_{\alpha}\pdev{L}{\mathbf{v}_{\alpha}}=0.
\end{equation}
Concludiamo quindi che, in un sistema meccanico chiuso, il vettore:
\begin{equation}
\mathbf{P}\equiv \sum_{\alpha}\pdev{L}{\mathbf{v}_{\alpha}}\;,
\end{equation}
rimane costante durante il moto; $\mathbf{P}$ è detto \textit{momento} o \textit{quantità di moto} del sistema. Differenziando la Lagrangiana $L=\sum_{\alpha}\frac{1}{2}m_{\alpha}\mathbf{v}_{\alpha}^2-U(\mathbf{r}_1,\ldots,\mathbf{r}_n)$, troviamo che il momento è dato in termini delle velocità delle particelle da:
 \begin{equation}
 \mathbf{P}\equiv \sum_{\alpha}m_{\alpha}\mathbf{v}_{\alpha}\;.
 \end{equation}
L'additività del momento è evidente. In più, a differenza dell'energia, il momento del sistema è uguale alla somma dei momenti $\mathbf{p}_{\alpha}=m_{\alpha}\mathbf{v}_{\alpha}$ delle singole particelle. \\
L'equazione \eqref{sec3_dldra} ha un semplice significato fisico. La derivata $\partial L/\partial\mathbf{r}_{\alpha}=-
\partial U/\partial\mathbf{r}_{\alpha}$ è la forza $\mathbf{F}_{\alpha}$ che agisce sulla particella $\alpha$-esima. Quindi l'equazione \eqref{sec3_dldra} significa che la somma delle forze agenti su tutte le particelle in un sistema chiuso è zero\footnote{In particolare, per un sistema di sole due particelle, $\mathbf{F}_1+\mathbf{F}_2=0$, che coincide con la \textit{terza legge di Newton}.}.
\begin{equation}
\sum_{\alpha} \mathbf{F}_{\alpha}=0\;.
\end{equation}
Se il moto è descritto per mezzo di coordinate generalizzate $q_i$, le derivate della Lagrangiana rispetto alle velocità generalizzate:
\begin{equation}
p_i=\pdev{\lag}{\dot{q}_i}\;,
\end{equation}
sono chiamate \textit{momenti generalizzati}, e le derivate della Lagrangiana rispetto alle coordinate generalizzate:
\begin{equation}
F_i=\pdev{\lag}{q_i}\;,
\end{equation}
sono chiamate \textit{forze generalizzate}. In questa notazione, le equazioni di Eulero-Lagrange diventano:
\begin{equation}
\dot{p}_i=F_i\;.
\end{equation}
In coordinate cartesiane, i momenti generalizzati sono le componenti dei vettori $\mathbf{p}_{\alpha}$. In generale, tuttavia, le $p_i$ sono funzioni omogenee lineari delle velocità generalizzate $\dot{q}_i$ e non si riducono a prodotti di massa e velocità.
\subsection{Centro di massa}
Il momento di un sistema meccanico chiuso assume valori diversi in diversi sistemi di riferimento inerziali. Se un sistema di riferimento $K'$ si muove con velocità $\mathbf{V}$ relativamente a un altro sistema di riferimento $K$, allora le velocità $\mathbf{v}_{\alpha}'$ e $\mathbf{v}_{\alpha}$ delle particelle relative ai due sistemi di riferimento sono tali che $\mathbf{v}_{\alpha}=\mathbf{v}_{\alpha}'+\mathbf{V}$. I momenti $\mathbf{P}$ e $\mathbf{P}'$ nei due sistemi di riferimento sono dunque legati da:
\begin{equation}
\mathbf{P}=\sum_{\alpha} m_{\alpha}\mathbf{v}_{\alpha}=\sum_{\alpha}m_{\alpha}\mathbf{v}_{\alpha}'+\mathbf{V}\sum_{\alpha}m_{\alpha}\;,
\end{equation}
oppure:
\begin{equation}
\mathbf{P}=\mathbf{P}'+\mathbf{V}\sum_{\alpha}m_{\alpha}\;.
\end{equation}
In particolare, esiste sempre un sistema di riferimento $K'$ in cui il momento totale è zero. Imponendo $\mathbf{P}'=0$ troviamo dunque la velocità di tale sistema di riferimento:
\begin{equation}
\mathbf{V}=\frac{\mathbf{P}}{\sum_{\alpha}m_{\alpha}}=\frac{\sum_{\alpha}m_{\alpha}\mathbf{v}_{\alpha}}{\sum_{\alpha}m_{\alpha}}\;. \label{sec3_cdmvel}
\end{equation}
Questa formula mostra che la relazione fra il momento $\mathbf{P}$ e la velocità $\mathbf{V}$ del sistema è la stessa di quella che sussiste fra il momento e la velocità di una singola particella di massa $M=\sum m_{\alpha}$, la somma delle masse delle particelle del sistema. Il secondo membro della \eqref{sec3_cdmvel} può essere scritto come la derivata totale rispetto al tempo della quantità:
\begin{equation}
\mathbf{R}\equiv \frac{\sum_{\alpha} m_{\alpha}\mathbf{r}_{\alpha}}{\sum_{\alpha} m_{\alpha}}\;.
\end{equation}
Possiamo affermare che la velocità del sistema nel suo complesso è la variazione della posizione nello spazio del punto il cui raggio vettore è $\mathbf{R}$. Questo punto è chiamato \textit{centro di massa} del sistema. La legge di conservazione del momento per un sistema chiuso può essere riformulata dicendo che il centro di massa del sistema si muove di moto rettilineo uniforme. \\

L'energia di un sistema meccanico che è nel complesso a riposo è solitamente chiamata \textit{energia interna} $E_i$. L'energia totale di un sistema che si muove nel complesso con velocità $V$ può essere scritta:
\begin{equation}
E=\frac{1}{2}MV^2+E_i\;. \label{sec3_totalenergy}
\end{equation}
Infatti le energie $E$ e $E'$ di un sistema in due sistemi di riferimento $K$ e $K'$ sono legate da:
\begin{align}
E &= \frac{1}{2}\sum_{\alpha}m_{\alpha}v_{\alpha}^2+U \notag \\
&= \frac{1}{2}\sum_{\alpha}m_{\alpha}(\mathbf{v}_{\alpha}'+\mathbf{V})^2+U \notag \\
&= \frac{1}{2}MV^2+\mathbf{V}\cdot\sum_{\alpha} m_{\alpha}\mathbf{v}_{\alpha}'+\frac{1}{2}\sum_{\alpha} m_{\alpha}v_{\alpha}'^2+U \notag \\
&=E'+\mathbf{V}\cdot\mathbf{P}'+\frac{1}{2}MV^2\;.
\end{align}
Se il centro di massa è a riposo in $K'$, allora $\mathbf{P}'=0$, $E=E_i$ e si ottiene la \eqref{sec3_totalenergy}.
\subsection{Momento angolare}
Deriviamo adesso la legge di conservazione che discende dall'isotropia dello spazio. Questa isotropia implica che le proprietà meccaniche di un sistema chiuso non cambiano quando esso è ruotato interamente in qualunque maniera nello spazio. Consideriamo quindi una rotazione infinitesima del sistema, e otteniamo la condizione affinché la Lagrangiana rimanga invariata. \\

Usiamo il vettore $\delta\boldsymbol{\phi}$ per indicare una rotazione infinitesima, il cui modulo è l'angolo di rotazione $\delta\phi$ e la cui direzione è quella dell'asse di rotazione. Ricaviamo innanzitutto la variazione del raggio vettore da un'origine sull'asse a una qualunque particella del sistema. Lo spostamento è collegata alla rotazione da $|\delta\mathbf{r}|=r\sin\theta\delta\phi$, dove $\theta$ è l'angolo compreso fra l'asse di rotazione e $\mathbf{r}$. La direzione di $\delta\mathbf{r}$ è perpendicolare al piano individuato dai vettori $\mathbf{r}$ e $\delta\boldsymbol{\phi}$. Risulta chiaro dunque che:
\begin{equation}
\delta\mathbf{r}=\delta\boldsymbol{\phi}\times\mathbf{r}\;. \label{sec3_deltar}
\end{equation}
Inoltre, quando il sistema ruota, anche le velocità delle particelle variano in direzione secondo la relazione:
\begin{equation}
\delta\mathbf{v}=\delta\boldsymbol{\phi}\times \mathbf{v}\;. \label{sec3_deltav}
\end{equation}
Scriviamo a questo punto la condizione che la variazione della Lagrangiana sia nulla:
\begin{equation}
\delta L =\sum_{\alpha}\left(\pdev{L}{\mathbf{r}_{\alpha}}\cdot\delta\mathbf{r}_{\alpha}+\pdev{L}{\mathbf{v}_{\alpha}}\cdot\delta\mathbf{v}_{\alpha}\right)=0\;. \label{sec3_deltalag}
\end{equation}
Sostituendo la \eqref{sec3_deltar} e la \eqref{sec3_deltav} nella \eqref{sec3_deltalag}, e ricordando che $\partial L/\partial\mathbf{v}_{\alpha}=\mathbf{p}_{\alpha}$ e $\partial L/\partial\mathbf{r}_{\alpha} =\dot{\mathbf{p}}_{\alpha}$, otteniamo:
\begin{equation}
\sum_{\alpha}(\dot{\mathbf{p}}_{\alpha}\cdot\delta\boldsymbol{\phi}\times\mathbf{r}_{\alpha} + \mathbf{p}_{\alpha}\cdot\delta\boldsymbol{\phi}\times\mathbf{v}_{\alpha})=0\;.
\end{equation}
Permutando ciclicamente i fattori e portando $\delta\boldsymbol{\phi}$ fuori dalla sommatoria:
\begin{equation}
\delta\boldsymbol{\phi}\cdot\sum_{\alpha}(\mathbf{r}_{\alpha}\times\dot{\mathbf{p}}_{\alpha}+\mathbf{v}_{\alpha}\times\mathbf{p}_{\alpha})= \delta\boldsymbol{\phi}\cdot\frac{\diff}{\diff[t]}\sum_{\alpha}\mathbf{r}_{\alpha}\times\mathbf{p}_{\alpha}=0\;.
\end{equation}
Dato che $\delta\boldsymbol{\phi}$ è arbitrario, segue che $(\diff/\diff[t])\sum\mathbf{r}_{\alpha}\times\mathbf{p}_{\alpha}=0$, e concludiamo che il vettore:
\begin{equation}
\mathbf{L}\equiv \sum_{\alpha}\mathbf{r}_{\alpha}\times\mathbf{p}_{\alpha}\;,
\end{equation}
chiamato \textit{momento angolare} del sistema, è conservato durante il moto in un sistema chiuso. Come il momento, esso è additivo.
\pagebreak
\section{Elettromagnetismo}
L'equazione del moto in presenza di campi elettrico e magnetico è data da
\begin{equation}
m\dev{\mathbf{v}}{t}=e\left(\mathbf{E}+\frac{\mathbf{v}}{c}\times\mathbf{B}\right)\;,
\end{equation}
dove $e$ è la carica dell'elettrone, $\mathbf{v}$ è la sua velocità, $\mathbf{E}$ è il campo elettrico, $\mathbf{B}$ il campo magnetico e $c=1/\sqrt{\varepsilon_0\mu_0}$ è la velocità della luce nel vuoto. Il secondo membro rappresenta la \textit{forza di Lorentz generalizzata}. Dalle equazioni di Maxwell, sappiamo che $\nabla\times\mathbf{E}=-\partial\mathbf{B}/\partial t$. Poiché $\nabla\cdot\mathbf{B}=0$, allora $\mathbf{B}=\nabla\times\mathbf{A}$, dove $\mathbf{A}$ è il potenziale vettore del campo magnetico. Introduciamo il simbolo di Ricci $\epsilon_{ijk}$:
\begin{equation}
\epsilon_{ijk}\equiv
\begin{cases}
+1 \quad\mbox{se}\; (i,j,k)=(1,2,3),(2,3,1),(3,1,2) \\
-1 \quad\mbox{se}\; (i,j,k)=(3,2,1),(1,3,2),(2,1,3) \\
0 \quad\mbox{se due indici sono uguali}
\end{cases}\;.
\end{equation}
Tramite il simbolo di Ricci è possibile esprimere i prodotti vettoriali: $(\mathbf{a}\times\mathbf{b})_i=\epsilon_{ijk}a_jb_k$ \footnote{In queste espressioni si usa la notazione di Einstein, ovvero sono omesse le sommatorie sugli indici ripetuti, i.e. $(\mathbf{a}\times\mathbf{b})_i=\sum_{j,k}\epsilon_{ijk}a_jb_k$.}. Vale inoltre la relazione $\epsilon_{ijk}\epsilon_{abk}=\delta_{ia}\delta_{jb}-\delta_{ib}\delta_{ja}$. L'espressione della divergenza del campo magnetico può essere allora riscritta come:
\begin{equation}
\nabla\cdot\mathbf{B}=\partial_iB_i=\partial_i(\nabla\times\mathbf{A})_i=\partial_i\epsilon_{ijk}\partial_jA_k=
\epsilon_{ijk}\partial_i\partial_jA_k=0\;.
\end{equation}
E parimenti, l'espressione del rotore del campo elettrico diventa:
\begin{equation}
\nabla\times\mathbf{E}=-\pdev{\mathbf{B}}{t}=-\frac{\partial}{\partial t}(\nabla\times\mathbf{A})\;,
\end{equation}
cioè:
\begin{equation}
\nabla\times\left(\mathbf{E}+\pdev{\mathbf{A}}{t}\right)=0\;.
\end{equation}
Se un vettore ha rotore nullo, allora esso può essere scritto come il gradiente di una funzione scalare cambiato di segno, quindi:
\begin{equation}
\mathbf{E}+\pdev{\mathbf{A}}{t}=-\nabla\Phi\;.
\end{equation}
$\Phi$ rappresenta il potenziale scalare del campo elettrico. Allora, sostituendo nell'espressione della forza di Lorentz i valori dei campi, otteniamo:
\begin{align}
F_i&=e\left[-\partial_i\Phi-\pdev{A_i}{t}+\frac{1}{c}(\epsilon_{ijk}v_j(\nabla\times\mathbf{A})_k)\right] \notag \\
&=e\left[-\partial_i\Phi-\pdev{A_i}{t}+\frac{1}{c}(\epsilon_{ijk}\epsilon_{lkm}v_j\partial_lA_m)\right] \notag \\
&=e\left[-\partial_i\Phi-\pdev{A_i}{t}+\frac{1}{c}((\delta_{il}\delta_{jm}-\delta_{im}\delta_{jl})v_j\partial_lA_m)\right] \notag \\
&=e\left[-\partial_i\Phi-\pdev{A_i}{t}+\frac{1}{c}(v_j\partial_jA_i-v_l\partial_lA_i)\right] \notag \\
&=e\left[-\partial_i\left(\Phi-\frac{1}{c}\mathbf{v}\cdot\mathbf{A}\right)-\left(\frac{\partial}{\partial t}+\frac{1}{c}\nabla\cdot\mathbf{v}\right)A_i\right] \notag \\
&=e\left[-\partial_i\left(\Phi-\frac{1}{c}\mathbf{v}\cdot\mathbf{A}\right)-\frac{\diff}{\diff[t]}\frac{\partial}{\partial v_i}\left(\Phi-\frac{1}{c}\mathbf{v}\cdot\mathbf{A}\right)\right]\;.
\end{align}
Pertanto la Lagrangiana di un elettrone che si muove in presenza di campi elettrico e magnetico sarà:
\begin{equation}
L=\frac{1}{2}m\dot{\mathbf{r}}^2-e\left(\Phi(\mathbf{r},t)-\frac{1}{c}\dot{\mathbf{r}}\cdot\mathbf{A}(\mathbf{r},t)\right)\;.
\end{equation}
\subsection{Trasformazioni di gauge}
Una \textit{trasformazione di gauge} è una particolare trasformazione dei potenziali $\mathbf{A}$ e $\Phi$ così definita:
\begin{equation}
\begin{cases}
\mathbf{A}_1\to\mathbf{A}_2=\mathbf{A}_1+\nabla\Lambda(\mathbf{r},t) \\
\\
\Phi_1\to\Phi_2=\Phi_1-\dfrac{1}{c}\dfrac{\partial}{\partial t}\Lambda(\mathbf{r},t)
\end{cases}\;,
\end{equation}
dove $\Lambda(\mathbf{r},t)$ è una generica funzione scalare. Osserviamo che:
\begin{equation}
\mathbf{B}_2=\nabla\times \mathbf{A}_2=\nabla\times(\mathbf{A}+\nabla\Lambda)=\nabla\times\mathbf{A}_1+\nabla\times\nabla\Lambda\;,
\end{equation}
ma il rotore di un gradiente è nullo, dunque si ottiene $\mathbf{B}_2=\mathbf{B}_1$, cioè il campo magnetico è invariante per trasformazioni di gauge. Inoltre:
\begin{align}
\mathbf{E}_2&=-\nabla\Phi_2-\frac{1}{c}\pdev{\mathbf{A}_2}{t}=-\nabla\Phi_1+\frac{1}{c}\nabla\pdev{\Lambda}{t}-\frac{1}{c}\pdev{\mathbf{A}_1}{t}-\frac{1}{c}\frac{\partial}{\partial t}\nabla\Lambda \notag \\
&=-\nabla\Phi_1-\frac{1}{2}\pdev{\mathbf{A}_1}{t}=\mathbf{E}_1\;.
\end{align}
Dunque i potenziali ottenuti tramite una trasformazione di gauge generano i medesimi campi $\mathbf{E}$ e $\mathbf{B}$. Ci aspettiamo similmente che la Lagrangiana scritta in termini dei potenziali $\mathbf{A}_2,\Phi_2$ sia la stessa di quella scritta in termini di $\mathbf{A}_1,\Phi_1$. Infatti, si ha:
\begin{align}
L_2 &= \frac{1}{2}m\dot{\mathbf{r}}^2-e\left(\Phi_2-\frac{\dot{\mathbf{r}}}{c}\cdot\mathbf{A}_2\right) \\
&=\frac{1}{2}m\dot{\mathbf{r}}^2-e\left(\Phi_1-\frac{1}{c}\pdev{\Lambda}{t}-\frac{\dot{\mathbf{r}}}{c}\cdot \mathbf{A}_1-\frac{\dot{\mathbf{r}}}{c}\cdot\nabla\Lambda\right) \notag \\
&= \frac{1}{2}m\dot{\mathbf{r}}^2-e\left(\Phi_1-\frac{\dot{\mathbf{r}}}{c}\cdot\mathbf{A}_1\right)+\frac{e}{c}\left(\pdev{\Lambda}{t}+\dot{\mathbf{r}}\cdot\nabla\Lambda\right)\notag \\
&= L_1+\frac{e}{c}\dev{\Lambda}{t}\;.
\end{align}
Poiché le due Lagrangiane differiscono per la derivata totale rispetto al tempo di $\Lambda(\mathbf{r},t)$, esse generano le stesse equazioni del moto. Dunque si conclude che la Lagrangiana è invariante per trasformazione di gauge.
\pagebreak
\section{Calcolo variazionale}
Consideriamo un sistema costituito da una particella di massa $m$ che si muove sotto l'effetto del campo gravitazionale fra due punti di coordinate $(x_0,y_0)$ e $(x_1,y_1)$. Vogliamo trovare la traiettoria che congiunge i due punti per cui il tempo di percorrenza è minimo. Sappiamo che in generale è $\diff[t]=\diff[s]/v$. Allora il tempo di percorrenza $\tau$ è dato da:
\begin{align}
\tau &=\int_0^{\tau}\diff{t}=\int_{x_0}^{x_1}\frac{\sqrt{\diff{x}^2+\diff{y}^2}}{\sqrt{2gx}}=\frac{1}{\sqrt{2g}}\int_{x_0}^{x_1}\sqrt{\frac{1+(y')^2}{x}}\diff{x}\equiv\tau[y(x)]^{x_1}_{x_0}\;.
\end{align}
Dunque si tratta di minimizzare il funzionale $\tau[y(x)]$ nell'intervallo $[x_0,x_1]$. Se $y(x)$ è la funzione che lo minimizza, allora per ogni incremento $\delta y(x)$ si ha $\tau[y(x)]<\tau[y(x)+\delta y(x)]$. In generale, si ha:
\begin{equation}
I[y(x)]=\int_{x_0}^{x_1} F(y,y',x)\diff[x]\;.
\end{equation}
Eseguiamo la trasformazione:
\begin{equation}
\begin{cases}
y(x)\longmapsto y(x)+\delta y(x) \\
\\
y'(x)\longmapsto y'(x)+\dfrac{\diff}{\diff[x]}\delta y(x)=y'(x)+\delta y'(x)
\end{cases}\;.
\end{equation}
L'incremento del funzionale $I$ sarà allora dato da:
\begin{align}
I[y(x)+\delta y(x)]&=\int_{x_0}^{x_1} F(y+\delta y, y'+\delta y',x) \notag \\
&= \int_{x_0}^{x_1} \left[F(y,y',x)+\left(\pdev{F}{y}\delta y+\pdev{F}{y'}\delta y'\right)\right]\diff[x] \notag \\
&= \int_{x_0}^{x_1} F(y,y',x)\diff[x]+\int_{x_0}^{x_1}\left(\pdev{F}{y}\delta y+\pdev{F}{y'}\dev{\delta y}{x}\right)\diff[x] \notag \\
&=I[y(x)]+\int_{x_0}^{x_1} \left(\pdev{F}{y}\delta y+\pdev{F}{y'}\dev{\delta y}{x}\right)\diff[x]\;.
\end{align}
A questo punto imponiamo che la variazione del funzionale $\delta I=I[y+\delta y]-I[y]$ sia nulla, a condizione che la variazione di $y$ si annulli agli estremi, ossia $\delta y(x_1)=\delta y(x_0)=0$:
\begin{align}
\delta I&= \int_{x_0}^{x_1}\left(\pdev{F}{y}\delta y+\pdev{F}{y'}\dev{\delta y}{x}\right)\diff[x]=
\int_{x_0}^{x_1} \pdev{F}{y}\delta y\;\diff[x]+\int_{x_0}^{x_1} \pdev{F}{y'}\dev{\delta y}{x}\;\diff[x]= \notag \\
&=\int_{x_0}^{x_1}\pdev{F}{y}\delta y\;\diff[x]+\left.\pdev{F}{y'}\delta y\right|_{x_0}^{x_1}-\int_{x_0}^{x_1} \frac{\diff}{\diff[x]}\pdev{F}{y'}\delta y\;\diff[x]=0\;.
\end{align}
Il secondo termine è identicamente nullo in quanto abbiamo supposto che la variazione $\delta y$ si annulli agli estremi. Abbiamo dunque:
\begin{equation}
\delta I=\int_{x_0}^{x_1} \left(\frac{\diff}{\diff[x]}\pdev{F}{y'}-\pdev{F}{y}\right)\delta y\;\diff[x]=0\;.
\end{equation}
Poiché $\delta y$ è arbitrario, dobbiamo imporre che il termine in parentesi sia identicamente nullo, ottenendo così le ben note equazioni di Eulero-Lagrange:
\begin{equation}
\frac{\diff}{\diff[x]}\pdev{F}{y'}-\pdev{F}{y}=0\;.
\end{equation}
Nel nostro caso, $F$ non dipende esplicitamente da $y$ ($F=\sqrt{(1+y'^2)/x}$), quindi $y$ è una coordinata ciclica e di conseguenza il momento generalizzato coniugato $\partial F/\partial y'$ è costante:
\begin{equation}
\pdev{F}{y'}=\frac{y'}{\sqrt{x(1+y'^2)}}=c\;.
\end{equation}
Da questa espressione ricaviamo il valore di $y'$ in funzione di $x$ (ponendo $c^2=C$):
\begin{equation}
y'=\sqrt{\frac{Cx}{1-Cx}}\;.
\end{equation}
Integrando, infine, troviamo l'equazione della traiettoria che minimizza il tempo di percorrenza:
\begin{equation}
\int \diff[y]=\int \sqrt{\frac{Cx}{1-Cx}}\diff[x]\qquad \Longrightarrow\qquad  y-K=\int \sqrt{\frac{Cx}{1-Cx}}\diff[x]\;.
\end{equation}
Effettuando la sostituzione $x=\cos^2\theta/C$, $\diff[x]=-\frac{2}{C}\cos\theta\sin\theta\diff[\theta]$ troviamo:
\begin{align}
y-K&=\int -\frac{\cos\theta}{\sin\theta}\frac{2}{C}\cos\theta\sin\theta\;\diff[\theta] \notag \\
&= -\frac{2}{C}\int \cos^2\theta\;\diff[\theta]=-\frac{2}{C}\int \frac{1+\cos(2\theta)}{2}\diff[\theta] \notag \\
&=\frac{1}{C}\int \diff[\theta]-\frac{1}{C}\cos (2\theta)\;\diff[\theta]=-\frac{1}{C}\theta+\frac{1}{2C}\sin(2\theta)\;.
\end{align}
Posto $2\theta=t$, troviamo le equazioni parametriche della curva che minimizza il tempo di percorrenza:
\begin{equation}
\begin{cases}
x=\dfrac{1}{2C}(1+\cos t) \\
\\
y=K+\dfrac{1}{2C}(t+\sin t)
\end{cases}\;,
\end{equation}
che rappresentano una cicloide.
\pagebreak
\section{Oscillatori armonici}
Un oscillatore armonico, nella forma più generale, è descritto da un'equazione del tipo:
\begin{equation}
m\ddot{q}(t)=-\beta\dot{q}(t)-kq(t)+f(t), \qquad \beta,k>0\;.
\end{equation}
\subsection{Caso I - Oscillatore libero smorzato}
Nell'oscillatore libero smorzato si ha $f(t)\equiv 0$, quindi l'equazione del moto è:
\begin{equation}
m\ddot{q}+\beta\dot{q}+kq=0\;. \label{sec5_damped}
\end{equation}
Dividendo per $m$ e ponendo $\omega_0^2=k/m$, $\gamma=\beta/m$ si ottiene:
\begin{equation}
\ddot{q}+\gamma\dot{q}+\omega_0^2q=0\qquad  \Longleftrightarrow\qquad  \left(\frac{\diff^2}{\diff[t]^2}+\gamma\frac{\diff}{\diff[t]}+\omega_0^2\right)q=0\;.
\end{equation}
Cerchiamo una soluzione della forma $q(t)=Ae^{i\alpha t}$. Sostituendo, si ottiene:
\begin{equation}
(-\alpha^2+i\alpha\gamma+\omega_0^2)e^{i\alpha t}=0\;.
\end{equation}
Poiché $e^{i\alpha t}\ne 0$ sempre, si ha che l'uguaglianza è verificata per i valori di $\alpha$ che verificano l'equazione:
\begin{equation}
\alpha^2-i\alpha\gamma-\omega_0^2=0\;,
\end{equation}
le cui soluzioni sono:
\begin{equation}
\alpha=\frac{i\gamma}{2}\pm\frac{1}{2}\sqrt{4\omega_0^2-\gamma^2}\;.
\end{equation}
Definiamo il \textit{fattore di qualità} dell'oscillatore $Q$ come $Q=1/\gamma$. Abbiamo dunque:
\begin{equation}
\alpha=\frac{i}{2Q}\pm\frac{1}{2}\sqrt{4\omega_0^2-\frac{1}{Q^2}}\;.
\end{equation}
Esaminiamo i vari casi. Se $\omega_0^2>1/Q^2$, le soluzioni della \eqref{sec5_damped} saranno:
\begin{equation}
q(t)=e^{i\left(\frac{i}{2Q}\pm\Omega\right)t}=e^{-t/2Q}e^{\pm i\Omega t}, \qquad \Omega=\omega_0\sqrt{1-\frac{1}{(2Q\omega_0)^2}}\;.
\end{equation}
Osserviamo che la presenza del fattore $e^{-t/2Q}$ implica sottosmorzamento delle oscillazioni. Nel caso in cui $\omega_0^2\gg 1/4Q^2$, avremo un oscillatore lievemente smorzato. Scriviamo le due soluzioni indipendenti nella forma:
\begin{equation}
q(t)=A_ce^{-t/2Q}\cos(\Omega t)+A_se^{-t/2Q}(\sin\Omega t)\;.
\end{equation}
Nell'approssimazione di smorzamento lieve, $A_ce^{-t/2Q}\simeq A_se^{-t/2Q} = A$. L'energia dell'oscillatore sarà data da:
\begin{equation}
E=\frac{1}{2}kA^2\simeq A_{c,s}^2e^{-t/Q}\;,
\end{equation}
dunque l'energia diminuisce seguendo un decadimento esponenziale. Nel caso $\omega_0^2<1/Q^2$, abbiamo l'oscillatore \textit{sovrasmorzato}. Le soluzioni in questo caso, entrambi reali, saranno:
\begin{equation}
q(t)=e^{i\left(\frac{i}{2Q}\pm i\Omega\right)t}=e^{-t/2Q}e^{\pm\Omega t}, \qquad \Omega=\omega_0\sqrt{1-\frac{1}{(2Q\omega_0)^2}}\;.
\end{equation}
Nel caso $\omega_0^2=1/4Q^2$, avremo l'oscillatore \textit{criticamente smorzato}. Avremo una sola soluzione indipendente:
\begin{equation}
q_1(t)=e^{-t/2Q}\;.
\end{equation}
Per trovare la seconda soluzione indipendente, la cerchiamo nella forma $q_2=q_1(t)g(t)$. La sostituzione nell'equazione dell'oscillatore restituisce:
\begin{equation}
\ddot{q}_1f+2\dot{q}_1\dot{g}+q_1\ddot{g}+\frac{1}{Q}\dot{q}_1g+\frac{1}{Q}q_1\dot{g}+\omega_0^2q_1g=0\;.
\end{equation}
cioè:
\begin{equation}
g\left(\ddot{q}_1+\frac{1}{Q}\dot{q}_1+\omega_0^2q_1\right)+\dot{g}\left(2\dot{q}_1+\frac{1}{Q}q_1\right)+q_1\ddot{g}=0\;.
\end{equation}
Il coefficiente di $g$ è nullo in quanto $q_1$ è soluzione dell'equazione. Rimane pertanto:
\begin{equation}
\dot{g}\left(2\dot{q}_1+\frac{1}{Q}q_1\right)+q_1\ddot{g}=0 \Longleftrightarrow \dot{g}\left(-\frac{1}{Q}e^{-t/2Q}+\frac{1}{Q}e^{-t/2Q}\right)+q_1\ddot{g}=0\;.
\end{equation}
E dunque:
\begin{equation}
q_1\ddot{g}=0\qquad  \Longrightarrow \qquad \ddot{g}=0\;,
\end{equation}
in quanto $q_1\ne 0$. Integrando due volte troviamo infine $g(t)=at+b$. Quindi la soluzione generale per l'oscillatore criticamente smorzato sarà:
\begin{equation}
q(t)=q_1(t)+q_2(t)=e^{-t/2Q}+(at+b)e^{-t/2Q}=(at+b+1)e^{-t/2Q}\;.
\end{equation}
\subsection{Caso II - Oscillatore forzato}
Inseriamo adesso un termine di sorgente $f(t)$ (non consideriamo però in prima analisi il termine di smorzamento, i.e $\gamma=0$). L'equazione sarà allora:
\begin{equation}
\ddot{q}+\omega_0^2q=f(t)\;, \label{sec5_forced}
\end{equation}
ovvero:
\begin{equation}
Oq=f(t), \qquad O\equiv\frac{\diff^2}{\diff[t^2]}+\omega_0^2\;.
\end{equation}
Scomponiamo la sorgente in sorgenti puntiformi e cerchiamo una funzione $G(t)$ tale che:
\begin{equation}
OG(t)=\delta(t)\;,
\end{equation}
dove $\delta(t)$ è la \textit{delta di Dirac}. Effettuando una traslazione otteniamo:
\begin{equation}
\frac{\diff^2}{\diff[t^2]}G(t-t')+\omega_0^2G(t-t')=\delta(t-t')
\end{equation}
Allora la funzione:
\begin{equation}
q(t)\equiv \int_{-\infty}^{\infty} \diff[t']G(t-t')f(t')\;,
\end{equation}
è soluzione della \eqref{sec5_forced}. Infatti:
\begin{equation}
O_tq(t)=O_t\int_{-\infty}^{+\infty}\diff{t'}G(t-t')f(t')=\int_{-\infty}^{+\infty}\diff{t'}O_tG(t-t')f(t')=\int_{-\infty}^{+\infty}\diff{t'}\delta(t-t')f(t')=f(t)\;.
\end{equation}
Indicando con $G_R$ la funzione tale che $OG_R=\delta$, allora $\forall \alpha,\beta$ la funzione $G(t)=G_R(t)+\alpha\cos(\omega_0 t)+\beta\sin(\omega_0 t)$ è soluzione dell'equazione, dove $\alpha\cos(\omega_0 t)+\beta\sin(\omega_0 t)$ è soluzione dell'omogenea associata $OG=0$. Abbiamo come condizioni iniziali $G(0)=0$ e $\dot{G}(0)=1$. Imponendole troviamo:
\begin{align}
&\alpha=0, &\beta=\frac{1}{\omega_0}\;.
\end{align}
Dunque la funzione:
\begin{equation}
G_R=\frac{1}{\omega_0}\sin(\omega_0 t)\qquad (t>0)\;,
\end{equation}
risolve $OG_R=\delta$, e pertanto:
\begin{align}
q(t) &=\int_{-\infty}^{\infty}\diff{t'}G_R(t-t')f(t')\;, \notag \\
G_R(t-t') &=\begin{cases}
0\qquad \mbox{se}\; t-t'<0 \\
\\
\dfrac{1}{\omega_0}\sin(\omega_0(t-t'))\quad \mbox{se}\;t-t'>0\;.
\end{cases}\;,
\end{align}
risolve $Oq=f(t)$. In particolare, poiché $t-t'<0$ per $t'>t$, l'espressione di $q$ si riduce a:
\begin{equation}
q(t)=\frac{1}{\omega_0}\int_{-\infty}^t\diff[t']\sin(\omega_0(t-t'))f(t')\;.
\end{equation}
Riprendiamo in esame l'equazione dell'oscillatore smorzato e forzato:
\begin{equation}
\ddot{q}+\frac{1}{Q}\dot{q}+\omega_0^2q=f(t)\;,
\end{equation}
ed applichiamo lo stesso ragionamento: cerchiamo una funzione $G$ che risolve:
\begin{equation}
\ddot{G}+\frac{1}{Q}\dot{G}+\omega_0^2G=\delta(t)\;.
\end{equation}
Si ha, pertanto:
\begin{equation}
G_R(t)=\frac{e^{-t/2Q}}{\Omega}\sin(\Omega t),\qquad \Omega=\sqrt{\omega_0^2-\frac{1}{4Q^2}}\;.
\end{equation}
Indicando con $q_{om}$ la soluzione dell'equazione omogenea associata si ottiene:
\begin{equation}
q(t)=\int_{-\infty}^{\infty}\frac{e^{-(t-t')/2Q}}{\Omega}\sin(\Omega(t-t'))f(t')\diff[t']+q_{om}\;.
\end{equation}
Se $\omega_0^2\gg 1/4Q^2$, allora $\Omega \sim \omega_0$. Scriviamo inoltre $f(t)=f_0\sin(\omega t)$. Quindi abbiamo:
\begin{align}
q(t) &= f_0\int_0^t\frac{e^{-(t-t')/2Q}}{\omega_0}\sin(\omega_0(t-t'))\sin(\omega t')\diff{t'}+q_{om} \notag \\
&=\frac{f_0}{2i\omega_0}e^{-t/2Q}\int_0^te^{t'/2Q}(e^{i\omega_0t}e^{-i\omega_0t'}-e^{-i\omega_0t}e^{i\omega_0t'})\left(\frac{e^{i\omega t}-e^{-i\omega t}}{2i}\right)\diff{t'}+q_{om}
\end{align}
Se $t\gg Q$, $q_{om}\simeq 0$:
\begin{align}
q(t) &= -\frac{f_0}{4\omega_0}e^{-t/2Q}\int_{-\infty}^{\infty}\left\{\left[e^{(i(\omega-\omega_0-i/2Q)t'}-e^{i(-\omega-\omega_0-i/2Q)t'}\right]e^{i\omega_0t}+\right. \notag \\
&+\left.\left[e^{i(-\omega+\omega_0-i/2Q)t'}-e^{i(\omega+\omega_0-i/2Q)t'}\right]e^{-i\omega_0t}\right\}\diff{t'}\;.
\end{align}
I termini aventi $\omega$ e $\omega_0$ di segno concorde tendono a zero e possono essere trascurati:
\begin{equation}
q(t)=-\frac{f_0}{4\omega_0}e^{-t/2Q}\int_{-\infty}^{\infty}\left[e^{t'/2Q+i[(\omega-\omega_0)t'+\omega_0t]}+e^{t'/2Q-i[(\omega-\omega_0)t'+\omega_0t]}\right]\diff{t'}\;.
\end{equation}
Poiché sto sommando un numero e il suo complesso coniugato (c.c), prendo due volte la parte reale:
\begin{align}
q(t)&\simeq -\frac{f_0}{4\omega_0}e^{-t/2Q}\left[\frac{e^{i\omega_0t}\cdot \left. e^{i(\omega-\omega_0-i/2Q)t'}\right|_0^t}{i(\omega-\omega_0-i/2Q)}+\mbox{c.c.}\right] \notag \\
&=-\frac{f_0}{4\omega_0}e^{-t/2Q}\left[e^{i\omega_0t}\frac{e^{i(\omega-\omega_0-i/2Q)t}-1}{i(\omega-\omega_0-i/2Q)}+\mbox{c.c.}\right]\;.
\end{align}
Per $t\gg Q$, si ha $e^{-t/Q}\ll 1$, quindi il termine può essere trascurato per tempi lunghi:
\begin{align}
q(t)&= -\frac{f_0}{4\omega_0}\left[\frac{e^{i\omega_0t}e^{i(\omega-\omega_0)t}}{i(\omega-\omega_0-i/2Q)}+\mbox{c.c.}\right] \notag \\
&= -\frac{f_0}{4\omega_0}\frac{e^{i\omega t}}{i(\omega-\omega_0-i/2Q)}+\mbox{c.c.}\;.
\end{align}
Questa rappresenta l'espressione della soluzione generale dell'equazione del moto. L'energia dell'oscillatore infine sarà data da:
\begin{equation}
E(\omega)=|q(t)|^2=\frac{1}{(\omega-\omega_0)^2+1/4Q^2}\;.
\end{equation}
Osserviamo che l'energia si distribuisce, in funzione della frequenza, secondo una Lorentziana.
\pagebreak
\section{Piccole oscillazioni}
\subsection{Richiami di algebra lineare}
Sia $V$ uno spazio vettoriale sul corpo complesso e $u,v,w$ i suoi elementi, denotati $| u\ket,|v\ket,|w\ket$. Un \textit{prodotto scalare} su $V$ è un'applicazione $({},{}):V\times V\to\mathbb{C}$ denotata con $(u,v)=\bra u|\cdot|v\ket=\bra u|v\ket$ che gode delle seguenti proprietà:
\begin{enumerate}
\item $\bra u|\lambda v+\mu w\ket=\lambda\bra u|v\ket+\mu \bra u|w\ket$ (linearità nella seconda variabile);
\item $\bra u|v\ket=\bra v|u\ket^*$ (sesquilinearità nella prima variabile);
\item $\bra u|v\ket=0\; \forall v\qquad  \Longleftrightarrow \qquad u= 0$ (non degenerazione del prodotto scalare).
\end{enumerate}
Un prodotto scalare non degenere induce una \textit{norma} $||u||^2=\bra u|u\ket$ con le seguenti proprietà:
\begin{enumerate}
\item $||\lambda u||=|\lambda|||u||\; \forall \lambda\in \mathbb{C}$;
\item $||u+v||\le||u||+||v||$ (disuguaglianza triangolare);
\item $||u||=0 \quad \Longleftrightarrow\quad  u\equiv 0$.
\end{enumerate}
\subsection*{Trasformazioni lineari Hermitiane}
Sia $T$ un operatore lineare e $u,v\in V$. Consideriamo il seguente prodotto scalare:
\begin{equation}
(u,Tv)=u_i^*T_{ij}v_j=T_{ij}u_i^*v_j=({}^tT_{ij}^*u_i)^*v_j=(T^{\dagger}u,v)
\end{equation}
dove:
\begin{equation}
(T^{\dagger})_{ij}\equiv ({}^tT)_{ij}^*
\end{equation}
è l'operatore \textit{aggiunto} di $T$. In particolare, un operatore $T$ si dice \textit{hermitiano} se è autoaggiunto, cioè $T=T^{\dagger}$. \\

Introducendo la \textit{notazione di Dirac}:
\begin{equation}
\bra Tu|v\ket=\bra u|Tv\ket\equiv \bra u|T|v\ket\;.
\end{equation}
Un operatore ammette un \textit{sistema completo di autovalori} quando:
\begin{equation}
T|k\ket=\lambda_k|k\ket,\qquad k=1,\ldots,n\;.
\end{equation}
Un'osservazione significativa è che \textit{gli autovalori di un operatore Hermitiano sono tutti reali}. Infatti:
\begin{equation}
\bra k|T|k\ket=\begin{cases}
\lambda_k\bra k|k\ket\quad \mbox{se $T$ agisce a destra} \\
\\
\lambda_k^*\bra k|k\ket\quad \mbox{se $T$ agisce a sinistra}
\end{cases}\;.
\end{equation}
Allora $(\lambda_k-\lambda_k^*)\bra k|k\ket=0$. Ma $\bra k|k\ket=||k||^2$ è sicuramente diverso da zero, dunque risulta $\lambda_k=\lambda_k^*$ e quindi $\lambda_k\in\mathbb{R}$.
\\

Se un operatore $T$ è Hermitiano, allora:
\begin{itemize}
\item $T$ ha tutti gli autovalori reali;
\item autovettori relativi ad autovalori diversi sono ortogonali.
\end{itemize}
Abbiamo dunque:	
\begin{align}
&|k\ket\bra k|l\ket=0,\qquad |k\ket\bra k|k\ket=|k\ket\;, \\
&|k\ket\bra k|\left(\alpha|k\ket+\sum_{l\ne k}\beta_l|l\ket\right)=\alpha|k\ket\;.
\end{align}
Si definisce di conseguenza un \textit{operatore di proiezione} o \textit{proiettore}:
\begin{equation}
P_k\equiv |k\ket\bra k|\;.
\end{equation}
Valgono i seguenti fatti:
\begin{itemize}
\item $P_k^2=P_k$;
\item i proiettori sono operatori Hermitiani;
\item $\sum_k |k\ket\bra k|=\mathbb{I}_{n\times n}$.
\end{itemize}
Se un operatore $T$ è Hermitiano, quindi ammette un set completo di autovettori, allora la sua \textit{decomposizione spettrale} sarà:
\begin{equation}
T=\sum_k |k\ket\lambda_k\bra k|\;.
\end{equation}
Si verificha che l'operatore inverso $T^{-1}$ ammette un set completo di autovettori, e la sua decomposizione spettrale sarà:
\begin{equation}
T^{-1}=\sum_l |l\ket\frac{1}{\lambda_l}\bra l|\;.
\end{equation}
Infatti:
\begin{equation}
TT^{-1}=\sum_{k,l}|k\ket\lambda_k\bra k|l\ket\frac{1}{\lambda_l}\bra l|=\sum_{k,l}|k\ket\lambda_k\delta_{kl}\frac{1}{\lambda_l}\bra l|=\sum_k|k\ket\bra k|=\mathbb{I}_{n\times n}\;.
\end{equation}
A partire dalla decomposizione spettrale di un operatore $T$, è possibile definire una \textit{funzione di operatori}:
\begin{equation}
f(T)=\sum_k |k\ket f(\lambda_k)\bra k|\;.
\end{equation}
Se $f(\lambda_k)\in\mathbb{R}$ per ogni $k$, allora $f$ è Hermitiano. \\
\\
Consideriamo adesso un sistema descritto da $N$ coordinate generalizzate $\xi_1,\ldots,\xi_N$, con cui è possibile esprimere le coordinate cartesiane $\mathbf{r}_{a}$ tramite espressioni indipendenti dal tempo, i.e. $\mathbf{r}_a=\mathbf{r}_a(\xi)$, dove $\xi=\{\xi_1,\ldots,\xi_N\}$. La Lagrangiana del sistema sarà data da:
\begin{equation}
L(\xi,\dot{\xi})=\sum_a\frac{1}{2}m_a\dot{\mathbf{r}}_a\cdot\dot{\mathbf{r}}_a-U(\mathbf{r}_a)=\sum_{i,j}\frac{1}{2}A_{ij}(\xi)\dot{\xi}_i\dot{\xi}_j-V(\xi)\;,
\end{equation}
dove:
\begin{equation}
A_{ij}(\xi)=\sum_a m_a\pdev{\mathbf{r}_a}{\xi_i}\cdot\pdev{\mathbf{r}_a}{\xi_j}\;.
\end{equation}
Sia $\bar{\xi}$ un punto di equilibrio stabile del sistema, cioè tale che:
\begin{align}
&\left.\frac{\partial}{\partial\xi_i}V(\xi)\right|_{\xi=\bar{\xi}_i}=0,\qquad i=1,\ldots, N\;, \\
&\left.\frac{\partial^2}{\partial\xi_i\partial\xi_j}V(\xi)\right|_{\xi_i=\bar{\xi}_i}\equiv V_{ij}=V_{ji}>0\; \forall i,j\;,
\end{align}
allora, sviluppando la Lagrangiana intorno al punto $\xi=\bar{\xi}$ e introducendo il nuovo set di coordinate generalizzate $q_i\equiv \xi_i-\bar{\xi}_i$ otteniamo:
\begin{equation}
L(q,\dot{q})=\frac{1}{2}T_{ij}\dot{q}_i\dot{q}_j-\frac{1}{2}V_{ij}q_iq_j\;.
\end{equation}
Le equazioni del moto generate da questa Lagrangiana saranno:
\begin{equation}
T_{ij}\ddot{q}_j+V_{ij}q_j=0\;,
\end{equation}
che costituiscono un sistema differenziale lineare omogeneo nella variabile $q(t)\equiv\{q_1(t),\ldots,q_N(t)\}$. Cercando la soluzione nella forma $q_i(t)=q_i(0)e^{i\omega t}$ troviamo che, posto $q(0)=\{q_1(0),\ldots,q_N(0)\}$:
\begin{equation}
(-\omega^2T+V)q(0)=0\;. \label{sec7_eqsecolare}
\end{equation}
In questa equazione le incognite sono i valori che $\omega^2$ deve assumere affinché l'unica soluzione non sia quella identicamente nulla e le componenti dei vettori $q(0)$. Di quest'ultimi, essendo il sistema lineare ed omogeneo, si potranno trovare unicamente i rapporti fra le componenti di ciascun vettorie $q_a(0)/q_b(0)$ per $a,b=1,\ldots, N$. Il problema è del tutto analogo a quello della diagonalizzazione di una matrice $(-\lambda \mathbb{I}_N+A)a=0$, in cui $\lambda$ sono gli autovalori e $a$ i corrispondenti autovettori della matrice $A$. Il problema \eqref{sec7_eqsecolare} non è perfettamente identico alla diagonalizzazione di una matrice, ed in più vi è il problema di garantire la positività degli $\omega^2$. Se trovassimo un valore $\omega^2<0$, il moto corrispondente sarebbe del tipo $q(t)=ce^{|\omega|t}+de^{-|\omega|t}$ che non si svolge in un intorno del punto di equilibrio stabile $\bar{\xi}$ precedentemente determinato. Tuttavia, il problema \eqref{sec7_eqsecolare} può essere ricondotto a quello della diagonalizzazione di una matrice. Osserviamo innanzitutto che $T$ è una matrice Hermitiana (infatti è reale e simmetrica) e definita positiva:
\begin{equation}
\bra \dot{q}|\mathcal{T}|\dot{q}\ket=\sum_{i,j}\dot{q}_i\mathcal{T}_{ij}\dot{q}_j>0\qquad \forall \dot{q}\;.
\end{equation}
A questo punto è sufficiente riscrivere l'equazione \eqref{sec7_eqsecolare} nella forma:
\begin{equation}
\frac{1}{\sqrt{T}}(-\omega^2T+V)\frac{1}{\sqrt{T}}\sqrt{T}q(0)=\left(-\omega^2\mathbb{I}_N+\frac{1}{\sqrt{T}}V\frac{1}{\sqrt{T}}\right)\sqrt{T}q(0)\;,
\end{equation}
che ha esattamente la forma del problema di diagonalizzazione, con $A=T^{-1/2}VT^{1/2}$. Notiamo che:
\begin{itemize}
\item $A$ è simmetrica: ${}^tA={}^tT^{-1/2}{}^tV{}^tT^{-1/2}=T^{-1/2}VT^{-1/2}=A$, in quanto sia $T$ (e le sue funzioni) che $V$ sono simmetrice;
\item $A$ è definita positiva: $\bra u|A|u\ket=\bra u|T^{-1/2}VT^{-1/2}|u\ket=\bra v|V|v\ket>0$, con $|v\ket=T^{-1/2}|u\ket$.
\end{itemize}
La matrice $A$ avrà autovalori positivi e autovettori "a sufficienza" che chiameremo $Q_i$ (vettori colonna con $N$ componenti reali corrispondenti agli autovalori $\omega^2_i$) normalizzabili a $1$: $\bra Q_i|Q_j\ket=\delta_{ij}$. La più generale soluzione reale corrispondente all'autovalore $\omega_i^2$ di $A$ è quindi:
\begin{equation}
\alpha_iQ_i\cos(\omega_it)+\beta_iQ_i\sin(\omega_it)\;,
\end{equation}
e la soluzione generale dell'equazione delle piccole oscillazioni è:
\begin{equation}
q(t)=\sum_{i=1}^N\left(\alpha_iQ_i\cos(\omega_it)+\beta_iQ_i\sin(\omega_it)\right),\qquad \alpha_i,\beta_i\in\mathbb{R}\;.
\end{equation}
Resta da calcolare l'unica soluzione che evolve nel tempo con dati iniziali $q(0),\dot{q}(0)$. Ponendo $t=0$ nell'equazione precedente si ottiene:
\begin{equation}
q(0)=\sum_{i=1}^N \alpha_iQ_i\qquad \implies\qquad \alpha_i = \bra Q_i|q(0)\ket\;.
\end{equation}
Mentre, derivando la soluzione e imponendo $t=0$ troviamo:
\begin{equation}
\dot{q}(0)=\sum_{i=1}^N\omega_i\beta_iQ_i\qquad \implies\qquad \beta_i=\omega_i^{-1}\bra Q_i|\dot{q}(0)\ket\;.
\end{equation}
\pagebreak
\section{Problema dei due corpi}
Consideriamo il moto di due corpi di massa $m_1$, $m_2$ descritto in coordinate Cartesiane dalla Lagrangiana:
\begin{equation}
L=\frac{1}{2}m_1\dot{\mathbf{r}}_1^2+\frac{1}{2}m_2\dot{\mathbf{r}}_2^2-V_1(\mathbf{r}_1)-V_2(\mathbf{r}_2)-V_{12}(\mathbf{r}_1,\mathbf{r}_2)\;,
\end{equation}
dove $V_{12}$ è il potenziale di interazione. Se questo è nullo, i sistemi sono disaccoppiati e possono essere trattati singolarmente. Supponiamo che sul sistema non agiscano campi esterni, i.e. $V_1=V_2\equiv 0$. Notiamo che il potenziale di interazione deve essere invariante per traslazione, quindi non può dipendere separatamente da $\mathbf{r}_1$ e $\mathbf{r}_2$. Allora $V(\mathbf{r}_1,\mathbf{r}_2)\equiv V(\mathbf{r}_1-\mathbf{r}_2)$, dove $\mathbf{r}_1-\mathbf{r}_2$ è la \textit{distanza relativa} fra i due corpi. Allora la Lagrangiana sarà:
\begin{equation}
L=\frac{1}{2}m_1\dot{\mathbf{r}}_1^2+\frac{1}{2}m_2\dot{\mathbf{r}}_2^2-V(\mathbf{r}_1-\mathbf{r}_2)\;.
\end{equation}
Se applichiamo al sistema una variazione continua, allora il moto dei due corpi sarà descritto dalle medesime equazioni, i.e. la Lagrangiana deve rimanere invariata. Applichiamo al sistema la trasformazione continua:
\begin{equation}
\begin{cases}
\mathbf{r}_i\longmapsto\mathbf{r}_i+\epsilon\mathbf{n}\\
\\
\dot{\mathbf{r}}_i\longmapsto\dot{\mathbf{r}}_i
\end{cases}\;.
\end{equation}
La variazione della Lagrangiana sarà data da:
\begin{equation*}
\delta L=\pdev{T}{\dot{r}_{1k}}\delta\dot{r}_{1k}+\pdev{T}{\dot{r}_{2k}}\delta\dot{r}_{2k}-\pdev{V}{r_{1k}}\delta r_{1k}-\pdev{V}{r_{2k}}\delta r_{2k}=0\;.
\end{equation*}
Poiché nella trasformazione in questione le velocità non cambiano, $\delta\dot{r}_{1k}=\delta\dot{r}_{2k}=0$. Dunque:
\begin{align*}
\delta\lag&=-\pdev{V}{r_{1k}}\delta r_{1k}-\pdev{V}{r_{2k}}\delta r_{2k}=\mbox{(Eulero-Lagrange)} \\
&=\frac{\diff}{\diff[t]}\left(\pdev{\lag}{\dot{r}_{1k}}+\pdev{\lag}{\dot{r}_{2k}}\right) \\
&=\frac{\diff}{\diff[t]}(m_1\dot{\mathbf{r}}_1+m_2\dot{\mathbf{r}}_2)=0\;.
\end{align*}
Da cui deduciamo che la quantità:
\begin{equation}
\mathbf{P}\equiv m_1\dot{\mathbf{r}}_1+m_2\dot{\mathbf{r}}_2\;,
\end{equation}
detta \textit{impulso totale}, è un integrale primo del moto. A questo punto, definiamo il nuovo set di coordinate generalizzate:
\begin{equation}
\begin{cases}
\mathbf{r}\equiv \mathbf{r}_1-\mathbf{r}_2 \\
\\
\mathbf{R}\equiv \dfrac{m_1\mathbf{r}_1+m_2\mathbf{r}_2}{m_1+m_2}
\end{cases}\;.
\end{equation}
Invertendo, ricaviamo le espressioni di $\mathbf{r}_1,\mathbf{r}_2$ in funzione delle nuove coordinate $\mathbf{r},\mathbf{R}$, ottenendo:
\begin{equation*}
\begin{cases}
\mathbf{r}_1=\mathbf{R}+\dfrac{m_2}{M}\mathbf{r} \\
\\
\mathbf{r}_2=\mathbf{R}-\dfrac{m_1}{M}\mathbf{r}
\end{cases}, \qquad
\begin{cases}
\dot{\mathbf{r}}_1=\dot{\mathbf{R}}+\dfrac{m_2}{M}\dot{\mathbf{r}} \\
\\
\dot{\mathbf{r}}_2=\dot{\mathbf{R}}-\dfrac{m_1}{M}\dot{\mathbf{r}}
\end{cases}\;,
\end{equation*}
dove si è posto $M=m_1+m_2$. Allora la Lagrangiana, nelle nuove coordinate, diventa:
\begin{align}
L &= \frac{1}{2}m_1\left(\dot{\mathbf{R}}+\frac{m_2}{M}\dot{\mathbf{r}}\right)^2+\frac{1}{2}m_2\left(\dot{\mathbf{R}}-\frac{m_1}{M}\dot{\mathbf{r}}\right)^2-V(\mathbf{r})\notag  \\
&= \frac{1}{2}m_1\dot{\mathbf{R}}^2+\frac{m_1m_2}{M}\dot{\mathbf{R}}\cdot\dot{\mathbf{r}}+\frac{m_1m_2^2}{2M^2}\dot{\mathbf{r}}^2+\frac{1}{2}m_2\dot{\mathbf{R}}^2-\frac{m_1m_2}{M}\dot{\mathbf{R}}\cdot\dot{\mathbf{r}}+\frac{m_2m_1^2}{2M^2}\dot{\mathbf{r}}^2-V(\mathbf{r})\notag \\
&=\frac{1}{2}(m_1+m_2)\dot{\mathbf{R}}^2+\frac{m_1m_2}{m_1+m_2}\dot{\mathbf{r}}^2-V(\mathbf{r})\;.
\end{align}
Poniamo $\mu\equiv m_1m_2/(m_1+m_2)$, detta \textit{massa ridotta}, ottenendo:
\begin{equation}
L=\frac{1}{2}M\dot{\mathbf{R}}^2+\frac{1}{2}\mu\dot{\mathbf{r}}^2-V(\mathbf{r})\;.
\end{equation}
Osserviamo che $L=L_{cdm}(\mathbf{R},\dot{\mathbf{R}})+L_{rel}(\mathbf{r},\dot{\mathbf{r}})$, dunque nel nuovo set di coordinate, riusciamo a scomporre il sistema in due sistemi disaccoppiati. \\

Consideriamo adesso la Lagrangiana del sistema del centro di massa (i.e. $\dot{\mathbf{R}}\equiv0$):
\begin{equation}
L_{rel}=\frac{1}{2}\mu\dot{\mathbf{r}}^2-V(\mathbf{r})\;.
\end{equation}
Notiamo che se $V$ è funzione del modulo di $\mathbf{r}$, allora non esistono direzioni privilegiate per il moto. Applichiamo adesso una rotazione infinitesima:
\begin{equation}
\begin{cases}
\mathbf{r}\longmapsto\mathbf{r}'=\mathbf{r}+\delta\mathbf{r} \\
\\
\delta\mathbf{r}=-\mathbf{n}\times\mathbf{r}\cdot\epsilon=\epsilon\mathbf{r}\times\mathbf{n} \\
\\
\delta\dot{\mathbf{r}}=\epsilon\dot{\mathbf{r}}\times\mathbf{n}
\end{cases}\;,
\end{equation}
e scriviamo la variazione della Lagrangiana:
\begin{align}
\delta L_{rel} &=\pdev{L_{rel}}{r_{\alpha,k}}\delta r_{\alpha,k}+\pdev{L_{rel}}{\dot{r}_{\alpha,k}}\delta\dot{r}_{\alpha,k} \notag \\
&=\pdev{L_{rel}}{r_{\alpha,k}}\delta r_{\alpha,k}+\pdev{L_{rel}}{\dot{r}_{\alpha,k}}\frac{\diff}{\diff{t}}\delta r_{\alpha,k} \notag \\
&=\left(\frac{\diff}{\diff{t}}\pdev{L_{rel}}{\dot{r}_{\alpha,k}}\right)\delta r_{\alpha,k}+\pdev{L_{rel}}{r_{\alpha,k}}\left(\frac{\diff}{\diff{t}}\delta r_{\alpha,k}\right) \notag \\
&= \frac{\diff}{\diff{t}}\left(\pdev{L_{rel}}{\dot{r}_{\alpha,k}}\delta r_{\alpha,k}\right)\;.
\end{align}
Allora se $\delta L_{rel}=0$, i.e. la Lagrangiana è invariante per rotazioni, la quantità:
\begin{equation}
\pdev{L_{rel}}{\dot{r}_{\alpha,k}}\delta r_{\alpha,k}
\end{equation}
è un integrale primo del moto. Si ha in particolare:
\begin{equation}
\pdev{L_{rel}}{\dot{r}_{\alpha,k}}\delta r_{\alpha,k}=\mu\dot{r}_{\alpha,k}\epsilon(\mathbf{r}^a\times\mathbf{n})_k=\;\mbox{costante}\;.
\end{equation}
da cui, per l'arbitrarietà di $\epsilon$, si conclude che:
\begin{equation}
\mu\dot{r}_{\alpha,k}(\mathbf{r}^a\times\mathbf{n})_k=\;\mbox{costante}\;.
\end{equation}
Allora:
\begin{equation}
\mu\dot{r}_{\alpha,k}(\mathbf{r}^a\times\mathbf{n})_k=\mu\dot{r}_{\alpha,k}\epsilon_{kij}r_{\alpha,i} n_j=n_j\epsilon_{jki}(\mu\dot{r}_{\alpha,k})r_{\alpha,i}=-n_j(\epsilon_{jik}r_{\alpha,i} \mu\dot{r}_{\alpha,k})=-\mathbf{n}\cdot\mathbf{L}\;,
\end{equation}
dove:
\begin{equation}
\mathbf{L}\equiv \mathbf{r}\times\mu\dot{\mathbf{r}}
\end{equation}
è il momento angolare totale del sistema. \\

Data dunque la Lagrangiana:
\begin{equation}
L_{rel}=\frac{1}{2}\mu\dot{\mathbf{r}}^2-V(|\mathbf{r}|)\;,
\end{equation}
e la variazione infinitesima $\delta\mathbf{r}=\mathbf{n}\times\mathbf{r}\delta\varphi$, sappiamo che sotto queste condizioni $\delta\lag_{rel}=\delta\phi n_a(\epsilon_{abc}r_b\mu\dot{r}_c)=\delta\phi\mathbf{n}\cdot\mathbf{L}=0$. Calcoliamo adesso:
\begin{equation}
\mathbf{L}\cdot\mathbf{r}(t)=r_aL_a=\epsilon_{abc}r_ar_b\mu\dot{r}_c=0 \qquad \forall t\;.
\end{equation}
Deduciamo dunque che, fissato $\mathbf{L}$, l'orbita in ogni istante è perpendicolare a $\mathbf{L}$, dunque il moto è piano. Calcoliamo adesso l'area spazzata dal raggio vettore tra gli istanti $t$ e $t+\delta t$:
\begin{equation}
\delta A=r\cdot r\dev{\varphi}{t}\delta t=r^2\dot{\varphi}\delta t\;.
\end{equation}
Definiamo dunque la \textit{velocità areolare}:
\begin{equation}
\dev{A}{t}=r^2\dot{\varphi}\;.
\end{equation}
Poiché, in coordinate sferiche, $|\mathbf{L}|=\mu r^2\dot{\varphi}$, possiamo esprimere la velocità areolare in termini del momento angolare come:
\begin{equation}
\dev{A}{t}=\frac{|\mathbf{L}|}{\mu}\;.
\end{equation}
Scriviamo adesso la Lagrangiana $L_{rel}$ in coordinate sferiche:
\begin{equation}
L_{rel}=\frac{1}{2}\mu(\dot{r}^2+r^2\dot{\theta}^2+r^2\sin^2\theta\dot{\varphi}^2)-V(r)\;.
\end{equation}
Fissiamo $\theta=\pi/2$ e $\dot{\theta}=0$:
\begin{equation}
L_{rel}=\frac{1}{2}\mu(\dot{r}^2+r^2\dot{\varphi}^2)-V(r)\;.
\end{equation}
Se sostituiamo adesso $\dot{\varphi}=L/(\mu r^2)$, otteniamo:
\begin{equation}
L_{rel}=\frac{1}{2}\mu\dot{r}^2+\frac{L^2}{2\mu r^2}-V(r)\;.
\end{equation}
Tuttavia questa espressione risulta sostanzialmente errata in quanto le variazioni che abbiamo effettuato non sono indipendenti. Passiamo dunque a considerare l'energia:
\begin{equation}
E=\frac{1}{2}(\mu\dot{r}^2+\mu r^2\dot{\varphi}^2)+V(r)\;.
\end{equation}
Possiamo adesso sostituire $\dot{\varphi}=L^2/(\mu r^2)$, ottenendo:
\begin{equation}
E=\frac{1}{2}\mu\dot{r}^2+\frac{L^2}{2\mu r^2}+V(r)\equiv \frac{1}{2}\mu\dot{r}^2+V_{\mathrm{eff}}(r)\;,
\end{equation}
dove:
\begin{equation}
V_{\mathrm{eff}}=\frac{L^2}{2\mu r^2}+V(r)\;,
\end{equation}
è detto \textit{potenziale efficace}.
\subsection{Potenziale gravitazionale}
Fissiamo adesso $V(r)=-k/r$, $k>0$. Questa è la tipica espressione di un potenziale di tipo gravitazionale. Si ha dunque:
\begin{equation}
\dot{\varphi}=\frac{L}{\mu r^2}\qquad \Longrightarrow\qquad \diff{\varphi}=\frac{L}{\mu r^2}\diff{t}\;,
\end{equation}
ossia:
\begin{equation}
\frac{\mu r^2}{L}\diff{\varphi}=\diff{t}\qquad \Longleftrightarrow\qquad \frac{L}{\mu r^2}\frac{\diff}{\diff{\varphi}}=\frac{\diff}{\diff{t}}\;.
\end{equation}
Poniamo $u=1/r$, ottenendo:
\begin{equation}
\frac{1}{u^2}\frac{\diff}{\diff[t]}=\frac{L}{\mu}\frac{\diff}{\diff[\varphi]}\;.
\end{equation}
Allora risulta:
\begin{equation}
\dev{r}{t}=-\frac{1}{u^2}\dev{u}{t}=-\frac{L}{\mu}\dev{u}{\varphi}\;.
\end{equation}
Scriviamo adesso l'energia in funzione di $u$:
\begin{align}
E&=\frac{1}{2}\mu\left(\dev{r}{t}\right)^2+\frac{L^2}{2\mu r^2}-\frac{k}{r} \notag \\
&= \frac{1}{2}\mu \frac{L^2}{\mu^2}\left(\dev{u}{\varphi}\right)^2+\frac{L^2}{2\mu}u^2-ku \notag \\
&= \frac{L^2}{2\mu}\left[\left(\dev{u}{\varphi}\right)^2+u^2\right]-ku\;.
\end{align}
Imponendo $\diff[E]/\diff[\varphi]=0$, otteniamo l'equazione:
\begin{equation}
\dev[2]{u}{\varphi}+u=\frac{\mu k}{L^2}\;,
\end{equation}
che è l'equazione di un oscillatore armonico soggetto ad una forza esterna costante. Le soluzioni saranno date da:
\begin{equation}
u(\varphi)=A\cos(\varphi+\varphi_0)+\frac{\mu k}{L^2}\;.
\end{equation}
Possiamo fissare $\varphi_0=0$ e sostituire $u=1/r$:
\begin{equation}
\frac{1}{r(\varphi)}=\frac{1}{p}+A\cos\varphi\;, \label{sec8_orbit}
\end{equation}
dove $p=L^2/(\mu k)$. Osserviamo che si ha $r(\varphi)=r(\varphi+2\pi)$, quindi l'orbita è chiusa e il moto è periodico. Riscriviamo la \eqref{sec8_orbit} nella forma:
\begin{equation}
\frac{p}{r}=pA\cos\varphi +1\;.
\end{equation}
Poniamo $pA=e$:
\begin{equation}
p=r+re\cos\varphi\;.
\end{equation}
In coordinate Cartesiane si ha:
\begin{align}
p &=\sqrt{x^2+y^2}+ex \notag \\
p-ex &= \sqrt{x^2+y^2} \notag \\
p^2-2pex+e^2x^2&=x^2+y^2\;.
\end{align}
Ovvero:
\begin{equation}
p^2=(1-e^2)x^2+2pex+y^2=(1-e^2)\left[x^2+2\frac{pe}{1-e^2}x+\left(\frac{ep}{1-e^2}\right)^2\right]-\frac{e^2p^2}{1-e^2}+y^2\;,
\end{equation}
dove abbiamo aggiunto e sottratto la quantità $e^2 p^2/(1-e^2)$ per completare il quadrato:
\begin{equation}
\left(1+\frac{e^2}{1-e^2}\right)p^2=(1-e^2)\left(x+\frac{ep}{1-e^2}\right)^2+y^2\;.
\end{equation}
Posto $x_C=-e p/(1-\varepsilon^2)$ otteniamo:
\begin{equation}
(1-e^2)(x-x_C)^2+y^2=\frac{p^2}{1-e^2}\;,
\end{equation}
cioè:
\begin{equation}
\frac{(x-x_C)^2}{\dfrac{p^2}{(1-e^2)^2}}+\frac{y^2}{\dfrac{p^2}{1-e^2}}=1\;,
\end{equation}
che, a seconda del valore di $e$, detto \textit{eccentricità}, può rappresentare le seguenti orbite:
\begin{equation}
\begin{cases}
e <1 \Longrightarrow \;\mbox{l'orbita è un'ellisse} \\
\\
e = 0 \Longrightarrow\;\mbox{l'orbita è una circonferenza} \\
\\
e >1 \Longrightarrow\;\mbox{l'orbita è un'iperbole}
\end{cases}\;.
\end{equation}
Adesso costruiamo un'altra costante del moto nel caso del potenziale gravitazionale. Sappiamo che:
\begin{equation}
E=\frac{\mathbf{p}^2}{2m}-\frac{k}{r}=\;\mbox{costante}\;.
\end{equation}
Definiamo il \textit{vettore di Lenz}:
\begin{equation}
\mathbf{A}=\mathbf{p}\times\mathbf{L}+km\frac{\mathbf{r}}{r}\;.
\end{equation}
Osserviamo innanzitutto che il vettore di Lenz è perpendicolare al momento angolare:
\begin{equation}
\mathbf{A}\cdot\mathbf{L}=\mathbf{p}\times\mathbf{L}\cdot\mathbf{L}+\frac{km}{r}\mathbf{r}\cdot\mathbf{L}=0\;,
\end{equation}
in quanto $\mathbf{p}\times \mathbf{L}\perp \mathbf{L}$ per definizione di prodotto vettoriale e $\mathbf{r}\perp \mathbf{L}$ per lo stesso motivo, in quanto $\mathbf{L}=\mathbf{r}\times\mathbf{p}$. Calcoliamo adesso il modulo quadro di $\mathbf{A}$:
\begin{align}
A^2&= p^2L^2+k^2m^2+2\frac{mk}{r}\mathbf{r}\cdot(\mathbf{p}\times\mathbf{L}) \notag \\
&= 2m\left(E-\frac{k}{r}\right)L^2+k^2m^2+\frac{2km}{r}\mathbf{L}\cdot (\mathbf{r}\times\mathbf{p}) \notag \\
&= 2mEL^2-\frac{2mkL^2}{r}+k^2m^2+\frac{2kmL^2}{r}=2mEL^2+k^2m^2\;.
\end{align}
Consideriamo infine la quantità:
\begin{align}
\frac{\diff}{\diff{t}}(\mathbf{p}\times\mathbf{L}) &= \dot{\mathbf{p}}\times\mathbf{L}=\mathbf{F}\times \mathbf{L}=-\frac{k}{r^3}\mathbf{r}\times\mathbf{L} \notag \\
&=-\frac{km}{r^3}\mathbf{r}\times(\mathbf{r}\times\dot{\mathbf{r}})=-\frac{km}{r^3}\left(r^2\dot{\mathbf{r}}-(\mathbf{r}\cdot\dot{\mathbf{r}})\mathbf{r}\right) \notag \\
&= -km\left(\frac{\dot{\mathbf{r}}}{r}-\mathbf{r}\frac{\dot{r}}{r^2}\right)=-km\frac{\diff}{\diff{t}}\left(\frac{\mathbf{r}}{r}\right)\;.
\end{align}
Allora:
\begin{equation}
0=\frac{\diff}{\diff{t}}(\mathbf{p}\times\mathbf{L})+km\frac{\diff}{\diff{t}}\left(\frac{\mathbf{r}}{r}\right)=\frac{\diff}{\diff{t}}\left(\mathbf{p}\times\mathbf{L}+km\frac{\mathbf{r}}{r}\right)\equiv \dev{\mathbf{A}}{t}\;,
\end{equation}
e questo dimostra che il vettore di Lenz è una costante del moto.
\pagebreak
\section{Trasformazioni}
Consideriamo un generico sistema descritto in un sistema di coordinate generalizzate $q=\{q_1,\ldots,q_N\}$ dalla Lagrangiana $L(q,\dot{q},t)$. Le equazioni del moto si otterrano dalle equazioni di Eulero-Lagrange:
\begin{equation}
\frac{\diff}{\diff{t}}\pdev{L}{\dot{q}}-\pdev{L}{q}=0\;.
\end{equation}
In un secondo set di coordinate generalizzate $Q=\{Q_1,\ldots,Q_N\}$, legato al primo dalla relazione $q=q(Q,t)$, si ha:
\begin{equation}
L'(Q,\dot{Q},t)\qquad \implies\qquad \frac{\diff}{\diff{t}}\pdev{L'}{\dot{Q}}-\pdev{L'}{Q}=0\;.
\end{equation}
Inoltre, per il principio di invarianza in valore:
\begin{equation}
L'(Q,\dot{Q},t)=L(q(Q,t),\dot{q}(Q,\dot{Q},t),t)\;.
\end{equation}
Allora si ha:
\begin{align}
\frac{\diff}{\diff{t}}\pdev{L'}{\dot{Q}_i}-\pdev{L'}{Q_i} &= \frac{\diff}{\diff{t}}\left(\pdev{L}{\dot{q}_k}\pdev{\dot{q}_k}{\dot{Q}_i}\right)-\pdev{L}{q_k}\pdev{q_k}{Q_i}-\pdev{L}{\dot{q}_k}\pdev{\dot{q}_k}{Q_i} \notag \\
&= \left(\frac{\diff}{\diff{t}}\pdev{L}{\dot{q}_k}\right)\pdev{\dot{q}_k}{\dot{Q}_i}+\pdev{L}{\dot{q}_k}\frac{\diff}{\diff{t}}\pdev{\dot{q}_k}{\dot{Q}_i}-\pdev{L}{q_k}\pdev{q_k}{Q_i}-\pdev{L}{q_k}\pdev{\dot{q}_k}{Q_i}\;.
\end{align}
Ricordando che $\partial\dot{q}_k/\partial\dot{Q}_i=\partial q_k/\partial Q_i$, si ottiene:
\begin{equation}
\frac{\diff}{\diff{t}}\pdev{L'}{\dot{Q}_i}-\pdev{L'}{Q_i} = \left(\frac{\diff}{\diff{t}}\pdev{L}{\dot{q}_k}-\pdev{L}{q_k}\right)\pdev{q_k}{Q_i}+\pdev{L}{\dot{q}_k}\left(\frac{\diff}{\diff{t}}\pdev{q_k}{Q_i}-\frac{\partial}{\partial Q_i}\dev{q_k}{t}\right)\;.
\end{equation}
Il secondo termine è identicamente nullo per la commutatività delle derivate. Allora otteniamo la relazione tra le equazioni di Eulero-Lagrange generate da $L$ e quelle generate da $L'$:
\begin{equation}
\frac{\diff}{\diff{t}}\pdev{L'}{\dot{Q}_i}-\pdev{L'}{Q_i}=\pdev{q_k}{Q_i}\left(\frac{\diff}{\diff{t}}\pdev{L}{\dot{q}_k}-\pdev{L}{q_k}\right)\;.
\end{equation}
$\partial q_k/\partial Q_i$ rappresenta l'elemento di posto $(k,i)$ della matrice Jacobiana della trasformazione. Poiché la trasformazione può essere invertita, la Jacobiana risulterà invertibile.
\pagebreak
\section{Formalismo Hamiltoniano}
Si definisce \textit{Hamiltoniana} la funzione:
\begin{equation}
H(q,\dot{q},p,t)\equiv p\dot{q}-L(q,\dot{q},t)\;.
\end{equation}
Vogliamo adesso che $H$ non dipenda dalle velocità generalizzare $\dot{q}$: differenziando, si ha:
\begin{align}
\diff{H} &= \dot{q}\diff{p}+p\diff{\dot{q}}-\pdev{L}{q}\diff{q}-\pdev{L}{\dot{q}}\diff{\dot{q}}-\pdev{L}{t}\diff{t} \notag \\
&=\left(p-\pdev{L}{\dot{q}}\right)\diff{\dot{q}}+\dot{q}\diff{p}-\pdev{L}{q}\diff{q}-\pdev{L}{t}\diff{t}\;.
\end{align}
Se imponiamo $p=\partial L/\partial\dot{q}$, $H$ non dipende da $\dot{q}$. Si ha dunque:
\begin{align}
\diff{H} &= \dot{q}\diff{p}-\pdev{L}{q}\diff{q}-\pdev{L}{t}\diff{t}\;, \\
\diff{H} &= \pdev{H}{q}\diff{q}+\pdev{H}{p}\diff{p}+\pdev{H}{t}\diff{t}\;.
\end{align}
Dal confronto si ottengono pertanto le seguenti relazioni:
\begin{align}
\pdev{H}{t} &=-\pdev{L}{t}\;, \\
\pdev{H}{q} &=-\pdev{L}{q}=-\frac{\diff}{\diff{t}}\pdev{L}{\dot{q}}=-\dev{p}{t}=-\dot{p}\;, \\
\pdev{H}{p} &= \dot{q}\;.
\end{align}
Da queste ricaviamo le \textit{equazioni di Hamilton}:
\begin{align}
&\dot{q}_k=\pdev{H}{p_k}, &\dot{p}_k=-\pdev{H}{q_k}\,.
\end{align}
Il vantaggio del formalismo Hamiltoniano consiste nel dover risolvere, per un sistema con $n$ gradi di libertà, $2n$ equazioni differenziali del prim'ordine anziché $n$ del secondo ordine del formalismo Lagrangiano.
\subsection{La funzione di Routh}
In certi casi è conveniente, nel cambio di variabili, sostituire solo alcune delle velocità generalizzate con i rispettivi momenti, invece che tutte. Supponiamo di avere solo due coordinate $q$ e $\xi$ e trasformiamo le variabili $q,\xi,\dot{q},\dot{\xi}$ in $q,\xi,p,\dot{\xi}$, dove $p$ è il momento generalizzato corrispondente a $q$. Il differenziale della Lagrangiana è:
\begin{align}
\diff{L} &= \pdev{L}{q}\diff{q}+\pdev{L}{\dot{q}}\diff{\dot{q}}+\pdev{L}{\xi}\diff{\xi}+\pdev{L}{\dot{\xi}}\diff{\dot{\xi}} \notag \\
&=\dot{p}\diff{q}+p\diff{\dot{q}}+\pdev{L}{\xi}\diff{\xi}+\pdev{L}{\dot{\xi}}\diff{\dot{\xi}}\;.
\end{align}
Quindi:
\begin{equation}
\diff{(L-p\dot{q})}=\dot{p}\diff{q}-\dot{q}\diff{p}+\pdev{L}{\xi}\diff{\xi}+\pdev{L}{\dot{\xi}}\diff{\dot{\xi}}\;.
\end{equation}
Definiamo la \emph{Routhiana} come:
\begin{equation}
R(q,p,\xi,\dot{\xi})\equiv p\dot{q}-L\;,
\end{equation}
dove la velocità $\dot{q}$ è espressa in termini di $p$ tramite l'equazione $p=\partial L/\partial\dot{q}$. Il differenziale di $R$ è:
\begin{equation}
\diff{R}=-\dot{p}\diff{q}+\dot{q}\diff{p}-\pdev{L}{\xi}\diff{\xi}-\pdev{L}{\dot{\xi}}\diff{\dot{\xi}}\;.
\end{equation}
Otteniamo pertanto:
\begin{align}
&\dot{q}=\pdev{R}{p}, &\dot{p}=-\pdev{R}{q}\;, \\
&\pdev{L}{\xi} =-\pdev{R}{\xi}, &\pdev{L}{\dot{\xi}}=-\pdev{R}{\dot{\xi}}\;.
\end{align}
Se sostituiamo queste equazioni nella Lagrangiana per la coordinate $\xi$, abbiamo:
\begin{equation}
\frac{\diff}{\diff{t}}\pdev{R}{\dot{\xi}}=\pdev{R}{\xi}\;.
\end{equation}
Pertanto la Routhiana è un'Hamiltoniana rispetto alla coordinata $q$ e una Lagrangiana rispetto alla coordinata $\xi$. L'energia del sistema è:
\begin{equation}
E=\dot{q}\pdev{L}{\dot{q}}+\dot{\xi}\pdev{L}{\dot{\xi}}-L=p\dot{q}+\dot{\xi}\pdev{L}{\dot{\xi}}-L\;.
\end{equation}
In termini della Routhiana:
\begin{equation}
E=R-\dot{\xi}\pdev{R}{\dot{\xi}}\;.
\end{equation}
L'uso della Routhiana è conveniente in particolare quando alcune delle coordinate sono cicliche.
\section{Trasformazioni canoniche infinitesime}
Consideriamo una generica Hamiltoniana $H(q,p,t)$ e la trasformazione:
\begin{equation}
\begin{cases}
q_i\longmapsto Q_i\equiv q_i+\epsilon f_i(q,p,t) \\
\\
p_i \longmapsto P_i\equiv p_i+\epsilon g_i(q,p,t)
\end{cases}\;.
\end{equation}
Sia infine $K(Q,P,t)$ l'Hamiltoniana scritta in termini delle nuove coordinate. Vogliamo ricavare delle condizioni sulle funzioni $h(q,p,t)$ tali che:
\begin{align}
H(q,p,t)+\epsilon h(q,p,t) &= K(Q,P,t)\;, \notag \\
q &\equiv q(Q,P,t)\;, \notag \\
p &\equiv p(Q,P,t)\;.
\end{align}
Si ha innanzitutto che:
\begin{equation}
h(q,p,t)=\frac{1}{\epsilon}[K(Q,P,t)-H(q,p,t)]\;.
\end{equation}
Deriviamo rispetto a $q_i$:
\begin{align}
\pdev{h}{q_i} &= \frac{1}{\epsilon}\left(\pdev{K}{Q_k}\pdev{Q_k}{q_i}+\pdev{K}{p_k}\pdev{P_k}{q_i}-\pdev{H}{q_i}\right) \notag \\
&= \frac{1}{\epsilon}\left(-\dot{P}_k\pdev{Q_k}{q_i}+\dot{Q}_k\pdev{P_k}{q_i}+\dot{p}_i\right) \notag \\
&=\frac{1}{\epsilon}\left[-\left(\dot{p}_k+\epsilon\dev{g_k}{t}\right)\left(\delta_{ki}+\epsilon\pdev{f_k}{q_i}\right)+\left(\dot{q}_k+\epsilon\dev{f_k}{t}\right)\epsilon\pdev{g_k}{q_i}+\dot{p}_i\right] \notag \\
&= \frac{1}{\epsilon}\left(-\dot{p}_i-\epsilon\dot{p}_k\pdev{f_k}{q_i}-\epsilon\dev{g_i}{t}-\epsilon^2\dev{g_k}{t}\pdev{f_k}{q_i}+\epsilon\dot{q}_k\pdev{g_k}{q_i}+\epsilon^2\dev{f_k}{t}\pdev{g_k}{q_i}+\dot{p}_i\right)
\end{align}
Semplificando e trascurando i termini in $\epsilon^2$, rimaniamo con:
\begin{align}
\pdev{h}{q_i}&\simeq -\dot{p}_k\pdev{f_k}{q_i}-\dev{g_i}{t}+\dot{q}_k\pdev{g_k}{q_i} \notag \\
&=-\pdev{g_i}{t}-\pdev{g_i}{q_k}\dot{q}_k-\pdev{g_i}{p_k}\dot{p}_k-\pdev{f_k}{g_i}\dot{p}_k+\pdev{g_k}{q_i}\dot{q}_k\;.
\end{align}
Dunque si ha 
\begin{equation}
\pdev{h}{q_i}\simeq -\pdev{g_i}{t}+\dot{q}_k\left(\pdev{g_k}{q_i}-\pdev{g_i}{q_k}\right)-\dot{p}_k\left(\pdev{g_i}{p_k}+\pdev{f_k}{q_i}\right)\;,
\end{equation}
e similmente
\begin{equation}
\pdev{h}{p_i}\simeq \pdev{f_i}{t}+\dot{q}_k\left(\pdev{f_i}{q_k}+\pdev{g_k}{q_i}\right)+\dot{p}_k\left(\pdev{f_k}{p_i}-\pdev{f_i}{q_k}\right)\;.
\end{equation}
Per garantire la compatibilità delle soluzioni, imponiamo le condizioni:
\begin{equation}
\begin{cases}
\partial g_k/\partial q_i=\partial g_i/\partial q_k \\
\\
\partial f_k/\partial p_i=\partial f_i/\partial p_k \\
\\
\partial g_i/\partial p_k+\partial f_k/\partial q_i=0
\end{cases}\qquad \Longrightarrow \qquad
\begin{cases}
g_i=\partial g/\partial q_i \\
\\
f_i=\partial f/\partial p_i
\end{cases}\;,
\end{equation}
per ogni $i,k$. Si ha inoltre:
\begin{equation}
\frac{\partial^2 f}{\partial p_k\partial q_i}+\frac{\partial^2 g}{\partial q_i\partial p_k}=0\;,
\end{equation}
da cui segue che deve essere necessariamente $g=-f$. Si ottengono dunque le condizioni:
\begin{equation}
\begin{cases}
\dfrac{\partial h}{\partial q_i}=\dfrac{\partial}{\partial t}\dfrac{\partial f}{\partial q_i} \\
\\
\dfrac{\partial h}{\partial p_i}=\dfrac{\partial}{\partial t}\dfrac{\partial f}{\partial p_i}
\end{cases}\qquad \implies \qquad h=\pdev{f}{t}\;.
\end{equation}
Allora le \textit{trasformazioni canoniche infinitesime} saranno della forma:
\begin{equation}
H(q,p,t)+\epsilon\pdev{f}{t}(q,p,t)=K(Q,P,t)\;.
\end{equation}
Vogliamo adesso sapere quale trasformazione canonica si ottiene scegliendo $f(q,p,t)\equiv H(q,p,t)$. In questo caso, si ottiene:
\begin{equation}
\begin{cases}
q_k\longmapsto q_k+\epsilon\dfrac{\partial H}{\partial p_k}=q_k(t)+\epsilon\dot{q}_k\simeq q_k(t+\epsilon) \\
\\
p_k\longmapsto p_k-\epsilon\dfrac{\partial H}{\partial q_k}=p_k(t)+\epsilon\dot{p}_k\simeq p_k(t+\epsilon)
\end{cases}\;.
\end{equation}
Allora
\begin{equation}
K(Q,P,t) = \begin{cases}
H(q,p,t)+\epsilon\dfrac{\partial f}{\partial t} \\
\\
K\left(q+\epsilon\dfrac{\partial f}{\partial p},p-\epsilon\dfrac{\partial f}{\partial q},t\right)=H(q,p,t)+\epsilon\dfrac{\partial H}{\partial q_k}\dfrac{\partial f}{\partial p_k}-\epsilon\dfrac{\partial H}{\partial p_k}\dfrac{\partial f}{\partial q_k}
\end{cases}\;.
\end{equation}
Dal confronto otteniamo che:
\begin{align}
&\pdev{f}{t}=\pdev{H}{q_k}\pdev{f}{p_k}-\pdev{H}{p_k}\pdev{f}{q_k}=-\dot{p}_k\pdev{f}{p_k}-\dot{q}_k\pdev{f}{q_k}\;,
\end{align}
da cui:
\begin{equation}
\pdev{f}{t}+\pdev{f}{q_k}\dot{q}_k+\pdev{f}{p_k}\dot{p}_k\equiv \dev{f}{t}=0\;.
\end{equation}
In particolare, si osserva che tutte le funzioni generatrici di trasformazioni canoniche sono costanti del moto.
\section{Parentesi di Poisson}
Sia $f(p,q,t)$ una funzione delle coordinate, momenti e tempo. La sua derivata totale rispetto al tempo è data da:
\begin{equation}
\dev{f}{t}=\pdev{f}{t}+\sum_k\left(\pdev{f}{q_k}\dot{q}_k+\pdev{f}{p_k}\dot{p}_k\right)\;.
\end{equation}
Sostituendo le equazioni di Hamilton per $\dot{q}_k,\dot{p}_k$ otteniamo:
\begin{equation}
\dev{f}{t}=\pdev{f}{t}+\{H,f\}\;,
\end{equation}
dove:
\begin{equation}
\{H,f\}\equiv \sum_k\left(\pdev{H}{p_k}\pdev{f}{q_k}-\pdev{H}{q_k}\pdev{f}{p_k}\right)
\end{equation}
è detta \emph{parentesi di Poisson} delle quantità $H$ e $f$. Osserviamo che una quantità $f$ è un intergrale primo del moto se:
\begin{equation}
\dev{f}{t}=\pdev{f}{t}+\{H,f\}=0\;.
\end{equation}
In particolare, se l'integrale primo non dipende esplicitamente dal tempo, la condizione diventa:
\begin{equation}
\{H, f\}=0\;.
\end{equation}
Per qualunque coppia di funzioni $f,g$ la parentesi di Poisson è definita in modo simile:
\begin{equation}
\{f,g\} \equiv \sum_k\left(\pdev{f}{p_k}\pdev{g}{q_k}-\pdev{f}{q_k}\pdev{g}{p_k}\right)\;.
\end{equation}
Le parentesi di Poisson godono delle seguenti proprietà:
\begin{itemize}
\item $\{f,g\}=-\{g,f\}$;
\item se $c$ è una costante $\{f,c\}=0$;
\item sono lineari in entrambi gli argomenti;
\item $\{f_1f_2,g\}=f_1\{f_2,g\}+f_2\{f_1,g\}$.
\end{itemize}
Prendendo la derivata parziale della parentesi di Poisson rispetto al tempo, troviamo:
\begin{equation}
\frac{\partial}{\partial t}\{f,g\}=\left\{\pdev{f}{t},g\right\}+\left\{f,\pdev{g}{t}\right\}\,.
\end{equation}
Se una delle funzioni $f,g$ è uno dei momenti o delle coordinate, la parentesi di Poisson si riduce ad una derivata parziale:
\begin{align}
\{f,q_k\} &= \pdev{f}{p_k}\;, \\
\{f,p_k\} &= -\pdev{f}{q_k}\;.
\end{align}
Scegliendo nelle formule precedenti $f=q_i$ o $f=p_i$, otteniamo le \emph{parentesi di Poisson} canoniche:
\begin{align}
\{q_i,q_k\} =\{p_i,p_k\} &=0\;, \\
\{p_i,q_k\} &=\delta_{ik}\;.
\end{align}
Inoltre, vale la seguente relazione, chiamata \emph{identità di Jacobi}:
\begin{equation}
\{f,\{g,h\}\}+\{g,\{h,f\}\}+\{h,\{f,g\}\}=0\;.
\end{equation}
Un'altra importante proprietà delle parentesi di Poisson è il \emph{teorema di Poisson}: se $f,g$ sono due integrali del moto, allora la loro parentesi di Poisson è ancora un integrale del moto. Se $f,g$ non dipendono dal tempo, allora fissando $h=H$ nell'identità di Jacobi si ottiene:
\begin{equation}
\{H,\{f,g\}\}+\{f,\{g,H\}\}+\{g,\{H,f\}\}=0\;.
\end{equation}
Quindi, se $\{H,g\}=\{H,f\}=0$ per ipotesi, allora $\{H,\{f,g\}\}=0$ e quindi la tesi è dimostrata. Se invece $f,g$ dipendono esplicitamente dal tempo, allora partiamo da:
\begin{align}
\frac{\diff}{\diff{t}}\{f,g\} &= \left\{\pdev{f}{t},g\right\}+\left\{f,\pdev{g}{t}\right\}-\{f,\{g,H\}\}-\{g,\{H,f\}\} \notag \\
&= \left\{\pdev{f}{t}+\{H,f\},g\right\}+\left\{f,\pdev{g}{t}+\{H,g\}\right\}\notag \\
&=\left\{\dev{f}{t},g\right\}+\left\{f,\dev{g}{t}\right\}=0\;.
\end{align}

\chapter{Relatività Speciale}
\section{Introduzione}
\subsection{Pippone sul perché la relatività è saltata fuori}
\lipsum
\subsection{Spaziotempo di Minkowski}
Un punto dello spaziotempo di Minkowski è detto \textit{evento}, e le traiettorie sono dette \textit{linee d'universo}. La bisettrice 
indica come si muove la luce, tutti gli altri eventi possono propagarsi nel doppio cono individuato dalle bisettrici. $V_+$ è denominato
 \textit{cono futuro} e $V_-$ invece \textit{cono passato}. La quantità:
\begin{equation}
\Delta\tau =\int_{t_A}^{t_B}\diff{t}\sqrt{1-\frac{\mathbf{v}^2}{c^2}}\;,
\end{equation}
è detta \textit{tempo proprio}.
\section{Trasformazioni di Lorentz}
\subsection{In 1+1 dimensioni}
Consideriamo due sistemi di riferimento inerziali, $K, K'$, con $K'$ che si muove rispetto a $K$ con velocità $\beta$. All'istante $t=0$,
 le origini dei due sistemi coincidono. Il versore dell'asse $ct'$ nel sistema $K$ è $\frac{1}{\sqrt{1+\beta^2}}(\beta,1)$; allora le 
coordinate in $K$ dei punti $A$ ed $E$ saranno:
\begin{align}
A &\equiv \frac{\tau}{\sqrt{1+\beta^2}}(\beta,1)\;, \\
E &\equiv -\frac{\tau}{\sqrt{1+\beta^2}}(\beta,1)\;.
\end{align}
Per costruzione, si ha:
$$
A=E+r(1,1)+s(-1,1)\;,
$$
da cui si ottiene:
\begin{align}
& r=\frac{1+\beta}{\sqrt{1+\beta^2}}\tau\;, & s=\frac{1-\beta}{\sqrt{1+\beta^2}}\tau\;,
\end{align}
e di conseguenza le coordinate di $R$ nel sistema di riferimento $K$ saranno:
\begin{equation}
R=\frac{\tau}{\sqrt{1+\beta^2}}(1,\beta)\;.
\end{equation}
$R$ pertanto giace su una retta simmetrica a $ct'$ rispetto alla bisettrice, che rappresenta l'asse $x'$ nel sistema $K$. Passando da 
$K'$ a $K$ avremo dunque le seguenti relazioni:
\begin{align}
A &= \begin{pmatrix}
t' \\
0
\end{pmatrix}\longmapsto a(\beta)\begin{pmatrix}
1 \\
\beta
\end{pmatrix}t\;, \notag \\
R &= \begin{pmatrix}
0 \\
x'
\end{pmatrix}\longmapsto b(\beta)\begin{pmatrix}
\beta \\
1
\end{pmatrix}x\;.
\end{align}
Se tali trasformazioni sono lineari, allora conoscendo una base dello spazio è possibile sapere come viene trasformato ogni punto. 
Intanto, osserviamo che mandano rette in rette. La linearità discende direttamente dall'omogeneità dello spazio. In virtù di ciò, 
possiamo riscrivere le trasformazioni come:
\begin{equation}
\begin{cases}
t' = a(\beta)t+b(\beta)\beta x \\
\\
x'=a(\beta)\beta t+b(\beta)x
\end{cases}\;. \label{ch2_lorentztransf1}
\end{equation}
Restano quindi da determinare $a(\beta),b(\beta)$. Osserviamo che se $K'$ si muove rispetto a $K$ con velocità $\beta$, allora $K$ si 
muoverà rispetto a $K'$ con velocità $-\beta$. Allora le equazioni \eqref{ch2_lorentztransf1} diventano:
\begin{equation}
\begin{cases}
t = a(-\beta)t'-b(\beta)\beta x' \\
\\
x = a(-\beta)\beta t'+b(-\beta)x
\end{cases}\;.
\end{equation}
Dato che mandando $\beta$ in $-\beta$ le trasformazioni devono coincidere, si conclude che $a,b$ sono funzioni pari di $\beta$, cioè:
\begin{align}
& \begin{cases}
  t'=a(|\beta|)t+b(|\beta|)\beta x \\
\\
x'=a(|\beta|)\beta t+b(|\beta|)x
 \end{cases}\;,
&\begin{cases}
 t=a(|\beta|)t'-b(|\beta|)\beta x' \\
\\
x=-a(|\beta|)\beta t'+b(|\beta|)x'
\end{cases}\;.
\end{align}
In forma matriciale:
\begin{equation}
\begin{pmatrix}
t' \\
x
\end{pmatrix}=\begin{pmatrix}
a & \beta b \\
\beta a & b
\end{pmatrix}\begin{pmatrix}
t \\
x
\end{pmatrix} = \begin{pmatrix}
a & \beta b \\
\beta a & b
\end{pmatrix}\begin{pmatrix}
a & -\beta b \\
-\beta a & b
\end{pmatrix}\begin{pmatrix}
t' \\
x'
\end{pmatrix}\;.
\end{equation}
Allora dovrà essere:
\begin{equation}
\begin{pmatrix}
a & \beta b \\
\beta a & b
\end{pmatrix}\begin{pmatrix}
a & -\beta b \\
-\beta a & b
\end{pmatrix}=\begin{pmatrix}
a^2-\beta^2ab & -\beta ab+\beta b^2 \\
\beta a^2-\beta ab & -\beta^2ab+b^2
\end{pmatrix}=\begin{pmatrix}
1 & 0 \\
0 & 1
\end{pmatrix}\;.
\end{equation}
Da ciò si ricava:
\begin{equation}
\begin{cases}
\beta b(a-b) =0\quad \Longrightarrow\quad a=b \\
\\
a^2(1-\beta^2)=1\quad \Longrightarrow\quad a^2=\dfrac{1}{1-\beta^2}
\end{cases}\;.
\end{equation}
Dunque, posto $\beta=v/c$, si ha:
\begin{equation}
a=b=\frac{1}{\sqrt{1-v^2/c^2}}\equiv \gamma(v)\;.
\end{equation}
In conclusione, le equazioni:
\begin{equation}
\begin{cases}
 x'=\gamma(x+\beta t) \\
\\
t'=\gamma(t+\beta x)
\end{cases}\;, \label{ch2_lorentztransf2}
\end{equation}
rappresentano le \textit{trasformazioni di Lorentz ortocrone e proprie in 1+1 dimensioni}. In forma matriciale, le trasformazioni 
saranno:
\begin{equation}
L= \begin{pmatrix}
\gamma & \beta\gamma \\
\beta\gamma & \gamma
\end{pmatrix}\;.
\end{equation}
Ricaviamo infine la legge di composizione delle velocità. Eseguiamo il differenziale delle trasformazioni \eqref{ch2_lorentztransf2}:
\begin{equation}
\begin{cases}
\diff{x'}=\gamma(\diff{x}+\beta\diff{t}) \\
\\
\diff{t'}=\gamma(\diff{t}+\beta\diff{x})
\end{cases}\;,
\end{equation}
e dunque:
\begin{equation}
\dev{x'}{t'}=v'=\frac{\diff{x}+\beta\diff{t}}{\diff{t}+\beta\diff{x}}=\frac{\diff{x}/\diff{t}+u}{1+u\diff{x}/\diff{t}}=\frac{u+v}{1+uv/c^2}\;.
\end{equation}
La quantità conservata nella metrica di Minkowski è la \textit{lunghezza di Minkowski} $\sqrt{(\Delta x)^2-(c\Delta t)^2}$. \\
Osserviamo che $\det L=\gamma^2-\beta^2\gamma^2=\gamma^2(1-\beta^2)=1$. Allora esisterà un parametro $\vartheta$, detto \textit{rapidità}, tale che:
\begin{equation}
\begin{cases}
\gamma=\cosh\vartheta \\
\\
\beta\gamma=\sinh\vartheta
\end{cases}\;.
\end{equation}
Allora le trasformazioni di Lorentz possono essere scritte nella forma:
\begin{equation}
\begin{pmatrix}
\diff{t'} \\
\\
\diff{x'}
\end{pmatrix}=\begin{pmatrix}
\cosh\vartheta & \sinh\vartheta \\
\\
\sinh\vartheta & \cosh\vartheta
\end{pmatrix}\begin{pmatrix}
\diff{t} \\
\\
\diff{x}
\end{pmatrix}\;.
\end{equation}
Verifichiamo se vi sono analogie con le rotazioni in $\mathbb{R}^2$. Se applichiamo due trasformazioni con rapidità $\vartheta_1$ e 
$\vartheta_2$, si ha:
\begin{align}
\begin{pmatrix}
t' \\
\\
x'
\end{pmatrix} &=\begin{pmatrix}
\cosh\vartheta_1 & \sinh\vartheta_1 \\
\\
\sinh\vartheta_1 & \cosh\vartheta_1
\end{pmatrix}\begin{pmatrix}
\cosh\vartheta_2 & \sinh\vartheta_2 \\
\\
\sinh\vartheta_2 & \cosh\vartheta_2
\end{pmatrix}\begin{pmatrix}
t \\
\\
x
\end{pmatrix} \notag \\
&=\begin{pmatrix}
\cosh\vartheta_1\cosh\vartheta_2+\sinh\vartheta_1\sinh\vartheta_2 & \cosh\vartheta_2\sinh\vartheta_2+\sinh\vartheta_1\cosh\vartheta_2 \\
\\
\sinh\vartheta_1\cosh\vartheta_2+\cosh\vartheta_1\sinh\vartheta_2 & \sinh\vartheta_1\sinh\vartheta_2+\cosh\vartheta_1\cosh\vartheta_2
\end{pmatrix}\begin{pmatrix}
t \\
\\
x
\end{pmatrix}\notag \\
&=\begin{pmatrix}
\cosh(\vartheta_1+\vartheta_2) & \sinh(\vartheta_1+\vartheta_2) \\
\\
\sinh(\vartheta_1+\vartheta_2) & \cosh(\vartheta_1+\vartheta_2)
\end{pmatrix}\begin{pmatrix}
t \\
\\
x
\end{pmatrix}\;.
\end{align}
Osserviamo che anche in questa forma la lunghezza di Minkowski è conservata:
\begin{align}
\diff{\tau}^2= (\diff{t'})^2-(\diff{x'})^2 &= (\cosh\vartheta\diff{t}+\sinh\vartheta\diff{x})^2-(\sinh\vartheta\diff{t}+\cosh\vartheta\diff{x})^2 \notag \\
&=(\cosh^2\vartheta-\sinh^2\vartheta)\diff{t}^2-(\cosh^2\vartheta-\sinh^2\vartheta)\diff{x}^2 \notag \\
&= \diff{t}^2-\diff{x}^2\;.
\end{align}
Il gruppo di matrici indotte dalle trasformazioni di Lorentz è denominato \textit{gruppo ortogonale speciale} (1,1), e di denota con 
$SO(1,1)$. Ricaviamo anche in questa forma la legge di composizione delle velocità. Sapendo che:
\begin{equation}
\frac{v}{c}=\beta=\frac{\sinh\vartheta}{\cosh\vartheta}=\tanh\vartheta\;,
\end{equation}
si ha:
\begin{equation}
\beta_1\circ\beta_2=\tanh(\vartheta_1+\vartheta_2)=\frac{\tanh\vartheta_1+\tanh\vartheta_2}{1+\tanh\vartheta_1\tanh\vartheta_2}=
\frac{\beta_1+\beta_2}{1+\beta_1\beta_2}\;.
\end{equation}
Moltiplicando tutto per $c$ otteniamo:
\begin{equation}
v_1\circ v_2=\frac{v_1+v_2}{1+v_1v_2/c^2}\;. \label{ch2_velocitycomp}
\end{equation}
La legge di composizione delle velocità della meccanica Newtoniana rientra nell'ordine zero di approssimazione per piccole velocità.
Infatti se $v_1/c,v_2/c\ll 1$, allora, sviluppando in serie di Taylor il denominatore della \eqref{ch2_velocitycomp} si ha:
\begin{equation}
v_1\circ v_2 =\frac{v_1+v_2}{1+v_1v_2/c^2}\simeq (v_1+v_2)\left(1-\frac{v_1v_2}{c^2}+\cdots\right)\simeq v_1+v_2\;.
\end{equation}
Introduciamo adesso il \textit{tensore metrico} $g_{\mu\nu}$ dato da:
\begin{equation}
 g_{\mu\nu}=\left(\begin{matrix}
                   -1 & {} \\
{} & \mathbb{I}_n
                  \end{matrix}\right) \qquad \qquad \mu,\nu=0,1,2,3\;.
\end{equation}
In 1+1 dimensioni, cioè nello spaziotempo $\mathbb{M}^2$:
\begin{equation}
g_{\mu\nu}=\begin{pmatrix}
-1 & 0 \\
0 & 1
\end{pmatrix}\;,
\end{equation}
con $\mu,\nu=0,1$. Le trasformazioni di Lorentz sono date da:
\begin{align}
\begin{pmatrix}
t' \\
x'
\end{pmatrix} &= L(\vartheta)\begin{pmatrix}
t \\
x
\end{pmatrix}\;, \\
(x'_{\mu})^Tg_{\mu\nu}x_{\nu} &= x_{\mu}^TL^T(\vartheta)g_{\mu\nu}L(\vartheta)x_{\nu}\;,
\end{align}
da cui $g=L^TgL$. \\

Dato un punto nel cono-luce, è sempre possibile trovare un sistema di riferimento in cui il punto non abbia componenti sugli assi 
spaziali, cioè in modo tale che:
\begin{equation}
\diff{x'}=\gamma(\diff{x}+\beta\diff{t})=0\qquad \Longrightarrow\qquad \dev{x}{t}=-\beta\;,
\end{equation}
quindi è sufficiente scegliere un sistema di riferimento che si muova rispetto al primo con velocità $-\beta$. Un vettore dello 
spaziotempo non avente componenti sugli assi spaziali è detto \textit{vettore-tempo}. Si osserva che il prodotto scalare fra due 
vettori-tempo è sempre negativo. Da questo si deduce che i coni-luce $V_+$ e $V_-$ sono invarianti per trasformazioni di Lorentz.
\subsection{In 3+1 dimensioni}
Il gruppo delle rotazioni in $\mathbb{R}^3$ è un sottogruppo delle matrici di Lorentz. Nello spazio di Minkowski $\mathbb{M}^4$, 
possiamo generalizzare quanto detto prima a 3+1 dimensioni:
\begin{equation}
\begin{pmatrix}
t' \\
x' \\
y' \\
z'
\end{pmatrix}=\begin{pmatrix}
A & B \\
0 & D
\end{pmatrix}\begin{pmatrix}
t \\
x \\
y \\
z
\end{pmatrix}\;,
\end{equation}
con la condizione che la matrice della trasformazione, che denotiamo $\Lambda$ sia tale che $\Lambda^T g\Lambda=g$. Allora i 
quadrivettori $(x,y,z,t)\in \mathbb{M}_4$ sono tali che:
\begin{equation}
\begin{cases}
x_{\mu}'=\Lambda_{\mu\nu}x_{\nu} \\
\\
\Lambda^Tg\Lambda= g \\
\\
\langle x', y'\rangle = \langle x, y\rangle
\end{cases}\qquad\qquad \Lambda\in SO(3,1)\;.
\end{equation}
Vi sono particolari forme della matrice di Lorentz $\Lambda$ che danno luogo alle cosidette \textit{inversioni}:
\begin{itemize}
 \item se $\Lambda=\left(\begin{matrix}
                       -1 & {} \\
{} & \mathbb{I}_3
                      \end{matrix}\right)$ abbiamo l'\textit{inversione temporale}, cioè:
\begin{equation}
\Lambda\begin{pmatrix}
t \\
\mathbf{x}
\end{pmatrix}=\begin{pmatrix}
-t \\
\mathbf{x}
\end{pmatrix}\;;
\end{equation}
\item se $\Lambda=\left(\begin{matrix}
                         1 & {} \\
{} & -\mathbb{I}_3
                        \end{matrix}\right)$ abbiamo l'\textit{inversione spaziale}, cioè:
                        
\begin{equation}
\Lambda\begin{pmatrix}
t \\
\mathbf{x}
\end{pmatrix}=\begin{pmatrix}
t \\
-\mathbf{x}
\end{pmatrix}\;;
\end{equation}
\item se $\Lambda=\left(\begin{matrix}
                         -1 & {} \\
{} & -\mathbb{I}_3
                        \end{matrix}\right)$ abbiamo l'\textit{inversione dello spaziotempo}, cioè:

\begin{equation}
\Lambda\begin{pmatrix}
t \\
\mathbf{x}
\end{pmatrix}=\begin{pmatrix}
-t \\
-\mathbf{x}
\end{pmatrix}\;.
\end{equation}
\end{itemize}
Il gruppo costituito dalle matrici indotte da trasformazioni di Lorentz è denominato \textit{gruppo di Lorentz} e si denota con 
$\mathcal{L}$. Si tratta di un gruppo continuo non compatto ed è unione di quattro sottogruppi:
$$
\mathcal{L}=\mathcal{L}^+_{\uparrow}\cup\mathcal{L}^+_{\downarrow}\cup\mathcal{L}^-_{\uparrow}\cup\mathcal{L}^-_{\downarrow}\;.
$$
Il sottogruppo $\mathcal{L}^+_{\uparrow}$ è il \textit{gruppo di Lorentz ortocrono}, cioè costituito da trasformazioni che non 
coinvolgono inversioni.
\section{Cenni di teoria dei gruppi}
\subsection{Rappresentazioni di gruppi}
Consideriamo il gruppo delle rotazioni in $\mathbb{R}^2$, con tensore metrico $g=\left(\begin{matrix}
                                                                                        1 & 0 \\
0 & 1
                                                                                       \end{matrix}\right)$. Una rotazione intorno all'asse $\hat{z}$ è data da:
\begin{equation}
R(\hat{z},\theta)=\begin{pmatrix}
\cos\theta & \sin\theta \\
-\sin\theta & \cos\theta
\end{pmatrix}\in SO(2)\;.
\end{equation}                 
Le matrici del gruppo delle rotazioni sono tali che $R^T=R^{-1}$. Abbiamo inoltre una rappresentazione del gruppo $(\mathbb{R},+)$ in 
$R(\theta)$ dotato del prodotto righe per colonne data da:
\begin{align}
\theta_1,\theta_2\in (\mathbb{R},+)\quad  &\longmapsto \quad \theta_1+\theta_2\;, \\
R(\theta_1),R(\theta_2)\in(R(\theta),\circ) \quad & \longmapsto \quad R(\theta_1)R(\theta_2)=R(\theta_1+\theta_2)\;.
\end{align}
Introduciamo adesso il \textit{generatore} $L_z$ del sottogruppo delle rotazioni intorno all'asse $\hat{z}$ dato da:
\begin{equation}
L_z\equiv \left.\dev{R(\theta)}{\theta}\right|_{\theta=0}=\begin{pmatrix}
0 & 1 \\
-1 & 0
\end{pmatrix}\;.
\end{equation}
Osserviamo che:
\begin{equation}
L_z^2=-\mathbb{I},\qquad L_z^3=-L_z,\qquad L_z^4=\mathbb{I}\;.
\end{equation}
Allora, esponenziando il generatore e sviluppando in serie di Taylor otteniamo:
\begin{align}
e^{\theta L_z} &= \mathbb{I}+\theta L_z+\frac{1}{2!}\theta^2L_z^2+\frac{1}{3!}\theta^3L_z^3+\frac{1}{4!}\theta^4L_z^4+\cdots \notag \\
&= \mathbb{I}+\theta L_z-\frac{1}{2}\theta^2\mathbb{I}-\frac{1}{6}\theta^3L_z^3+\frac{1}{24}\theta^4\mathbb{I}+\cdots \notag \\
&= \left(1-\frac{\theta^2}{2}+\frac{\theta^4}{24}+\cdots\right)\mathbb{I}+\left(\theta-\frac{\theta^3}{6}+\cdots\right)L_z\;.
\end{align}
Il termine proporzionale a $\mathbb{I}$ è lo sviluppo di $\cos\theta$, mentre quello proporzionale a $L_z$ è lo sviluppo di $\sin\theta$,
 dunque si ottiene:
\begin{equation}
e^{\theta L_z} =\cos\theta\begin{pmatrix}
1 & 0 \\
0 & 1
\end{pmatrix}+\sin\theta\begin{pmatrix}
0 & 1 \\
-1 & 0
\end{pmatrix}=\begin{pmatrix}
\cos\theta & \sin\theta \\
-\sin\theta & \cos\theta
\end{pmatrix}=R(\theta)\;.
\end{equation} 
 Vediamo se adesso troviamo dei generatori del gruppo di Lorentz nello spazio di Minkowski $\mathbb{M}^2$. Il tensore metrico sarà $g=
\mathrm{diag}(-1,1)$ e consideriamo una trasformazione di rapidità $\vartheta$:
\begin{equation}
L(\hat{x},\vartheta)=\begin{pmatrix}
\cosh\vartheta & \sinh\vartheta \\
\sinh\vartheta & \cosh\vartheta
\end{pmatrix}\in SO(1,1)\;.
\end{equation}
Anche in questo caso, ricordiamo, abbiamo una rappresentazione del gruppo $(\mathbb{R},+)$ in $\mathcal{L}$ dotato del prodotto righe 
per colonne del tutto analoga a quella delle rotazioni, cioè:
\begin{equation}
L(\hat{x},\vartheta_1)L(\hat{x},\vartheta_2)=L(\hat{x},\theta_1+\theta_2)\;.
\end{equation}
Le matrici di Lorentz soddisfano la relazione $L^TgL=g$. Definiamo a questo punto, analogamente al caso delle rotazioni, il 
\textit{generatore del sottogruppo ad un parametro del boost di Lorentz} (lungo l'asse $\hat{x}$) $M_x$:
\begin{equation}
M_x \equiv\left.\dev{L(\vartheta)}{\vartheta}\right|_{\vartheta=0}=\begin{pmatrix}
0 & 1 \\
1 &0
\end{pmatrix}\;.
\end{equation}
Si ha che $M_x^2=\mathbb{I}$ e $M_x^3=M_x$, allora, sempre in analogia alle rotazioni, tramite l'esponenziazione del generatore e lo sviluppo in serie di Taylor, troviamo che:
\begin{equation}
L(\hat{x},\vartheta)=\begin{pmatrix}
\cosh\vartheta & \sinh\vartheta \\
\sinh\vartheta & \cosh\vartheta
\end{pmatrix}=e^{\vartheta M_x}\;.
\end{equation}
Considerando adesso le rotazioni in $\mathbb{R}^3$, otteniamo che, dato che possiamo ruotare intorno a tre assi indipendentemente, 
abbiamo tre generatori distinti $L_x,L_y,L_z$, dati da:
\begin{equation}
L_x = \begin{pmatrix}
0 & 0 & 0 \\
0 & 0 & 1 \\
0 & -1 & 0
\end{pmatrix},\quad L_y =\begin{pmatrix}
0 & 0 & -1 \\
0 & 0 & 0 \\
1 & 0 & 0
\end{pmatrix}, \quad L_z=\begin{pmatrix}
0 & 1 & 0 \\
-1 & 0 & 0 \\
0 & 0 & 0
\end{pmatrix}\;.
\end{equation}
Sappiamo che le rotazioni non sono commutative, dunque, introducendo l'operatore \textit{commutatore} $[A,B]=AB-BA$ si ha:
\begin{equation}
\begin{cases}
[L_x,L_y]=-L_z \\
\\
[L_a,L_b]=-\epsilon_{abc}L_c
\end{cases}\quad [L_a]_{bc}=\epsilon_{abc}L_c\;.
\end{equation}
In $\mathbb{M}^4$, a seconda degli assi su cui viene eseguito il boost di Lorentz, abbiamo tre generatori di boost $M_x,M_y,M_z$:
\begin{equation}
M_x = \begin{pmatrix}
0 & 1 & 0 & 0 \\
1 & {} & {} &{} \\
0 & {} &\mbox{\huge{0}} & {} \\
0 & {} & {} & {}
\end{pmatrix},\quad M_y=\begin{pmatrix}
0 & 0 & 1 & 0 \\
0 & {} & {} & {} \\
1 & {} &\mbox{\huge{0}} & {} \\
0 & {} & {} & {}
\end{pmatrix}, \quad M_z=\begin{pmatrix}
0 & 0 & 0 & 1 \\
0 & {} & {} & {} \\
0 & {} &\mbox{\huge{0}} & {} \\
1 & {} & {} & {}
\end{pmatrix}\;.
\end{equation}
Dunque in $\mathbb{M}^4$ si hanno sei generatori, tre di boost e tre di rotazioni per cui valgono le relazioni:
\begin{equation}
\begin{cases}
[L_a,L_b]=-\epsilon_{abk}L_k \\
\\
[M_a,M_b]=\epsilon_{abk}L_k \\
\\
[L_a,M_b]=\epsilon_{abk}M_k
\end{cases}\;.
\end{equation}
Prendiamo in esame adesso un boost di Lorentz in $\mathbb{M}^4$ sull'asse $\hat{x}$, senza rotazioni sugli altri assi, di rapidità $
\vartheta$, cioè:
\begin{equation}
\begin{pmatrix}
t' \\
x' \\
y' \\
z'
\end{pmatrix}= \begin{pmatrix}
\cosh\vartheta & \sinh\vartheta & 0 & 0 \\
\sinh\vartheta & \cosh\vartheta & 0 & 0 \\
0 & 0 & 1 & 0 \\
0 & 0 & 0 & 1
\end{pmatrix}\begin{pmatrix}
t \\
x \\
y \\
z
\end{pmatrix}\;.
\end{equation}
Le trasformazioni di Lorentz in forma differenziale saranno pertanto:
\begin{equation}
\begin{cases}
\diff{t'}=\gamma(\diff{t}+\beta\diff{x}) \\
\\
\diff{x'}=\gamma(\diff{x}+\beta\diff{t}) \\
\\
\diff{y'}=\diff{y} \\
\\
\diff{z'}=\diff{z}
\end{cases}\;.
\end{equation}
Vogliamo adesso scrivere come trasforma la velocità componente per componente. Dividendo ciascuna delle componenti spaziali per la 
componente temporale si ottengono le relazioni:
\begin{align}
\dev{x'}{t'} &= \frac{v+\diff{x}/\diff{t}}{1+\frac{v}{c^2}\diff{x}/\diff{t}}\;, \\
\dev{y'}{t'} &= \frac{1}{\gamma}\frac{\diff{y}}{\diff{t}+\beta\diff{x}}=\frac{1}{\gamma}\frac{\diff{y}/\diff{t}}{1+\frac{v}{c^2}\diff{x}/\diff{t}}\;, \\
\dev{z'}{t'} &= \frac{1}{\gamma}\frac{\diff{z}/\diff{t}}{1+\frac{v}{c^2}\diff{x}/\diff{t}}\;.
\end{align}
Queste trasformazioni però non sono soddisfacenti, in quanto le componenti $y$ e $z$ della velocità dipendono dalla componente $x$. 
L'obiettivo è scrivere una trasformazione simile a quella per i quadrivettori. L'unica cosa è pensare la velocità anch'essa come un 
quadrivettore. Definiamo perciò la velocità $u$:
\begin{equation}
u_{\mu}=\dev{x_{\mu}}{\tau}\;,
\end{equation}
dove $\diff{\tau}$ è il tempo proprio. Sappiamo che la relazione fra il tempo proprio di un sistema di riferimento e lo stesso tempo misurato in un altro sistema di riferimento è data da $\diff{\tau}=\diff{t}/\gamma$. Allora:
\begin{equation}
u_{\mu}=\dev{x_{\mu}}{t}\dev{t}{\tau}=\gamma\dev{x}{t}\;,
\end{equation}
da cui:
\begin{equation}
 \begin{cases}
  u_0=\gamma\dfrac{\diff}{\diff{t}}(ct)=\gamma c \\
\\
\mathbf{u}=\gamma\dfrac{\diff{\mathbf{x}}}{\diff{t}}=\gamma\mathbf{v}
 \end{cases}\;.
\end{equation}
Per i quadrivettori, sapevamo che il prodotto scalare:
\begin{equation}
\sum_{\mu=0}^3x_{\mu}x_{\mu}=x^2-(ct)^2
\end{equation}
è invariante per trasformazioni di Lorentz. Per le \textit{quadrivelocità}, la quantità invariante è il modulo, dato da:
\begin{equation}
g_{\mu\nu}u_{\mu}u_{\nu}=-\gamma^2c^2+\gamma^2v^2=\frac{v^2-c^2}{1-v^2/c^2}=c^2\frac{v^2-c^2}{c^2-v^2}=-c^2\;,
\end{equation}
in questo caso, dato che si tratta del quadrato della velocità della luce, oltre ad essere invariante, è anche costante.
\subsection{Rappresentazioni irriducibili}
Trovare dunque una rappresentazione irriducibile, significa trovare l'unica decomposizione dello spazio in sottospazi invarianti non 
ulteriormente riducibili. Il problema delle rappresentazioni irriducibili può essere riportato a trovare i generatori delle rotazioni.
Il gruppo delle rotazioni, ricordiamo, è definito da:
\begin{equation}
x_i'=R_{ia}x_a\;,\qquad i,a=1,2,3\;.
\end{equation}
Le quantità scalari sono tali che $s'=s$, mentre i tensori sono matrici le cui componenti trasformano sotto rotazione nel seguente modo:
\begin{equation}
T_{ij}'=R_{ia}R_{jb}T_{ab}\;. \label{ch3_tensor}
\end{equation}
Tramite i tensori siamo in grado di trasformare i polinomi omogenei. Rappresentando il tensore $T$ come un vettore colonna di nove 
componenti, cioè:
\begin{equation}
T=\begin{pmatrix}
T_{11} \\
T_{12} \\
T_{13} \\
\vdots \\
T_{33}
\end{pmatrix}\;,
\end{equation}
allora la rotazione \eqref{ch3_tensor} può essere espressa come:
\begin{equation}
T'=(R\circ R)T\;,
\end{equation}
dove $R\in M(9,\mathbb{R})$ è tale che $R_1\circ R_1\cdot R_2\circ R_2=R_1R_2\circ R_1R_2$ e dunque $R_1R_2=R_{1\circ 2}$. In termini 
di versore $\hat{n}$ e angolo di rotazione $\varphi$ si ha:
\begin{equation}
(\hat{n},\varphi)\longmapsto R_{9\times 9}(\hat{n},\varphi),\qquad R_1^{9\times 9}\cdot R_2^{9\times 9}=R_{1\circ 2}^{9\times 9}\;.
\end{equation}
Abbiamo dunque una \textit{rappresentazione 9-dimensionale} del gruppo delle rotazioni. Come tensore, data la posizione e l'impulso di 
una singola particella, possiamo considerare il prodotto esterno $T_{ab}=x_ap_b$. Se un tensore $S$ è simmetrico, cioè $S_{ab}=S_{ba}$, sarà ancora simmetrico in un sistema di riferimento ruotato? Applichiamo una rotazione al tensore $S$:
\begin{equation}
S_{ij}'=R_{ia}R_{jb}S_{ab}=R_{ia}R_{jb}S_{ba}=R_{jb}R_{ia}S_{ba}=S_{ji}'\;.
\end{equation}
Stessa cosa se il tensore è antisimmetrico. In particolare, ogni tensore a due indici può essere scritto come somma di un tensore 
simmetrico e di uno antisimmetrico. Allora, lo spazio vettoriale dei tensori è somma diretta del sottospazio vettoriale delle matrici 
simmetrice e di quello delle matrici antisimmetriche, il primo avente dimensione 6 e il secondo 3:
\begin{equation}
T=S+A,\qquad S=\begin{pmatrix}
s_{11} & s_{12} & s_{13} \\
s_{12} & s_{22} & s_{23} \\
s_{13} & s_{23} & s_{33}
\end{pmatrix},\qquad A=\begin{pmatrix}
0 & a_{12} & a_{13} \\
-a_{12} & 0 & a_{23} \\
-a_{13} & -a_{23} & 0
\end{pmatrix}\;.
\end{equation}
Poiché gli spazi stanno in somma diretta, rappresentando il tensore in forma vettoriale e applicando una rotazione, le componenti 
provenienti dalla matrice simmetrica non devono mischiarsi con quelle provenienti dalla matrice antisimmetrica. Allora avremo qualcosa 
della forma:
\begin{equation}
\begin{pmatrix}
R_1^{6\times 6} & 0 \\
0 & R_1^{3\times 3}
\end{pmatrix}\begin{pmatrix}
s_{11} \\
s_{12} \\
\vdots \\
s_{33} \\
a_{11} \\
a_{12} \\
a_{13}
\end{pmatrix}\;.
\end{equation}
Notiamo inoltre che in tre dimensioni, i tensori antisimmetrici possono essere identificati con i vettori tramite la contrazione con 
l'indice antisimmetrico $\epsilon_{iab}$, $A_{ab}\epsilon_{iab}\equiv L_i$. Dunque bisogna ridurre il sottospazio delle matrici simmetriche. Sia $S$ un tensore simmetrico, e ne considero la traccia:
\begin{equation}
s= s_{11}+s_{22}+s_{33}\;.
\end{equation}
Allora:
\begin{equation}
S=\begin{pmatrix}
s_{11} & s_{12} & s_{13} \\
{} & s_{22} & s_{23} \\
{} & {} & s_{23}
\end{pmatrix}=\begin{pmatrix}
s_{11}-s/3 & s_{12} & s_{13} \\
{} & s_{22}-s/3 & s_{23} \\
{} & {} & s_{33}-s/3
\end{pmatrix}\;.
\end{equation}
Inoltre:
\begin{equation}
\delta_{ij}S'_{ij} = \delta^{ij}R_{ia}R_{jb}S_{ab}=R_{ai}^T\delta_{ij}R_{jb}S_{ab}=R_{ai}^TR_{ib}S_{ab}=\delta_{ab}S_{ab}\;.
\end{equation}
Da questa relazione concludiamo che la traccia di un tensore simmetrico è uno scalare (infatti abbiamo provato che è invariante per 
rotazione). Allora, in quanto sussiste una relazione di dipendenza lineare non banale, abbiamo trovato una rappresentazione irriducibile
del gruppo delle rotazioni in $\mathbb{R}^3$, costituita dai tensori simmetrici a traccia nulla (cinque parametri liberi), i vettori (matrici 
antisimmetriche, tre parametri liberi) e uno spazio unidimensionale formato dagli scalari (le tracce dei tensori simmetrici). Abbiamo
 dunque: $\mathbf{3}\otimes\mathbf{3}=\mathbf{5}\oplus\mathbf{3}\oplus\mathbf{1}$. Passando adesso al gruppo di Lorentz, le quantità invarianti per boost di Lorentz sono i 4-scalari. Avremo similmente i 4-tensori (rango 2: $T_{\{\mu\nu\}},F_{[\mu\nu]},\ldots$).
\section{Formulazione covariante della Dinamica}
\subsection{Principi della Dinamica}
\begin{enumerate}
 \item Principio d'inerzia. Tutti i sistemi di riferimento inerziali in moto relativo uniforme sono indistinguibili. Questo è un caso 
particolare del principio di relatività, quindi rimane valido anche nella formulazione relativistica;
\item l'accelerazione di un corpo è direttamente proporzionale alla forza applicata su di esso;
\item ad ogni azione corrisponde una reazione uguale e contraria, agente sulla stessa retta.
\end{enumerate}
L'introduzione delle trasformazioni di Lorentz ha avuto come conseguenza la perdita del concetto di \textit{simultaneità} degli eventi, 
infatti se per un osservatore in un determinato sistema di riferimento due eventi accadono nello stesso istante, è immediato verificare 
che esistono infiniti sistemi di riferimento in cui essi sono sfasati. Questa è una diretta conseguenza della costanza della velocità 
della luce. Inoltre, i sistemi in cui si lavorava nella meccanica Newtoniana erano \textit{isolati}, cioè in assenza di forze esterne. 
Con l'introduzione dei campi (gravitazionale, elettromagnetico) che regolano i diversi tipi di interazione, questa ipotesi non è più 
valida, unitamente all'ipotesi di \textit{istantaneità} della propagazione dell'interazione. Difatti, la velocità della luce è quella 
massima con cui un segnale può propagarsi, dunque, per esempio, se muovo una carica e di conseguenza modifico il campo elettromagnetico 
che essa crea, una seconda carica posta a distanza $r$ risentirà delle perturbazioni causate dalla prima al più dopo un tempo $r/c$. 
Pertanto, vengono meno i presupposti su cui veniva enunciato il terzo principio, e la sua validità non può essere estesa alla 
formulazione relativistica. Tuttavia, due conseguenze importanti derivanti dal terzo principio, sono la conservazione del \textit{tri-impulso} e del momento angolare, dunque è naturale domandarsi se tali quantità siano ancora costanti nel moto nel limite relativistico o 
ve ne siano comunque delle altre.
\subsection{Quantità di moto}
Ricordiamo la definizione della \textit{quadrivelocità} in un sistema di riferimento $K$:
\begin{equation}
 \begin{cases}
  u_0=\gamma(v)c \\
\\
\mathbf{u}=\gamma(v)\mathbf{v}
 \end{cases}\;,
\end{equation}
con il parametro $\gamma$ dipendente esclusivamente dal modulo della velocità. In un sistema di riferimento $K'$ che si muove rispetto a $K$ con velocità $\beta$ lungo l'asse $\hat{x}$, la quadrivelocità $u_{\mu}$ era data dalle trasformazioni di Lorentz:
\begin{equation}
 \begin{cases}
u_0'=\gamma(\beta)(u_0+\beta u_x) \\
\\
u_x'=\gamma(\beta)(u_x+\beta u_0) \\
\\
u_y'=u_y \\
\\
u_z'=u_z  
 \end{cases}\;.
\end{equation}
Supponiamo adesso di avere, in un sistema di riferimento $K_0$, un urto tra due particelle puntiformi di velocità iniziale uguale e di 
modulo $v$. Per la conservazione della quantità di moto:
\begin{align}
&\mathbf{p}_{\mathrm{in}}=\mathbf{p}_1+\mathbf{p}_2=0\;, &\mathbf{p}_{\mathrm{fin}}=\mathbf{p}_1'+\mathbf{p}_2'=0\;.
\end{align}
Cerchiamo adesso di scrivere la quantità di moto nel caso generale, fermo restando che essa sia proporzionale alla velocità:
\begin{equation}
 \mathbf{p}\stackrel{?}{=}f(v)\mathbf{v}\;. \label{ch4_tobedet}
\end{equation}
Il fattore di proporzionalità dev'essere funzione unicamente del modulo di $\vec{v}$, in quanto deve essere invariante per rotazione. 
Sappiamo che nel limite non relativistico $f(v\ll c)\simeq m$, dove $m$ è la quantità scalare chiamata \textit{massa} del corpo. Inoltre, la funzione $f$ deve essere iniettiva. Assumiamo adesso che esista una \textit{energia cinetica} $T(v)$, funzione scalare del modulo della velocità, anch'essa iniettiva, che si conserva, allora $T(v_1)+T(v_2)=T(v_1')+T(v_2')$. Se inoltre $v_1=v_2=v$ e $v_1'=v_2'=v'$, allora $T(v)=T(v')$. Nel sistema del centro di massa, l'urto può essere descritto dalle seguenti relazioni:
\begin{equation}
 \begin{cases}
  v_x^i=v_x^{i'} \\
\\
v_y^i=-v_y^{i'}
 \end{cases}\qquad i=1,2\;.
\end{equation}
Queste relazioni valgono per le tri-velocità anche nel limite relativistico. In termini di quadrivelocità, otteniamo che il modulo delle 
componenti spaziali è lo stesso, dunque possiamo riscrivere la \eqref{ch4_tobedet} come:
\begin{equation}
\mathbf{p}=g(v)\mathbf{u}\;,
\end{equation}
in quanto $\mathbf{u}=\gamma(v)\mathbf{v}$. Dalla relazione $\Delta(\mathbf{p}_1+\mathbf{p}_2)_y=0$ segue che $|\Delta \mathbf{p}_1|=|\Delta\mathbf{p}_2|$, cioè:
\begin{align}
&|\Delta p_1^y|=g(v_1)|\Delta u_1^y|\;, &|\Delta p_2^y|=g(v_2)|\Delta u_2^y|\;.
\end{align}
Sommando le relazioni ed imponendo che $\Delta u_1^y+\Delta u_2^y=0$, troviamo che $g(v_1)=g(v_2)$. Questa relazione è valida in ogni 
sistema di riferimento inerziale, dunque $g(v_1)=g(v_2)=g(v)$ è una quantità costante, che denotiamo $m$. Pertanto:
\begin{equation}
g(v)=\frac{f(v)}{\gamma(v)}=m\qquad \Longrightarrow\qquad f(v)=\gamma(v)m\;.
\end{equation}
A questo punto, possiamo definire il \textit{quadri-impulso} $p_{\mu}$ come $p=m\gamma u$, o in altre parole:
\begin{equation}
 \begin{cases}
  p_0=mu_0 \\
\\
\mathbf{p}=m\mathbf{u}
 \end{cases}\;.
\end{equation}
Se si conserva il tri-impulso, cioè $\Delta\mathbf{p}=0$, considero le trasformazioni di Lorentz:
\begin{equation}
\begin{cases}
\Delta p_x'=\gamma(\beta)(\Delta p_x+\beta\Delta p_0) \\
\\
\Delta p_0'=\gamma(\beta)(\Delta p_0+\beta\Delta p_x)
\end{cases}\;.
\end{equation}
Se $\Delta p_x=0$, allora esso sarà nullo in tutti i sistemi di riferimento inerziali, dunque si avrà anche $\Delta p_x'=0$, e dalla 
prima trasformazione, segue che anche $\Delta p_0=0$. Il secondo membro della seconda trasformazione è identicamente nullo, dunque sarà anche $\Delta p_0'=0$. Concludiamo che se il tri-impulso si conserva, allora si conserverà anche la componente $p_0$ del 
quadri-impulso. Vediamo dunque cosa rappresenta. Dalla definizione di quantità di modo appena data:
\begin{equation}
p_0=m\gamma c=mc\frac{1}{\left(1-v^2/c^2\right)^{1/2}}\;.
\end{equation}
Nel limite non relativistico $v/c\ll 1$ si ottiene, sviluppando in serie di Taylor:
\begin{equation}
p_0=\frac{1}{c}mc^2\left(1+\frac{v^2}{2c^2}+\cdots\right)=\frac{1}{c}\left(mc^2+\frac{1}{2}mv^2+\cdots\right)\;:
\end{equation}
In termini del problema delle due particelle di massa $m_1$ e $m_2$ e velocità $v_1$ e $v_2$ ciò comporta:
\begin{equation}
m_1c^2+\frac{1}{2}m_1v_1^2+m_2c^2+\frac{1}{2}m_2v^2_2=m_1c^2+\frac{1}{2}m_1(v_1')^2+m_2c^2+\frac{1}{2}m_2(v_2')^2\;,
\end{equation}
cioè:
\begin{equation}
\frac{1}{2}m_1v_1^2+\frac{1}{2}m_2v_2^2=\frac{1}{2}m_1(v_1')^2+\frac{1}{2}m_2(v_2')^2\;,
\end{equation}
che è esattamente la legge di conservazione dell'energia cinetica nel limite non relativistico. Concludiamo dunque che la componente 
$p_0$ rappresenta l'energia cinetica del sistema, e, generalizzando quanto visto, possiamo asserire che energia e tri-impulso sono le 
componenti di un quadrivettore denominato \textit{quadri-impulso}. Possiamo dunque scrivere l'impulso nella formulazione relativistica:
\begin{equation}
 \mathbf{p}=\frac{m\mathbf{v}}{\sqrt{1-v^2/c^2}}=m\gamma\mathbf{v}\;, \label{ch4_relativistic3mom}
\end{equation}
dove $m$ è un quadri-scalare. Da questa formula, appare che, all'aumentare della velocità della particella, aumenti proporzionalmente 
anche la sua massa, cioè $m\gamma=m(v)$. In tal caso, sarebbe giustificato il valore di $c$ come limite superiore di propagazione di un 
evento.
\subsection{Secondo principio}
Nella formulazione di Newton, ricordiamo, il secondo principio era dato da:
\begin{equation}
m\dev{\mathbf{v}}{t}=\mathbf{F}\;,
\end{equation}
che possiamo riscrivere come:
\begin{equation}
\dev{\mathbf{p}}{t}=\mathbf{F}\;,
\end{equation}
con $\mathbf{p}$ dato dalla \eqref{ch4_relativistic3mom}. Nel caso di moto unidimensionale con dato iniziale nullo, e assumendo che la massa sia invariante, abbiamo:
\begin{equation}
\frac{\diff}{\diff{t}}(\gamma v)=\frac{F}{m}=g\qquad \Longrightarrow\qquad \gamma v = gt\;.
\end{equation}
Sostituendo l'espressione di $\gamma$ otteniamo:
\begin{equation}
v=\frac{gt}{\gamma}=gt\sqrt{1-\frac{v^2}{c^2}}\;.
\end{equation}
Elevando al quadrato e dividendo per $c^2$ ambo i membri ottieniamo:
\begin{equation}
\left(\frac{v}{c}\right)^2=\left(\frac{gt}{c}\right)^2\left(1-\frac{v^2}{c^2}\right)\;.
\end{equation}
Ricavando da questa relazione il valore di $v/c$, si ha:
\begin{equation}
 \left(\frac{v}{c}\right)^2=\frac{(gt/c)^2}{1+(gt/c)^2}\;.
\end{equation}
Essendo tutte quantità sempre positive, possiamo osservare che $v/c<1$, che è in accordo con quanto precedentemente detto. Ricavando infine la velocità $v$:
\begin{equation}
\dev{x}{t}=v=\frac{gt}{\sqrt{1+(gt/c)^2}}\;,
\end{equation}
otteniamo che la linea d'universo descritta dalla particella è un'iperbole. \\

Consideriamo adesso il seguente esempio: in un certo sistema di riferimento, un blocco di massa $M$ è in quiete e due corpi puntiformi, 
ciascuno di massa $m$ si avvicinando ad esso con velocità uguali e opposte, rispettivamente $\mathbf{u}$ e $-mathbf{u}$ dirette lungo l'asse $\hat{y}$. L'urto è anaelastico. Allora, per la legge di conservazione della quantità di moto della meccanica classica, si ha:
\begin{equation}
M\cdot 0+m(u_x+u_y)=0=(M+2m)\cdot 0\;.
\end{equation}
Questa relazione deve valere in tutti i sistemi di riferimento inerziali. Consideriamo adesso un secondo sistema di riferimento, in moto 
rispetto al primo con velocità uniforme $\mathbf{v}$ diretta lungo l'asse $\hat{x}$. Allora, il blocco sarà visto in movimento con velocità 
$\mathbf{v}$ e le particelle avranno velocità $\mathbf{u}+\mathbf{v}$ e $-\mathbf{u}+\mathbf{v}$. Lungo l'asse $\hat{y}$ vale la relazione descritta nel caso precedente, mentre lungo l'asse $\hat{x}$ si ha:
\begin{equation}
(M+2m)v=(M+2m)v'\qquad \Longrightarrow\qquad v=v'\;.
\end{equation}
Anche in questo caso, la velocità del blocco rimane immutata lungo gli assi e quindi viene rispettato il principio di conservazione della
 quantità di moto. Adesso consideriamo il blocco in quiete e mandiamo contro di esso due impulsi luminosi $\epsilon/c$. In questo caso, il blocco assorbirà gli impulsi e rimarrà fermo, in accordo con il principio classico. Se adesso diamo a tutti gli elementi una velocità $\mathbf{v}$ lungo $\hat{x}$ e indichiamo con $\alpha$ l'angolo compreso tra la direzione degli impulsi e $\mathbf{v}$, sfruttando la relatività, otteniamo la seguente relazione:
 \begin{equation}
 M\gamma v+2\frac{\epsilon'}{c}\sin\alpha=M\gamma_fv_f\;.
 \end{equation}
Questa relazione sembra violare il principio di relatività, in quanto $v_f\ne v$. Questo è un assurdo nato dall'aver assunto che la massa
 del blocco rimanesse costante nell'evoluzione dell'evento. Imponendo che $v_f=v$, la relazione si modifica in:
 \begin{equation}
 M\gamma v+2\frac{\epsilon'}{c}\sin\alpha=M_f\gamma v\qquad \Longleftrightarrow\qquad (M_f-M)\gamma v=2\frac{\epsilon'}{c}\sin\alpha\;.
 \end{equation}
Il seno dell'angolo $\alpha$ corrisponde al rapporto $v/c$, per cui:
\begin{equation}
(M_f-M)\gamma v=\frac{2\epsilon'}{c}\frac{v}{c}\qquad \Longleftrightarrow\qquad \Delta(M\gamma c^2)=2\epsilon'=\Delta E_{\mathrm{luce}}\;,
\end{equation}
con la condizione:
\begin{equation}
\frac{\epsilon'}{\gamma}=\epsilon\;.
\end{equation}
In conclusione, siamo in grado di preservare la conservazione dell'impulso e dell'energia solo se ammettiamo che la massa non rimanga 
costante. A questo punto, emerge ancora più chiaramente, nell'equivalenza massa-energia, che l'energia abbia un ruolo primitivo rispetto 
alla massa. In virtù di queste considerazioni, vogliamo adesso fornire un'espressione del quadri-impulso indipendente dalla massa, ma 
solamente dall'energia:
\begin{equation}
 \begin{cases}
  p_0=\dfrac{E}{c}=m\gamma c \\
\\
\mathbf{p}=m\gamma\mathbf{v}
 \end{cases}\qquad \Longleftrightarrow\qquad 
\frac{\mathbf{p}}{E/c^2}=\mathbf{v}\;.
\end{equation}
Allora definiamo il quadri-impulso come:
\begin{equation}
 p_{\mu}=\left(\frac{E}{c},\mathbf{p}\right)\;.
\end{equation}
Adesso consideriamo la quantità $p_{\mu}p_{\mu}$. Con la nuova definizione si ha:
\begin{equation*}
 p_{\mu}p_{\mu}=\mathbf{p}^2-\frac{E}{c^2}\;,
\end{equation*}
dove si è usata la definizione del prodotto scalare di Minkowski. Usando invece la definizione che coinvolge la quadrivelocità, si ha:
\begin{equation}
p_{\mu}p_{\mu}=m^2u_{\mu}u_{\mu}=-(mc)^2\;.
\end{equation}
Uguagliando le due espressioni, otteniamo la \textit{relazione energia-impulso}:
\begin{equation}
 \frac{E^2}{c^2}-\mathbf{p}^2=m^2c^2\qquad  \Longleftrightarrow \qquad E^2=\mathbf{p}^2c^2+m^2c^4\;,
\end{equation}
dove la quantità $mc^2$ è detta \textit{massa a riposo}. Notiamo che questa relazione fornisce una relazione riguardante il quadrato 
dell'energia, quindi bisogna stare attenti a considerare il doppio segno quando si estrae la radice.
\begin{exm}[Effetto Compton] 
Consideriamo un fotone con quadri-impulso dato da:
\begin{equation}
 k_{\mu}=\hbar\left(\frac{\omega}{c},\mathbf{k}\right)\;.
\end{equation}
Abbiamo che $k^2=0$, il che implica che siamo sul bordo del cono-luce. Trattandosi di un fotone, la cosa ha senso. Ricordando la 
definizione di quadri-impulso appena data ciò implica che la massa del fotone è nulla, il che ha ancora senso. L'effetto fotoelettrico, d'altra parte, ci garantisce che:
\begin{equation}
 \begin{cases}
  E=\hbar\omega \\
\\
\mathbf{p}=\hbar\mathbf{k}
 \end{cases}\;,
\end{equation}
dove $E,\mathbf{p}$ sono l'energia e l'impulso dell'elettrone, soddisfacenti $E^2-\mathbf{p}^2c^2=m^2c^4$. Se adesso consideriamo un elettrone fermo, verso cui mandiamo il fotone di impulso $k_{\mu}$ e lunghezza d'onda $\lambda$, dopo 
l'assorbimento seguirà un'emissione, che risulterà inclinata di un angolo $\theta$ rispetto alla direzione iniziale e con una lunghezza 
d'onda $\lambda'\ne\lambda$, e dunque con impulso $k'\ne k$. Imponiamo la conservazione dell'energia sugli indici zero dei quadri-impulsi (lavoriamo in unità $c=1$):
\begin{equation}
 \hbar\omega+m=\hbar\omega'+E'\qquad  \Longleftrightarrow\qquad  E'-m=\hbar(\omega-\omega')\;. \label{ch4_sistema1}
\end{equation}
Poiché deve essere $\omega-\omega'>0$, da questa relazione otteniamo che la lunghezza d'onda $\lambda'$ è maggiore di quella iniziale. 
Sugli indici $i=1,2,3$ del quadri-impulso imponiamo che:
\begin{equation}
\hbar (k-k')=p'-p\;.
\end{equation}
Elevando al quadrato si ottiene:
\begin{equation}
\hbar^2(-2k\cdot k')=2m^2-2p\cdot p'\qquad \implies\qquad \hbar^2k\cdot k'=p\cdot p'-m^2\;.
\end{equation}
Poiché la quantità $p\cdot p'$ è invariante per trasformazione di Lorentz, ne consideriamo solo la componente zero, quindi:
\begin{equation}
\hbar^2k\cdot k'=m(E'-m)\;.
\end{equation}
In conclusione, otteniamo la relazione:
\begin{equation}
 \hbar^2\omega'\omega(1-\cos\theta)=m(E'-m)\;. \label{ch4_sistema2}
\end{equation}
Mettendo a sistema le equazioni \eqref{ch4_sistema1} e \eqref{ch4_sistema2} troviamo:
\begin{equation}
 \lambda'-\lambda=\frac{\hbar}{mc}(1-\cos\theta),\qquad \mbox{con}\;\frac{\hbar}{mc}=0,024\stackrel{\circ}{\mathrm{A}}\;.
\end{equation}
Inoltre, dopo alcuni passaggi algebrici, otteniamo un'altra relazione:
\begin{equation}
 \lambda E=hc\;,
\end{equation}
che ci consente di capire che per registrare variazioni bisogna usare lunghezze d'onda sufficientemente piccole.
\end{exm}

Vogliamo adesso fare qualche considerazione sul momento angolare $\mathbf{L}$. Sappiamo che $\mathbf{L}$ può essere interpretato come un tensore antisimmetrico tale che:
\begin{equation}
\epsilon_{ijk}L_k=x_ip_j-x_jp_i\;.
\end{equation}
In $\mathbb{M}^4$ estendiamo questo concetto considerando un tensore antisimmetrico $F_{\mu\nu}$ e introducendo il simbolo $\epsilon$ a quattro indici $\epsilon_{\alpha\beta\mu\nu}$. Applicando il simbolo al tensore si ha:
\begin{equation}
 \epsilon_{\alpha\beta\mu\nu}F_{\mu\nu}=\hat{F}_{\alpha\beta}\;.
\end{equation}
Otteniamo dunque un tensore a due indici antisimmetrico, detto il \textit{duale} di $F$. Se avevevamo completato il tri-impulso con 
l'energia, possiamo completare le tre componenti del momento angolare (che ricordiamo sono i generatori delle rotazioni) inserendole in
 un tensore antisimmetrico insieme ai generatori del gruppo di Lorentz:
\begin{equation}
\left(\begin{matrix}
       0 & \mathcal{L}_1 & \mathcal{L}_2 & \mathcal{L}_3 \\
-\mathcal{L}_1 & 0 & L_3 & -L_2 \\
-\mathcal{L}_2 & -L_3 & 0 & L_1 \\
-\mathcal{L}_3 & L_2 & -L_1 & 0
      \end{matrix}\right)\;.
\end{equation}
Possiamo definire dunque il \textit{tensore momento angolare} $M_{\mu\nu}$:
\begin{equation}
 M_{\mu\nu}\equiv \epsilon_{\alpha\beta\mu\nu}x_{\alpha}p_{\beta}\;.
\end{equation}
Poiché $M_{\mu\nu}$ è antisimmetrico, avrà sei componenti linearmente indipendenti, tre che generano le rotazioni e tre che generano i 
boost di Lorentz. Questo tensore si conserva.

\subsection{Forze}
In analogia con il secondo principio:
\begin{equation}
\dev{\mathbf{p}}{t}=\mathbf{F}\;,
\end{equation}
derivando il quadri-impulso rispetto al tempo proprio otteniamo qualcosa di simile a una quadri-forza:
\begin{equation}
\dev{p_{\mu}}{\tau}\equiv G_{\mu}=\dev{p_{\mu}}{t}\dev{t}{\tau}=\begin{cases}
\gamma\dfrac{\diff}{\diff{t}}\left(\dfrac{E}{c}\right)=\dfrac{\gamma}{c}\mathbf{F}\cdot\mathbf{v}\qquad \mu=0 \\
\\
\gamma\mathbf{F}\qquad \mu=1,2,3
\end{cases}\;.
\end{equation}
Poiché $G_{\mu}$ è un quadrivettore, esso deve trasformare secondo le trasformazioni di Lorentz. Consideriamo un sistema di riferimento in cui una particella è a riposo, e scriviamo $G_{\mu}$:
\begin{equation}
G^{\mathrm{rest}}_{\mu}=\begin{pmatrix}
0 \\
\\
\mathbf{F}^{\mathrm{rest}}
\end{pmatrix}\;.
\end{equation}
Se invece osserviamo la particella in un sistema di riferimento in moto relativo con velocità $\mathbf{v}$ lungo l'asse $\hat{x}$, allora 
$G_{\mu}$ sarà data da:
\begin{equation}
G^v_{\mu}=\begin{pmatrix}
\gamma\mathbf{F}\cdot\mathbf{v}/c \\
\\
\gamma\mathbf{F}
\end{pmatrix}\;.
\end{equation}
Queste due quantità saranno connesse da una trasformazione di Lorentz: 
\begin{align}
G_x^v &=\gamma F_x^v=\gamma(F_x^{\mathrm{rest}}+v\cdot 0)=\gamma F_x^{\mathrm{rest}}\;, \\
G_y^v &= \gamma F_y^v=G_y^v=G_y^{\mathrm{rest}}=F_y^{\mathrm{rest}}\;.
\end{align}
Quindi, indicando con $F_{||}$ la componente parallela al boost e con $F_{\perp}$ quelle ortogonali, otteniamo le seguenti trasformazioni:
\begin{equation}
 \begin{cases}
  F_{||}^v=F_{||}^{\mathrm{rest}} \\
\\
F_{\perp}^v =\dfrac{1}{\gamma} F_{\perp}^{\mathrm{rest}}
 \end{cases}\;.
\end{equation}
Allora il secondo principio covariante diventa:
\begin{equation}
\dev{\mathbf{p}}{t}=m\frac{\diff}{\diff{t}}(\gamma\mathbf{v})=\mathbf{F}\;.
\end{equation}
Esplicitando la derivata:
\begin{equation}
m\gamma\mathbf{a}+m\mathbf{v}\dev{\gamma}{t}=\mathbf{F}\;.
\end{equation}
Si ha che:
\begin{equation}
\frac{\diff}{\diff{t}}\left[\left(1-\frac{v^2}{c^2}\right)^{-1/2}\right]=-\frac{1}{2}\frac{\mathbf{v}\cdot\mathbf{a}}{(1-v^2/c^2)^{3/2}}=\gamma^3\frac{\mathbf{v}\cdot\mathbf{a}}{c^2}\;.
\end{equation}
Dunque otteniamo:
\begin{equation}
m\gamma\mathbf{a}+m\mathbf{v}\gamma^3\frac{\mathbf{v}\cdot\mathbf{a}}{c^2}=\mathbf{F}\;.
\end{equation}
Proiettando tutto nelle direzioni longitudinali ed ortogonali:
\begin{equation}
\begin{cases}
 (m\gamma)\vec{a}_{\perp}=\vec{F}_{\perp} \\
\\
m\gamma a_{||}+mv\gamma\dfrac{va_{||}}{c^2}=ma_{||}\gamma\left[1+\dfrac{v^2}{c^2}(1-v^2/c^2)^{-1/2}\right]=ma_{||}\gamma^3= F_{||}
\end{cases}\;.
\end{equation}
La prima relazione implica che $m(v)\equiv m\gamma$, mentre la seconda implica invece $m(v)=m\gamma^3$. Poiché sono contradditorie, non vale la pena considerarle e conviene dunque tenersi la massa come uno scalare.
\section{Meccanica analitica e relatività}
In meccanica analitica, l'azione $W$ era definita da:
\begin{equation}
 W=\int_{t_A}^{t_B} L[q(t),\dot{q}(t)]\;\diff{t}\;,
\end{equation}
dove $L$ è la funzione, detta \textit{Lagrangiana}, tale che $\delta W=0$ e soddisfacente a tal fine le equazioni di Eulero-Lagrange
\begin{equation}
\frac{\diff}{\diff{t}}\pdev{L}{\dot{q}}-\pdev{L}{q}=0\;.
\end{equation}
In analogia col caso analitico, vogliamo scrivere l'azione per una particella libera in ambito relativistico:
\begin{equation}
W=-mc^2\int_{t_A}^{t_B}\diff{\tau}=-mc^2\int_{t_A}^{t_B}\frac{\diff{t}}{\gamma}=-mc^2\int_{t_A}^{t_B}\diff{t}\sqrt{1-\frac{v^2}{c^2}}\;.
\end{equation}
Poniamo dunque $L(v)=(1-v^2/c^2)^{-1/2}$. Verifichiamo se questa è una buona definizione:
\begin{equation}
\begin{matrix}
\dfrac{\partial L}{\partial v}=m\gamma v =p \\
\\
\dfrac{\partial L}{\partial x}=0
\end{matrix}\qquad \implies \qquad \dev{p}{t}=0\;.
\end{equation}
Quindi è consistente. Scriviamo adesso l'energia:
\begin{equation}
E=pv-L=pv+\frac{mc^2}{\gamma}=m\gamma v^2+\frac{mc^2}{\gamma}=m\gamma\left(v^2+\frac{c^2}{\gamma^2}\right)=m\gamma c^2\;.
\end{equation}
Eliminando la dipendenza esplicita dalla velocità, cioè esprimendo l'energia in funzione dell'impulso, otteniamo la funzione 
\textit{Hamiltoniana} $H(p,q)$:
\begin{equation}
H=\sqrt{p^2c^2+m^2c^4}\;.
\end{equation}
Per una carica in presenza di campo elettromagnetico, abbiamo:
\begin{align}
L &= -\frac{mc^2}{\gamma}+q\left(\frac{\mathbf{v}}{c}\cdot \mathbf{A}(\mathbf{r},t)-\varphi(\mathbf{r},t)\right)\;, \\
H &= \sqrt{c^2\left(\mathbf{p}-\frac{q}{c}\mathbf{A}(\mathbf{r},t)\right)^2+m^2c^4}+q\varphi(\mathbf{r},t)\;.
\end{align}
\section{Esercizi}
\begin{enumerate}
\item Supponiamo di avere una particella in un sistema di riferimento $K$ con velocità $\mathbf{v}$ e con velocità $\mathbf{v}'$ in un secondo sistema di riferimento $K'$, in velocità relativa $u$ rispetto a $K$. Associo a $\mathbf{v},\mathbf{v}'$ le loro corrispondenti quadrivelocità $\stackrel{\sim}{v}$ e $\stackrel{\sim}{v}'$. Dimostrare che la legge di composizione delle velocità è uguale (sia per i trivettori, che per i quadrivettori). \\

Le trasformazioni di Lorentz sono date da:
\begin{equation}
 \begin{cases}
  \diff{t'}=\gamma_u(\diff{t}+u\diff{x}) \\
\\
\diff{x'}=\gamma_u(\diff{x}+u\diff{t})\;,
 \end{cases}
\end{equation}
con $v=\diff{x}/\diff{t}$ e $v'=\diff{x'}/\diff{t'}$. Sostituendo:
\begin{equation}
v'=\dev{x'}{t'}=\frac{\diff{x}+u\diff{t}}{\diff{t}+u\diff{x}}=\frac{v+u}{1+uv}\;.
\end{equation}
Dunque:
\begin{equation}
\stackrel{\sim}{v}'=\begin{pmatrix}
\gamma_{v'} \\
\gamma_{v'}v'
\end{pmatrix}\stackrel{?}{=}\begin{pmatrix}
\gamma_u & \gamma_uu	 \\
\gamma_uu & \gamma_u
\end{pmatrix}\begin{pmatrix}
\gamma_v \\
\gamma_vv
\end{pmatrix}\;.
\end{equation}
Svolgendo i prodotti ed eguagliando componente per componente otteniamo:
\begin{equation}
\begin{cases}
\gamma_{v'}=\gamma_u\gamma_v(1+uv) \\
\\
\gamma_u\gamma_v(u+v)=\gamma_{v'}v'
\end{cases}\;.
\end{equation}
Elevando al quadrato la prima equazione si ha:
\begin{equation}
\gamma_{v'}^2=\left(1-\left(\frac{u+v}{1+uv}\right)^2\right)^{-1}=\frac{(1+uv)^2}{(1+uv)^2-(u+v)^2}=\frac{(1+uv)^2}{(1-u)(1-v)}\;,
\end{equation}
da cui semplificando otteniamo l'uguaglianza. La seconda equazione è banalmente vera perché segue direttamente dalla legge di 
composizione delle velocità, quindi le leggi di composizione sono equivalenti.
\item Urto tra una particella di velocità $u$ e una parete che si avvicina con velocità $v$. Dire la velocità della particella dopo l'urto. \\
 
In un sistema di riferimento $K'$ in moto rispetto al primo con velocità $-\mathbf{v}$, la parete è in quiete. La velocità della particella 
in $K'$ sarà data da:
\begin{equation}
u'=\frac{u+v}{1+uv}\;.
\end{equation}
Dopo l'urto, la particella torna indietro con velocità $u_f'=-u'$. Ritornando adesso al sistema di riferimento $K$:
\begin{equation}
u_f=\frac{u_f'-v}{1-u_f'v}=\frac{-u'-v}{1+u'v}=-\left(\frac{\frac{u+v}{1+uv}+v}{1+\frac{u+v}{1+uv}\cdot v}\right)=
-\frac{u+2v+uv^2}{1+2uv+v^2}\;.
\end{equation}
Questa formula è consistente, infatti se $u=1$ (cioè la particella è un fotone), sostituendo nella relazione trovata otteniamo $u_f=-1$, 
cioè il fotone viene riflesso con velocità di modulo invariato (infatti la velocità della luce è costante) di segno opposto.
\item Stesso problema del precedente, ma le velocità $\mathbf{u}$ e $\mathbf{v}$ non sono collineari e formano un angolo $\alpha$. \\
 
Poiché $\mathbf{u}\equiv(u_x\cos\alpha,u_y\cos\alpha)$, si ha $\tan\alpha=u_y/u_x$. Dopo l'urto, la particella formerà un angolo 
$\alpha_f$ con la normale alla superficie dato da:
\begin{equation}
\tan\alpha_f=\frac{u_{fy}}{u_{fx}}\;.
\end{equation}
La quadrivelocità è $\stackrel{\sim}{u}\equiv(\gamma(u),\gamma(u)u_x,\gamma(u)u_y)$. Possiamo dunque definire gli angoli di incidenza e di riflessione in termini della quadrivelocità:
\begin{align}
&\tan\alpha=\frac{\tilde{u}_y}{\tilde{u}_x}\;, &\tan\alpha_f=\frac{\tilde{u}_{fy}}{\tilde{u}_{fx}}\;.
\end{align}
In termini matriciali:
\begin{equation}
\tilde{u}_f=L^{-1}\begin{pmatrix}
1 & 0 & 0 \\
0 & -1 & 0 \\
0 & 0 & 1
\end{pmatrix}L\tilde{u}=L^{-1}RL\tilde{u}\;.
\end{equation}
La matrice di Lorentz indotta dalla trasformazione è:
\begin{equation}
L=\begin{pmatrix}
\gamma(v) & \gamma(v)v & 0 \\
\gamma(v)v & \gamma(v) & 0 \\
0 & 0 & 1
\end{pmatrix}\;,
\end{equation}
quindi:
\begin{equation}
\tilde{u}_f=\begin{pmatrix}
\gamma(u_f) \\
\gamma(u_f)u_{fx} \\
\gamma(u_f)u_{fy}
\end{pmatrix}=\begin{pmatrix}
\gamma(v) & -\gamma(v)v & 0 \\
-\gamma(v)v & \gamma(v) & 0 \\
0 & 0 & 1
\end{pmatrix}\begin{pmatrix}
1 & 0 & 0 \\
0 & -1 & 0 \\
0 & 0 & 1
\end{pmatrix}\begin{pmatrix}
\gamma(v) & \gamma(v)v & 0 \\
\gamma(v)v & \gamma(v) & 0 \\
0 & 0 & 1
\end{pmatrix}\begin{pmatrix}
\gamma(u) \\
\gamma(u)u_x \\
\gamma(u)u_y
\end{pmatrix}\;.
\end{equation}
Eseguendo il prodotto matriciale, si ottiene:
\begin{equation}
\tilde{u}_f=\gamma^2(v)\begin{pmatrix}
1+v^2 & 2v & 0 \\
-2v & -(1+v^2) & 0 \\
0 & 0 & 1
\end{pmatrix}\begin{pmatrix}
\gamma(u) \\
\gamma(u)u_x \\
\gamma(u)u_y
\end{pmatrix}\;,
\end{equation}
e dunque scrivendo le componenti di $\tilde{u}_f$ possiamo ricavare l'angolo di rifrazione:
\begin{equation}
\tan\alpha_f=\frac{1}{\gamma^2(v)}\frac{\sin\alpha}{(1+v^2)\cos\alpha+2v/u}\;.
\end{equation}
Se la parete è ferma, $v=0$ e si ha $\tan\alpha=\tan\alpha_f$, coerente con la meccanica classica.
\item È possibile che un fotone ``decada'' in due fotoni con velocità formanti un angolo $\theta_{12}$? \\

Ad un fotone è associato un quadri-impulso $p(\gamma)=\hbar(\omega,\mathbf{k})$. Allora, scrivendo la conservazione del 
quadri-impulso $p=p_1+p_2$ ed elevando al quadrato si ha (ricordando che il modulo del quadri-impulso di un fotone è zero):
\begin{equation}
p^2=p_1^2+2p_1\cdot p_2+p_2^2\qquad \Longrightarrow\qquad 0=p_1\cdot p_2=-\omega_1\omega_2(\cos\theta_{12}-1)\;.
\end{equation}
Quindi la conservazione è rispettata se e solo se $\theta_{12}=0$, cioè i due fotoni prodotti sono collineari.
\item  Un protone e un neutrone formano uno stato legato denominato \textit{deutone}. Il difetto di massa $B$ è dato da $B=M_p+M_n-M_d$. Per un protone e un neutrone $M_p = M_n= 940$ MeV e $B=2.2$ MeV. Mostrare che la reazione $p+n\to d$ non è consistente con la conservazione del quadri-impulso. \\

I quadri-impulsi delle due particelle sono dati da:
\begin{align}
&p_{1\mu}=(M_p\gamma, M_p\gamma\mathbf{v}), &p_{2\mu}=(M_2\gamma,M_n\gamma\mathbf{v})\;.
\end{align}
Ci poniamo nel sistema del centro di massa delle due particelle: il tri-impulso iniziale, così come quello finale, è zero, quindi 
bisogna scrivere la conservazione dell'energia (componente zero):
\begin{equation}
M_d=M_p\gamma(v)+M_n\gamma(v)>M_p+M_n=B+M_d\;,
\end{equation}
da cui otteniamo $B<0$, che contraddice l'ipotesi, dunque questa reazione è impossibile.
\item Consideriamo adesso la reazione $p+n\to d+\gamma$, dove $\gamma$ è un fotone e verifichiamone la consistenza con la conservazione del quadri-impulso. \\

Nel sistema del c.d.m.:
\begin{equation}
M_p\gamma(v_p)+M_n\gamma(v_n)=M_d\gamma(v_d)+E_{\gamma}\;,
\end{equation}
dove abbiamo considerato anche l'energia del fotone. Supponiamo adesso che $v_p,v_n \ll 1$. Osserviamo che in questo caso, per non 
violare la conservazione del tri-impulso, il deutone prodotto deve avere una velocità non nulla. Inoltre deve essere soddisfatta:
\begin{equation}
M_p\gamma(v_p)\mathbf{v}_p+M_n\gamma(v_n)\mathbf{v}_n=M_d\gamma(v_d)\mathbf{v}_d+\mathbf{p}\;.
\end{equation}
Possiamo condensare le due equazioni in una usando il quadri-impulso:
\begin{equation}
p_p+p_n=p_d+p_{\gamma}\;.
\end{equation}
Elevando questa al quadrato:
\begin{equation}
(M_p+M_n)^2=(E_d+E_{\gamma})^2\;.
\end{equation}
Da cui otteniamo:
\begin{equation}
M_p+M_n=\sqrt{p^2+M_d^2}+p\;.
\end{equation}
Isoliamo la radice ed eleviamo al quadrato:
\begin{equation}
 p^2+M_d^2 = (M_n+M_p-p)^2 =M_n^2+M_p^2+p^2+2M_nM_p-2M_np-2M_pp\;,
\end{equation}
e quindi:
\begin{equation}
p=\frac{-M_d^2+(M_n+M_p)^2}{2(M_n+M_p)}=\frac{(M_n+M_p+M_d)(M_n+M_p-M_d)}{2(M_n+M_p)}=B\frac{2(M_n+M_p)-B}{2(M_n+M_p)}\;.
\end{equation}
Concludiamo dunque che:
\begin{equation}
|\mathbf{p}|=B\left(1-\frac{B}{2(M_n+M_p)}\right)\;.
\end{equation}
Il rapporto è di una parte su $10^5$, dunque la differenza di massa $B$ viene quasi interamente assorbita dal fotone. Di conseguenza, il 
deutone prodotto ha una velocità quasi nulla (non totalmente nulla, altrimenti verrebbe violata la conservazione del tri-impulso).
\item Consideriamo la reazione $e^-+e^+\to z_0$, con $m_0\simeq 90$ GeV. Vogliamo sapere, in termini di energia, se nel sistema del laboratorio è meno dispendioso tenere fermo l'elettrone e spararvi contro il positrone, oppure accelerare entrambi. \\

Nel sistema del c.d.m. si ha $p_++p_-=p_0$. In termini di energia: $E_++E_-=E_0$ e poiché $E_+=E_-=E$ si ha $2E=E_0$, e quindi $E\simeq
45$ GeV. Nel sistema del laboratorio, invece, si ha:
\begin{equation}
\begin{cases}
E^+_{\mathrm{lab}}+m_e=M_0\gamma(v_0)\;, \\
\\
m\gamma(v_+)v_+=M_0\gamma(v_0)v_0
\end{cases}\;.
\end{equation}
Riprendiamo la relazione dei quadri-impulsi e facciamone il quadrato:
\begin{equation}
p_+^2+p_-^2+2p_+\cdot p_-=p_0^2\qquad \Longrightarrow\qquad 2m_e^2+2E^+_{\mathrm{lab}}m_e=M_0^2\;,
\end{equation}
da cui:
\begin{equation}
E^+_{\mathrm{lab}}=\frac{M_0^2-2m_e^2}{2m_e^2}=\frac{M_0}{2}\left(\frac{M_0}{m_e}-2\frac{m_e}{M_0}\right)=\frac{M_0^2}{2m_e}\left(1-2\left(\frac{m_e}{M_0}\right)^2\right) \simeq 45\cdot 10^4 \mbox{GeV}\;.
\end{equation}
Quindi l'energia necessaria in questo caso è molto maggiore che nel caso precedente.
\item Consideriamo due particelle con quadri-impulsi:
 \begin{align}
&p_1=(E_1,\mathbf{p}_1), &p_2=(E_2,\mathbf{p}_2)\;.
 \end{align}
Descriviamo nel caso più generale il centro di massa. La formula Newtoniana non è più applicabile in quanto presuppone la validità del 
terzo principio per sistemi isolati. \\

Il quadri-impulso totale del sistema è dato da:
\begin{equation}
P=(E_1+E_2,\mathbf{p}_1+\mathbf{p}_2)\;.
\end{equation}
Possiamo sempre trovare un sistema di riferimento in cui la componente spaziale del quadri-impulso totale sia solo lungo $\hat{x}$, cioè:
\begin{equation}
P=(E_1+E_2,(\mathbf{p}_1+\mathbf{p}_2)_x,0,0)\;,
\end{equation}
e quindi, poiché l'impulso nel sistema del c.d.m. deve essere $P_{cm}=(E_1'+E_2',0,0,0)$, si ottiene:
\begin{equation}
p_x'=\gamma(v)\left(p_{1x}+p_{2x}-v(E_1+E_2)\right)=0\;.
\end{equation}
La velocità $\mathbf{v}$ sarà dunque data da:
\begin{equation}
\mathbf{v}=\frac{\mathbf{p}_1+\mathbf{p}_2}{E_1+E_2}\;.
\end{equation}
\item Consideriamo i decadimenti:
 \begin{align}
&B\to J/\Psi + K_0\;, &K_0\to \pi^++\pi^-\;.
 \end{align}
Calcolare l'energia di $K_0$. \\

$p_B-p_K=p_{J/\Psi}$, elevando al quadrato:
\begin{equation}
M_B^2+M_K^2-2M_BE_K=M^2_{J/\Psi}\;,
\end{equation}
da cui:
\begin{equation}
E_K=\frac{M^2_B+M_K^2-M^2_{J/\Psi}}{2M_B}\;,
\end{equation}
e per l'impulso:
\begin{equation}
\mathbf{p}_K^2=E_K^2-M_K^2=\frac{(M_B^2-M_K^2-M^2_{J/\Psi})^2}{4M_B^2}\;,
\end{equation}
cioè:
\begin{align}
|\mathbf{p}_K| &=\frac{M_B^2-M_K^2-M^2_{J/\Psi}}{2M_B}\;, \\
|\mathbf{v}_K| &=\frac{|\mathbf{p}_K|}{E_K}\ll 1\;.
\end{align}
\end{enumerate}
\chapter{Fisica Statistica}
\section{Richiami di Meccanica Classica}
Consideriamo un sistema conservativo classico con $n$ gradi di libertà. Per $N$ punti materiali, $n$ sarà uguale a  $3N$. Supponiamo di 
avere un set di coordinate generalizzate per il sistema $q_1,q_2,\ldots,q_n$, che possono essere Cartesiane, polari o qualunque set 
conveniente di coordinate. Le velocità generalizzate associate a queste coordinate sono $\dot{q}_1,\dot{q}_2,\ldots, \dot{q}_n$. \\
L'espressione della seconda legge di Newton tramite le equazioni di moto di Eulero-Lagrange è:
\begin{equation}
\label{sec1_eulerolagrange}
\frac{\diff}{\diff{t}}\pdev{L}{\dot{q_i}}-\pdev{L}{q_i}=0\;,
\end{equation}
dove per un semplice sistema non relativistico la \textit{Lagrangiana} $L$ è data da:
\begin{equation}
L(q_i,\dot{q}_i)=T-V\;.
\end{equation}
$T$ è l'energia cinetica e $V$ è l'energia potenziale. L'equazione \eqref{sec1_eulerolagrange} è facilmente verificata se le $q_i$ sono coordinate cartesiane, infatti si ha:
\begin{equation}
L(q_i,\dot{q}_i)=\frac{1}{2}\sum_j M_j\dot{q}_j^2 - V\;.
\end{equation}
e, posto $q_i=x$ si ha:
\begin{equation}
 M\ddot{x}=-\pdev{V}{x}\;.
\end{equation}
Ma $-\partial V\partial x$ altri non è che la componente $x$ della forza $\mathbf{F}$; pertanto si ha semplicemente:
\begin{equation}
 F_x=M\ddot{x}\;.
\end{equation}
La forma Hamiltoniana delle equationi del moto sostituisce le $n$ equazioni differenziali del second'ordine \eqref{sec1_eulerolagrange} con $2n$ equazioni differenziali del prim'ordine. Definiamo i \textit{momenti generalizzati} come:
\begin{equation}
\label{sec1_generalizedmomenta}
 p_i=\pdev{L}{\dot{q}_i}\;.
\end{equation}
La funzione \textit{Hamiltoniana} $H$ è data da:
\begin{equation}
H(p_i,q_i)=\sum_i p_i\dot{q}_i - L(q_i,\dot{q}_i)\;.
\end{equation}
Allora
\begin{align}
\diff{H} &= \sum_i\left(\pdev{H}{p_i}\diff{p_i}+\pdev{H}{q_i}\diff{q_i}\right) \notag \\
&=\sum_i(p_i\diff{\dot{q}_i}+\dot{q}_i\diff{p_i})-\sum_i\left(\pdev{L}{q_i}\diff{q_i}+\pdev{L}{\dot{q}_i}\diff{\dot{q}_i}\right)\;. \label{sec1_diffham}
\end{align}
I termini in $\diff{\dot{q}_i}$ si eliminano per la definizione \eqref{sec1_generalizedmomenta} delle $p_i$. Inoltre, dalle equazioni di Eulero-Lagrange \eqref{sec1_eulerolagrange} si vede che:
\begin{equation}
\pdev{L}{q_i}=\dot{p}_i\;.
\end{equation}
Quindi, dalla \eqref{sec1_diffham}, si deve avere:
\begin{align}
&\pdev{H}{p_i}=\dot{q}_i\;, &\pdev{H}{q_i}=-\dot{p}_i\;.
\end{align}
Queste sono le equazioni di moto di Hamilton.
\pagebreak
\section{Sistemi ed ensembles}
Lo sviluppo di un sistema composto da $N$ atomi nel tempo è noto quando si conoscono i valori delle $6N$ coordinate e momenti $p$ e $q$ 
come funzioni del tempo. È possibile rappresentare graficamente tale evoluzione come un'unica orbita nello spazio $6N$-dimensionale 
delle $p$ e delle $q$. Tale spazio è detto \textit{spazio delle fasi} o \textit{spazio} $\Gamma$ del sistema. \\

Le quantità fisiche di particolare interesse negli equilibri termodinamici sono quasi sempre le medie temporali su un segmento della 
traiettoria nello spazio delle fasi del sistema, con la media presa su un appropriato intervallo di tempo. Il problema è che per 
determinati sistemi, un intervallo di tempo appropriato può essere anche diversi anni. L'\textit{idoneità} dell'intervallo di tempo su 
cui eseguire le misure e le medie dipende da un tempo caratteristico detto \textit{tempo di rilassamento}. Il tempo di rilassamento 
descrive approssimativamente il tempo necessario a smorzarsi per una fluttuazione (spontanea o indotta) delle proprietà del sistema. 
Il valore del tempo di rilassamento dipende dalle particolari condizioni iniziali. J. Willard Gibbs fece un significativo passo avanti 
nella risoluzione del problema di calcolare valori medi di quantità fisiche. Egli suggerì, anzichè usare medie temporali, di immaginare 
un insieme di sistemi simili, ma opportunamente casuali, e di eseguire delle medie su questo insieme ad un istante fissato. Il gruppo 
di sistemi simili è chiamato \textit{ensemble}. \\

Un ensemble è composto da una moltitudine di sistemi costruiti l'uno simile all'altro. Ogni sistema nell'ensemble è una replica esatta 
del sistema attuale che si sta studiando ed è equivalente, ai fini pratici, ad esso. L'ensemble è opportunamente causale nel senso che 
ogni configurazione di coordinate e velocitè accessibile all'attuale sistema nel corso del tempo è rappresentato nell'ensemble da uno o 
più sistemi in ogni istante. Si dice che l'ensemble rappresenti il sistema. Gibbs propose di sostituire le medie temporali con le medie 
sugli ensemble, che sono medie su tutti i sistemi appartenenti all'ensemble ad un istante fissato. L'incertezza che si ha nella 
conoscenza delle condizioni iniziali di un sistema e le conseguenti incertezze sulle medie temporali sono perfettamente descritte dalla 
media sull'ensemble. \\
Il prossimo problema è stabilire come costruire opportunamente gli ensemble. Se il sistema da rappresentare è in equilibrio termico, 
possiamo ragionevolmente richiedere che la media sull'ensemble sia indipendente dal tempo. Le proprietà fisiche medie macroscopiche di 
un sistema in equilibrio termico non cambiano col tempo; quindi l'ensemble rappresentativo deve essere tale che le medie non dipendano 
dal particolare istante in cui esse sono state misurate.
\pagebreak
\section{Il teorema di Liouville}
Un ensemble può essere descritto dando il numero di sistemi:
\begin{equation}
P(\mathbf{p},\mathbf{q})\diff{\mathbf{p}}\diff{\mathbf{q}}\;,
\end{equation}
nell'elemento di volume $\diff{\mathbf{p}}\diff{\mathbf{q}}$ dello spazio delle fasi. Per $N$ punti materiali:
\begin{align}
\diff{\mathbf{p}}&=\prod_{i=1}^{3N}\diff{p_i}\;,\notag \\
\diff{\mathbf{q}} &=\prod_{i=1}^{3N}\diff{q_i}\;,
\end{align}
cioè, specifichiamo un ensemble dando la densità di sistemi nello spazio delle fasi (spazio $\Gamma$). Diremo che i sistemi 
sono rappresentati in senso statistico dall'ensemble $P(\mathbf{p},\mathbf{q})$. \\

La media sull'ensemble di una certa quantità $f(\mathbf{p},\mathbf{q})$ è dunque definita come:
\begin{equation}
\bar{f}=\frac{\int f(\mathbf{p},\mathbf{q})P(\mathbf{p},\mathbf{q})\diff{\mathbf{p}}\diff{\mathbf{q}}}{\int P(\mathbf{p},\mathbf{q})\diff{\mathbf{p}}\diff{\mathbf{q}}}\;.
\end{equation}
Volendo soddisfare la richiesta che la media sull'ensemble non dipenda dal tempo, bisogna imporre che $\partial P/\partial t=0$. 
Inoltre, vogliamo anche che il numero di stati sia costante lungo la traiettoria nello spazio delle fasi; per questo scriviamo un'equazione di continuità per $P$:
\begin{equation}
 \pdev{P}{t}+\vec{\nabla}\cdot (P\mathbf{v})=0\;,
\end{equation}
dove $\mathbf{v}$ è la velocità e $\vec{\nabla}\cdot$ denota l'operatore divergenza nello spazio delle fasi $6N$-dimensionale. Esplicitando la divergenza troviamo:
\begin{equation}
 \pdev{P}{t}+\sum_{i=1}^{3N}\left[\frac{\partial}{\partial q_i}(P\dot{q}_i)+\frac{\partial}{\partial p_i}(P\dot{p}_i)\right]=0\;,
\end{equation}
ed esplicitando le derivate dentro la sommatoria:
\begin{equation}
 \pdev{P}{t}=\sum_i \left[\pdev{P}{q_i}\dot{q}_i+\pdev{P}{p_i}\dot{p}_i+P\left(\frac{\partial}{\partial q_i}\dot{q}_i+\frac{\partial}
 {\partial p_i}\dot{p_i}\right)\right]=0\;.
\end{equation}
Il termine in parentesi tonde è identicamente nullo; infatti, dalle equazioni di Hamilton:
\begin{equation}
 \frac{\partial}{\partial q_i}\dot{q}_i+\frac{\partial}{\partial p_i}\dot{p}_i=\frac{\partial}{\partial q_i}\pdev{H}{p_i}-
 \frac{\partial}{\partial p_i}\pdev{H}{q_i}=0\;,
\end{equation}
poiché:
\begin{equation}
\frac{\partial^2 H}{\partial q_i\partial p_i}=\frac{\partial^2 H}{\partial p_i\partial q_i}\;.
\end{equation}
Sostituendo otteniamo il \textit{teorema di Liouville}:
\begin{equation}
 \pdev{P}{t}+\sum_i \left[\pdev{P}{q_i}\dot{q}_i+\pdev{P}{p_i}\dot{p}_i\right]\equiv \dev{P}{t}=0\;. \label{sec2_liouville}
\end{equation}
Il teorema di Liouville afferma sostanzialmente che il tasso di variazione di $P$ lungo una linea di flusso è zero. Avevamo precedentemente richiesto, affinché un ensemble fosse una soddisfacente rappresentazione di un sistema in equilibrio, che il valore della densità della funzione $P(\mathbf{p},\mathbf{q})$ fosse indipendente dal tempo in ogni punto dello spazio delle fasi. Imponendo $\partial P/\partial t=0$ in accordo con questa richiesta nella \eqref{sec2_liouville} si ottiene:
\begin{equation}
 \sum_{i=1}^{3N}\left(\pdev{P}{q_i}\dot{q}_i+\pdev{P}{p_i}\dot{p}_i\right)=0\;,
\end{equation}
o:
\begin{equation}
 \mathbf{v}\cdot\vec{\nabla}\;P=0\;. \label{sec2_grad}
\end{equation}
cioè il sistema di muove su una superficie a $P$ costante. \\

Un modo semplice di costruire un ensemble in equilibrio è di avere $P$ inizialmente distribuito uniformemente in tutto lo spazio delle 
fasi. In questo modo, si ha che $\partial P/\partial q_i$ e $\partial P/\partial p_i$ sono ambedue singolarmente nulli e, dalla \eqref{sec2_liouville}, si ottiene che l'ensemble così scelto è uniforme. \\
Una condizione ancora più generale per l'equilibrio statistico è di prendere $P$ come funzione solamente di certe quantità che sono 
costanti del moto del sistema. Sia $\alpha$ una quantità, come ad esempio l'energia, costante del moto, cioè tale che $\partial\alpha/
\partial t=0$. Se $\alpha(\mathbf{p},\mathbf{q})$ è una costante del moto, il sistema si muove lungo una superficie con $\alpha$ 
costante e, poichè $P$ è stata scelta come funzione unicamente di $\alpha$, ciò implica che il sistema si muove lungo una superficie con 
$P$ costante. Dunque la \eqref{sec2_grad} è verificata, e $\partial P/\partial t=0$ è consistente con il teorema di Liouville.
\pagebreak
\section{Entropia in Meccanica Statistica}
Dai principi della termodinamica sappiamo che l'entropia $S$ ha le seguenti proprietà:
\begin{enumerate}
 \item $\diff{S}$ è un differenziale esatto ed è uguale a $\delta Q/T$ per un processo reversibile, dove $\delta Q$ è la quantità di calore 
 scambiata dal sistema;
 \item l'entropia è additiva: $S=S_1+S_2$. L'entropia di un sistema combinato è la somma delle entropie delle singole parti;
 \item $\Delta S\ge 0$. Se lo stato di un sistema chiuso è dato macroscopicamente ad un certo istante, lo stato più probabile ad un 
 certo istante successivo è uno con entropia maggiore od uguale.
 \end{enumerate}
Definiamo l'entropia $\sigma$ di un sistema (in Meccanica Statistica classica) in equilibrio come:
\begin{equation}
 \sigma=\log \Delta\Gamma\;,
\end{equation}
dove $\Delta\Gamma$ è il volume dello spazio delle fasi accessibile al sistema, per esempio il volume corrispondente a valori dell'energia compresi tra $E_0$ e $E_0+\delta E$. Notiamo che le variazioni di entropia sono indipendenti dalle unità usate per misurare 
$\Delta\Gamma$. Poichè $\Delta\Gamma$ è un volume nello spazio delle fasi di $N$ punti materiali, esso ha le dimensioni:
$$
(\mbox{Momento}\times \mbox{Lunghezza})^{3N}=(\mbox{Azione})^{3N}\;.
$$
Sia $h$ l'unità di azione; allora $\Delta\Gamma/h^{3N}$ è una quantità adimensionale. Se definiamo:
\begin{equation}
 \sigma=\log\frac{\Delta\Gamma}{h^{3N}}=\log\Delta\Gamma-3N\log h\;,
\end{equation}
osserviamo che per le variazioni si ha:
\begin{equation}
 \delta\sigma=\delta\log\Delta\Gamma\;,
\end{equation}
indipendentemente dalle unità di misura. Osserviamo immediatamente che $\sigma$ è additiva. Consideriamo un sistema costituito da due parti, uno con $N_1$ punti materiali, l'altro con $N_2$ punti materiali. Allora:
\begin{equation}
N=N_1+N_2\;,
\end{equation}
e lo spazio delle fasi del sistema combinato è il prodotto degli spazi delle fasi delle singole parti:
\begin{equation}
 \Delta\Gamma=\Delta\Gamma_1\Delta\Gamma_2\;.
\end{equation}
L'additività dell'entropia segue direttamente:
\begin{equation}
 \sigma=\log\Delta\Gamma=\log(\Delta\Gamma_1\Delta\Gamma_2)=\log\Delta\Gamma_1+\log\Delta\Gamma_2=\sigma_1+\sigma_2\;.
\end{equation}
\pagebreak
\section{Esempi elementari di distribuzione di probabilità ed entropia}
Consideriamo un sistema di $N$ particelle indipendenti, aventi ciascuna un momento magnetico $\mu$ che può essere parallelo o 
antiparallelo rispetto ad un campo magnetico esterno $H$. L'energia di ogni particella è $E=\pm \mu H$, a seconda dell'orientazione 
dei momenti magnetici. Calcoliamo la distribuzione di probabilità del momento magnetico totale $M$ del sistema in assenza di campo 
magnetico esterno. In queste condizioni, la proiezione di ogni momento può essere con uguale probabilità $\pm \mu$. Siamo interessati 
alle configurazioni che risultano in $\frac{1}{2}(N+n)$ momenti positivi e $\frac{1}{2}(N-n)$ momenti negativi. Osserviamo innanzitutto 
che la probabilità (normalizzata) di uno specifico microstato è $(1/2)^N$, poiché per ogni singolo momento c'è una probabilità 
di $1/2$ che abbia l'orientamento richiesto, e ci sono $N$ particelle indipendenti. Inoltre, le $N$ particelle possono essere 
riordinate fra di loro in $N!$ modi, molti dei quali non risultano in configurazioni distinte nei gruppi di $\frac{1}{2}(N+n)$ e $
\frac{1}{2}(N-n)$ particelle. I riordinamenti interni a questi due gruppi, infatti, non sono considerati distinti e sono 
rispettivamente $[\frac{1}{2}(N+n)]!$ e $[\frac{1}{2}(N-n)]!$. Perciò, il numero totale $W(n)$ di configurazioni indipendenti che 
risultano in un momento netto $M=n\mu$ è:
\begin{equation}
\label{sec5_configurations}
 W(n)=\frac{N!}{[\frac{1}{2}(N+n)]![\frac{1}{2}(N-n)]!}\;,
\end{equation}
e la probabilità $w(M)$ di un momento netto $M=n\mu$ è ottenuta moltiplicando la \eqref{sec5_configurations} per la probabilità $(1/2)^N$ di una specifica configurazione, cioè:
\begin{equation}
\label{sec5_probability}
 w(M)=w(n\mu)=\left(\frac{1}{2}\right)^N\frac{N!}{[\frac{1}{2}(N+n)]![\frac{1}{2}(N-n)]!}\;.
\end{equation}
Per grandi valori dei fattoriali si può usare l'approssimazione di Stirling:
\begin{align}
x! &\simeq \sqrt{2\pi x}x^xe^{-x}\;, \notag \\
\log x! &\simeq \frac{1}{2}\log(2\pi x)+x\log x-x=\frac{1}{2}\log(2\pi)+\left(x+\frac{1}{2}\right)\log x -x\;. \label{sec5_stirling}
\end{align}
Dunque, prendendo i logaritmi di entrambi i membri e usando la \eqref{sec5_stirling}, la \eqref{sec5_probability} diventa:
\begin{align}
 \log w(M)&\simeq -N\log 2-\frac{1}{2}\log 2\pi+\left(N+\frac{1}{2}\right)\log N \notag \\
 &- \frac{1}{2}(N+n+1)\log\left[\frac{N}{2}\left(1+\frac{n}{N}\right)\right] \notag \\
 &- \frac{1}{2}(N-n+1)\log\left[\frac{N}{2}\left(1-\frac{n}{N}\right)\right]\;.
\end{align}
Per $n\ll N$, possiamo sviluppare in serie di Taylor:
\begin{equation}
 \log\left(1\pm\frac{n}{N}\right)=\pm\frac{n}{N}-\frac{n^2}{2N^2}+\cdots
\end{equation}
e dunque:
\begin{align}
 \log w(M) &\simeq -\frac{1}{2}\log 2\pi-N\log 2+\left(N+\frac{1}{2}\right)\log N \\
 &-\frac{1}{2}(N+n+1)\left[\log N-\log 2+\frac{n}{N}-\frac{n^2}{2N^2}\right] \notag \\
 &-\frac{1}{2}(N-n+1)\left[\log N-\log 2-\frac{n}{N}-\frac{n^2}{2N^2}\right]\;. \\
\end{align}
Sommando i termini simili, otteniamo:
\begin{equation}
\log w(M)\simeq -\frac{1}{2}\log N+\log 2-\frac{1}{2}\log(2\pi)-\frac{n^2}{2N}+\frac{n^2}{2N^2}\;.
\end{equation}
Il termine $n^2/2N^2$ è un infinitesimo di ordine superiore rispetto agli altri termini, e dunque può essere trascurato. Otteniamo perciò:
\begin{equation}
 \log w(M) \simeq -\frac{1}{2}\log\left(\frac{\pi N}{2}\right)-\frac{n^2}{2N}\;,
\end{equation}
e dunque:
\begin{equation}
 w(M)\simeq \left(\frac{2}{\pi N}\right)^{1/2}e^{-n^2/2n}\;.
\end{equation}
Osserviamo da questo risultato che la magnetizzazione ha una distribuzione Gaussiana centrata nello zero. Pertanto il valor medio 
della magnetizzazione in assenza di campo esterno è zero. La distribuzione di probabilità ha il suo unico  massimo in questo punto, per 
cui il valore più probabile coincide con il valor medio. Riprendiamo in considerazione l'espressione della distribuzione di probabilità \eqref{sec5_probability} e calcoliamo $\sum_{n=-N}^N w(n)$: posto $\frac{1}{2}(N-n)=k$ si ha
\begin{equation}
\sum_{n=-N}^N w(n)=\sum_{k=0}^N \frac{N!}{(N-k)!k!}\left(\frac{1}{2}\right)^k\left(\frac{1}{2}\right)^{N-k}= \left(\frac{1}{2}+\frac{1}{2}\right)^N=1\;.
\end{equation}
Concludiamo che la distribuzione di probabilità risulta già normalizzata. Per calcolarne media e varianza, introduciamo la \textit{funzione generatrice} di una distribuzione di probabilità. Se $p(k)$ è una distribuzione di probabilità normalizzata, si definisce funzione generatrice di $p(k)$:
\begin{equation}
F(x)=\sum_{k=0}^N p(k)x^k\;.
\end{equation}
Si hanno le seguenti relazioni:
\begin{itemize}
 \item $F(1)=1$ (condizione di normalizzazione);
 \item $F'(1)=\bar{k}$;
 \item $F''(1)=\bar{k(k-1)}=\bar{k^2-k}$.
\end{itemize}
Nel caso della $w(k)$ data da \eqref{sec5_probability} generalizzata a $p,q$ qualunque (purché $q=1-p$), si ha:
\begin{equation}
F(x)=\sum_{k=0}^N \binom{N}{k}(px)^k(1-p)^{N-k}=(px-p+1)^N\;,
\end{equation}
e dunque:
\begin{align}
\bar{k}&=F'(1)=Np\;, \\
\Delta k &= \sqrt{Npq}\;.
\end{align}
Allora l'entropia $\sigma$, definita come $\sigma(k)=\log w(k)$ sarà:
\begin{equation}
\sigma(k)=\log N!-\log k! -\log(N-k)!+k\log p+(N-k)\log(1-p)\;.
\end{equation}
Lo stato più probabile corrisponde a quello caratterizzato da entropia massima (il che giustifica l'uso del 
logaritmo, in quanto è iniettivo e dunque il punto di massimo di $w(k)$ coincide col punto di massimo di $\sigma(k)$). Imponiamo pertanto che $\sigma'(k)=0$, riscrivendo prima l'entropia usando l'approssimazione di Stirling \eqref{sec5_stirling}:
\begin{equation}
\sigma(k)=\log N!-\log k!-k+(k-N)\log(N-k)-(N-k)+k\log p+(N-k)\log(1-p)\;.
\end{equation}
Dunque:
\begin{equation}
\sigma'(k)=-\log k-1-1+\log(N-k)+1+1+\log p-\log(1-p)=\log\frac{p}{1-p}-\log\frac{k}{1-k}\stackrel{!}{=}0\;,
\end{equation}
da cui:
\begin{equation}
\frac{p}{1-p}=\frac{k}{1-k}\qquad  \Longrightarrow\qquad k_{\mathrm{max}}=Np=\bar{k}\;.
\end{equation}
Dunque il valore in cui l'entropia è massima coincide con il valor medio della distribuzione $w(k)$. Sviluppiamo adesso l'entropia in serie di Taylor intorno al punto $k=\bar{k}$:
\begin{equation}
\sigma(k)=\sigma(\bar{k})+\frac{1}{2}\sigma''(\bar{k})(k-\bar{k})^2+\cdots
\end{equation}
dove non figura la derivata prima perché abbiamo imposto che $\bar{k}$ sia un massimo di $\sigma(k)$. Allora la probabilità $w(k)$, sarà:
\begin{equation}
w(k)=e^{\sigma(k)}=A\exp\left[\frac{1}{2}\sigma''(\bar{k})(k-\bar{k})^2\right]\;,
\end{equation}
dove:
\begin{equation}
\sigma''(\bar{k})=\left. -\frac{1}{N-k}-\frac{1}{k}\right|_{k=\bar{k}}=-\frac{1}{Npq}\;.
\end{equation}
Quindi, in conclusione:
\begin{equation}
w(k)=Ae^{-{(k-\bar{k})^2/2Npq}}\;.
\end{equation}
$A$ è la costante di normalizzazione. Sapendo che $\int_{-\infty}^{\infty} e^{-x^2}\;dx=\sqrt{\pi}$ si trova:
\begin{equation}
A=\sqrt{\frac{1}{2\pi Npq}}\;,
\end{equation}
che sostituita in $w(k)$ restituisce:
\begin{equation}
w(k)=\sqrt{\frac{1}{2\pi Npq}}\exp\left(-\frac{(k-Np)^2}{2Npq}\right)\;.
\end{equation}
Nel caso degli spin magnetici, se il campo magnetico esterno $H$ è zero, il valore del momento magnetico totale netto $n$ più probabile è zero, che può essere raggiunto in $w(0)$ modi possibili, cioè:
\begin{equation}
w(0)=\frac{N!}{\left[\left(\frac{N}{2}\right)!\right]^2}\;.
\end{equation}
In questo caso, l'entropia sarà data da:
\begin{equation}
\sigma(0)=\log(w(0))=N\log 2\;.
\end{equation}
\pagebreak
\section{Calcolo dell'entropia di un gas perfetto tramite gli ensemble microcanonici}
Supponiamo di avere un ensemble microcanonico di $N$ punti materiali non interagenti di massa $M$ confinati nel volume $V$ aventi energia compresa tra $U-\delta U$ e $U$. Il volume nello spazio delle fasi sarà dato da:
\begin{equation}
\Delta\Gamma =\int \diff{q_1}\cdots \diff{q_{3N}}\int\diff{p_1}\cdots\diff{p_{3N}}=V^N\int\diff{p_1}\cdots\diff{p_{3N}}\;,
\end{equation}
dove l'integrale dei momenti è da calcolare tenendo presente il vincolo:
\begin{equation}
U-\delta U\le \frac{1}{2M}\sum_{i=1}^{3N} p_i^2\le U\;, \label{sec6_constraint}
\end{equation}
per come l'ensemble è stato costruito. Il volume accessibile nello spazio dei momenti è quello di un guscio di spessore $(\delta U)(M/2U)^{1/2}$ in una ipersfera di raggio $(2MU)^{1/2}$. È possibile provare che per un sistema costituito da un numero abbastanza grande di punti materiali il valore di $\log\Delta\Gamma$ non dipende dal valore di $\delta U$ (se lo fosse, avremmo il problema di decidere un valore di $\delta U$). Dimostriamo questo fatto. Scriviamo:
\begin{equation}
V(R)=CR^{\nu}\;,
\end{equation}
per indicare il volume di una sfera $\nu$-dimensionale di raggio $R$. Il volume di un guscio di spessore $s$ alla superficie dell'ipersfera è:
\begin{equation}
V_s=V(R)-V(R-s)=C[R^{\nu}-(R-s)^{\nu}]=CR^{\nu}\left[1-\left(1-\frac{s}{R}\right)^{\nu}\right]\;,
\end{equation}
o, per definizione della funzione esponenziale:
\begin{equation}
V_s\simeq CR^{\nu}[1-e^{-s\nu/R}]\;.
\end{equation}
Pertanto se $\nu$ è sufficientemente grande, e $s\nu\gg R$, $V_s$ è praticamente il volume $V(R)$ dell'intera ipersfera. Possiamo pertanto sostituire il vincolo \eqref{sec6_constraint} con la condizione di rilassamento:
\begin{equation}
0\le \frac{1}{2M}\sum_{i=1}^{3N} p_i^2\le U\;.
\end{equation}
In altre parole, vogliamo valutare il volume di una sfera $3N$-dimensionale di raggio $(2MU)^{1/2}$. Consideriamo l'integrale:
\begin{align}
I &= \int_{-\infty}^{\infty}e^{-(x_1^2+x_2^2+\cdots +x_{\nu}^2)}\diff{x_1}\diff{x_2}\cdots\diff{x_{\nu}} \notag \\
&=\left(\int_{-\infty}^{\infty}e^{-x^2}\diff{x}\right)^{\nu}=\pi^{\nu/2}\;.
\end{align}
Possiamo inoltre scrivere (usando coordinate polari):
\begin{align}
I =\int_0^{\infty}e^{-r^2}r^{\nu-1}\Omega_{\nu}\,\diff{r}=\frac{\Omega_{\nu}}{2}\int_0^{\infty}e^{-t}t^{(\nu-2)/2}\,\diff{t}=\frac{1}{2}\Omega_{\nu}\left(\frac{\nu}{2}-1\right)!\;,
\end{align}
dove $r^{\nu-1}\Omega_{\nu}$ denota la superficie della sfera $\nu$-dimensionale (in particolare, $\Omega_{\nu}$ denota l'angolo solido $\nu$-dimensionale) e l'ultima eguaglianza sussiste perché si tratta della funzione $\Gamma$ di Eulero. Uguagliando le due espressioni trovate per $I$ troviamo:
\begin{equation}
\Omega_{\nu}=\frac{2\pi^{\nu/2}}{\left(\frac{\nu}{2}-1\right)!}\;, %% Questo è l'elemento di angolo solido %%
\end{equation}
così che il volume della sfera è:
\begin{equation}
\mathcal{V}=\int_0^R \Omega_{\nu}\diff{R}=\frac{\pi^{\nu/2}}{(\nu/2)!}R^{\nu}\;.
\end{equation}
Quindi, con sufficiente precisione:
\begin{equation}
\Delta\Gamma=V^N\mathcal{V}=V^{\nu/3}\frac{\pi^{\nu/2}}{(\nu/2)!}R^{\nu}\;,
\end{equation}
e, usando l'approssimazione di Stirling per il fattoriale:
\begin{equation}
\sigma=\log\Delta\Gamma=N\log[V\pi^{3/2}(2MU)^{3/2}]-\frac{3N}{2}\log\frac{3N}{2}+\frac{3N}{2}\;, \label{sec6_entropy}
\end{equation}
dove nell'espressione di $\mathcal{V}$ abbiamo posto $\nu=3N$ e $R=(2MU)^{1/2}$. \\

Vogliamo adesso porre $\sigma$ in una forma in cui possiamo esaminare l'additività. Possiamo riscrivere la \eqref{sec6_entropy} nella forma:
\begin{equation}
\sigma=N\log\left[V\left(\frac{4\pi M}{3}\right)^{3/2}\left(\frac{U}{N}\right)^{3/2}\right]+\frac{3N}{2}\;,
\end{equation}
ma questa non è additiva in quanto il volume $V$ appare nell'argomento del logaritmo. In effetti, se gli $N$ punti materiali sono indistinguibili, non dobbiamo considerare differenti gli stati che differiscono solo per uno scambio di particelle identiche fra loro nello spazio delle fasi, cioè abbiamo sovrastimato il volume dello spazio delle fasi di un fattore $N!$ rispetto alle condizioni classiche. Prendendo in considerazione questo fattore si ha:
\begin{equation}
\sigma=\log\frac{\Delta\Gamma}{N!}=N\log\left[\frac{V}{N}\left(\frac{4\pi M}{3}\right)^{3/2}\left(\frac{U}{N}\right)^{3/2}e\right]+\frac{3}{2}N\;.
\end{equation}
Questa formula è effettivamente additiva, in quanto solo il volume e l'energia per particella compaiono nell'argomento del logaritmo. Per completare la formula resta solo da introdurre $h^{3N}$ come unità di volume dello spazio delle fasi, così che:
\begin{equation}
\sigma=\log\frac{\Delta\Gamma}{N!h^{3N}}=N\log\left[\frac{(2M)^{3/2}\pi^{3/2}e(V/N)(U/N)^{3/2}}{(\frac{3}{2})^{3/2}h^3}\right]+\frac{3}{2}N\;.
\end{equation}
Qui $h$ è la costante di Planck. Dunque l'entropia $S$ sarà:
\begin{equation}
S=k\sigma \simeq kN\log\left[\frac{(2M)^{3/2}\pi^{3/2}e(V/N)(U/N)^{3/2}}{(\frac{3}{2})^{3/2}h^3}\right]+\frac{3}{2}kN\;,
\end{equation}
dove $k$ è la costante di Boltzmann. Osserviamo che:
\begin{align}
\left.\pdev{S}{U}\right|_{V=\mbox{cost}} &=\frac{3}{2U}kN\stackrel{!}{=}\frac{1}{T}\;, \\
\left.\pdev{S}{V}\right|_{U=\mbox{cost}} &= \frac{kN}{V}\stackrel{!}{=}\frac{p}{T}\;.
\end{align}
Dunque la $S$ che abbiamo ricavato è sostanzialmente corretta, infatti ritroviamo $U=3N_AkT/2$, cioè l'energia cinetica media di un gas perfetto e $(\partial U/\partial T)=3R/2$, cioè il calore molare.
\pagebreak
\section{Ensemble canonico}
Nonostante il ragionamento sugli ensemble microcanonici porti ad una formula sostanzialmente corretta, sorge il problema di fissare l'energia di un certo gas (in generale non sarà proporzionale alla temperatura come nel caso dei gas perfetti). Vogliamo fissare un ensemble a partire dalla temperatura. Sappiamo che, in un sistema $t$ composto da due parti $R,S$, di cui $S$ è una parte molto ridotta:
\begin{equation}
\diff[w]_t=C\diff[\Gamma_t]=c\diff[\Gamma_S]\times\diff[\Gamma_R]\;.
\end{equation}
Questa relazione vale nell'approssimazione $E_S\ll E_t$. Fissato il sistema $S$ in un particolare microstato, esistono più microstati di $R$ che corrispondo a quel dato microstato di $S$. L'obbiettivo è trovare la distribuzione di probabilità:
\begin{equation}
\diff{w_s}=C\diff{\Gamma_s}\times\Delta\Gamma_R\;.
\end{equation}
Dobbiamo supporre che i due sistemi siano debolmente accoppiati, cioè il termine di interazione fra i due sistemi porti un contributo trascurabile, ossia, in termini di Hamiltoniane $H_t=H_R+H_S$. Per quanto detto prima, $\Delta\Gamma_R=\mathrm{exp}(\sigma_R)$. In termini di energia $E_R=E_t-E_S$ e, per l'additività dell'entropia:
\begin{equation}
\sigma_R(E_R)=\sigma_R(E_t-E_s)\simeq \sigma_R(E_t)+\left.\pdev{\sigma_R}{E}\right|_{E=E_t}(-E_S)+\cdots\;,
\end{equation}
in quanto $E_S/E_t\ll 1$. Ricordando che $\partial S/\partial E=1/T$, abbiamo, per l'entropia matematica $\partial\sigma/\partial E=1/kT$:
\begin{equation}
\sigma_R(E_R)=\sigma_R(E_t)-\frac{E_S}{kT}+\cdots
\end{equation}
Dunque:
\begin{equation}
\diff{w_s}=Ce^{\sigma_R(E_t)}e^{-E_S/kT}\diff{\Gamma_S}=Ae^{-E_S/kT}\diff{\Gamma_S}\;.
\end{equation}
In generale:
\begin{equation}
\diff{w}=Ae^{-E/kT}\diff{\Gamma}\;.
\end{equation}
$A$ è una costante di normalizzazione che si ricava imponendo $\int\diff{w}=1$:
\begin{equation}
\frac{1}{A}=\int_{-\infty}^{\infty} e^{-E/kT}\diff{\Gamma}=\int_{-\infty}^{\infty} e^{-E/kT}\frac{\diff{p}\diff{q}}{h^{3N}N!}\;.
\end{equation}
Posto $\beta=1/kT$, definiamo dunque la funzione di partizione di Boltzmann:
\begin{equation}
Z(\beta)\equiv\int_{-\infty}^{\infty} e^{-\beta E}\diff{\Gamma}\;.
\end{equation}
Dunque la distribuzione di probabilità può essere scritta come:
\begin{equation}
\diff[w]=\frac{1}{Z(\beta)}e^{-\beta E}\frac{\diff[p]\diff[q]}{h^{3N}N!}\;.
\end{equation}
Per la trattazione termodinamica del sistema, non è necessario tenere in considerazione tutte le variabili dello spazio delle fasi, perciò vogliamo arrivare ad un'espressione del tipo:
\begin{equation}
\diff[w]=p(E)\diff[E]\;,
\end{equation}
dove $p(E)$ è una densità di probabilità. Allora si ha:
\begin{equation}
\diff[w]=p(E)\diff[E]=Ae^{-E/kT}\dev{\Gamma}{E}\diff[E]\;. \label{sec7_diffw}
\end{equation}
La quantità $\diff[\Gamma]/\diff[E]$ rappresenta il numero di stati la cui energia è compresa tra i valori $E$ e $E+\diff[E]$. \footnote{O equivalentemente, tra 0 ed $E$.} Il numero di stati macroscopici si potrà scrivere come:
\begin{equation}
\Delta\Gamma\simeq \left.\dev{\Gamma}{E}\right|_{\bar{E}}\delta E\;.
\end{equation}
Insieme alla condizione $p(\bar{E})\delta E\sim 1$, otteniamo la fluttuazione dell'energia del sistema dal valore medio $\bar{E}$.
Ci aspettiamo che valutando la \eqref{sec7_diffw} in $E=\bar{E}$ si ritrovi lo stesso valore:
\begin{equation}
p(\bar{E})\delta E=Ae^{-\bar{E}/kT}\left.\dev{\Gamma}{E}\right|_{E=\bar{E}}\delta E\;.
\end{equation}
Sostituendo si ottiene:
\begin{equation}
1=Ae^{-\bar{E}/kT}\Delta\Gamma=Ae^{-\bar{E}/kT}e^{\sigma}=\frac{1}{Z}e^{-\bar{E}/kT} e^{\sigma}\;,
\end{equation}
ricordando che $\log\Delta\Gamma=\sigma$. Da questo ricaviamo $Z(\beta)$:
\begin{align}
Z(\beta)&=e^{-\bar{E}/kT+\sigma}= \mathrm{exp}\left(\frac{-kT\sigma+\bar{E}}{kT}\right)= \mathrm{exp}\left(-\frac{\bar{E}-TS}{kT}\right) =  \mathrm{exp}(-\beta F)\;.
\end{align}
Dove abbiamo introdotto l'\textit{energia libera di Helmotz}:
\begin{align}
F(T,V) &\equiv U(S,V)-TS\;, \\
\left.\pdev{U}{S}\right|_V&=\frac{1}{T}\;.
\end{align}
Data la funzione di partizione $Z(\beta)$ è possibile calcolare l'energia media come:
\begin{equation}
U=-\frac{\partial}{\partial \beta}\log Z(\beta)=\frac{\int\diff[\Gamma]\;H e^{-\beta H}}{\int\diff[\Gamma]\;e^{-\beta H}}\;.
\end{equation}
Dalla termodinamica sappiamo che:
\begin{equation}
F(T,V) = U-TS = U-\tau\sigma\qquad \implies \qquad \begin{cases}
k\sigma = -\left.\dfrac{\partial F}{\partial T}\right|_V\;, \\
\\
P = -\left.\dfrac{\partial F}{\partial V}\right|_T\;.
\end{cases}
\end{equation}
Ma $F(T,V)=-kT\log Z(V,T)=\tau\log Z$. Invertendo la seconda relazione, si ha:
\begin{equation}
\sigma=-\frac{F}{\tau}+\frac{U}{\tau}\;.
\end{equation}
E quindi:
\begin{align}
\sigma=-\left.\pdev{F}{\tau}\right|_V=\frac{\partial}{\partial\tau}(\tau\log Z) &= \log Z+\tau\frac{\partial}{\partial\tau}\log Z \notag \\
&= \log Z-\beta\frac{\partial}{\partial\beta}\log Z  \notag \\
&= \log Z-\beta\frac{1}{Z}\pdev{Z}{\beta} \notag \\
&=-\frac{F}{\tau}+\frac{1}{\tau}\frac{1}{Z}\int \diff[\Gamma]\; H e^{-\beta H}\;.
\end{align}
Consideriamo la funzione di partizione $Z(\beta)$ e un gas descritto dall'Hamiltoniana:
\begin{equation}
H=\sum_{i=1}^N H_1^{(i)}\;,
\end{equation}
cioè la somma delle Hamiltoniane di singole particelle. Allora la funzione di partizione, se trascuriamo i termini di interazione:
\begin{equation}
Z(\beta)\simeq \int\diff[p]_1\diff[q]_1e^{-\beta H_1^{(1)}} \cdots \int \diff[p]_N\diff[q]_N e^{-\beta H_1^{(N)}}\simeq Z_1\cdot Z_2 \cdots Z_N \cdot C\;,
\end{equation}
dove $C$ è una costante moltiplicativa che non dipende dalla temperatura. In particolare, se come in un gas, tutte le particelle sono fra loro identiche:
\begin{equation}
Z(\beta)\simeq C(Z_1)^N\;.
\end{equation}
Per un gas composto da $N$ molecole che non interagiscono fra loro, si ha per una singola molecola:
\begin{equation}
H_1=\sum_{j=1}^{P} a_i p_i^2\;.
\end{equation}
In particolare, se la molecola è monoatomica, $H_1=p^2/2M$ e $P=3$ e si ha $U=3kT/2$. Se invece è biatomica :
\begin{equation}
H_i=\frac{\mathbf{p}^2}{2M}+\frac{p_{\xi}^2}{2I_{\xi}}+\frac{p_{\chi}^2}{2I_{\chi}}\;.
\end{equation}
$I_{\xi},I_{\chi}$ rappresentano le componenti del momento di inerzia della molecola. In questo caso, i gradi di libertà sono $5$ e quindi $U=5kT/2$. In generale, si è osservato sperimentalmente che a basse temperature tutti i gas si comportano come monoatomici, cioè caratterizzati da un calore specifico $3R/2$. Aumentando la temperatura, però, per molecole poliatomiche, avviene che il calore specifico passa da $3R/2$ a $5R/2$ e poi a $7R/2$, come se i gradi di libertà della molecola venissero "scongelati" dall'aumento della temperatura. Quindi si pone il problema della non costanza del numero di gradi di libertà. \\

Riassumendo, data la funzione di partizione di Boltzmann:
\begin{equation}
Z(\beta,v)=\int \diff[\Gamma]_N e^{-\beta H(p,q)}\;,
\end{equation}
possiamo ricavare le quantità principali della termodinamica:
\begin{align}
U(\beta, v) &= -\pdev{\log Z}{\beta}\;, \\
c_v &= \left.\pdev{U}{T}\right|_V\;.
\end{align}
\begin{exm}
Consideriamo un cristallo unidimensionale costituito da $N$ particelle materiali di massa $m$ nelle posizioni $x_1,x_2,\ldots,x_N$. L'Hamiltoniana del sistema è:
$$
H=\sum_{i=1}^N \frac{p_i^2}{2m}+\sum_{i\ne j} V(r_{ij})\;.
$$
dove $V(r_{ij})$ è il potenziale di interzione fra le particelle. Possiamo approssimare il sistema affermando che la particella $i$-esima interagisca solo con le particelle immediatamente adiacenti, cioè:
\begin{equation}
H\simeq \sum_{i} \frac{p_i^2}{2m}+\sum_i V(|r_i-r_{i+1}|)\;.
\end{equation}
Inoltre, poiché sappiamo che il potenziale ha un certo minimo corrispondente alla posizione di equilibrio, possiamo approssimarlo sviluppando in serie di Taylor:
\begin{equation}
V(|r_i+r_{i+1}|)\simeq -V_0+\frac{1}{2}k(r_i-r_{i+1})^2\;.
\end{equation}
Alla luce di ciò, l'Hamiltoniana diventa:
\begin{equation}
H \simeq \sum_i \frac{p_i^2}{2m}+\sum_i \frac{1}{2}k(x_i-x_{i+1})^2\;.
\end{equation}
Otteniamo così una forma quadratica per l'Hamiltoniana. Diagonalizzando la matrice e trovando le autofrequenze, possiamo scindere il sistema in un sistema di $N$ oscillatori armonici disaccoppiati, per cui la Hamiltoniana vale:
\begin{equation}
H = \sum_{\alpha=1}^N \left[\frac{P_{\alpha}^2}{2m}+\frac{1}{2}m\omega_{\alpha}^2 X_{\alpha}^2\right]=\sum_{\alpha=1}^N H_{\alpha}\;.
\end{equation}
Per ogni $\alpha$, la funzione di partizione è data da:
\begin{equation}
Z_{\alpha}=\int\diff[p]_{\alpha} \mathrm{exp}\left(-\beta\frac{P_{\alpha}^2}{2m}\right)\int_{-\infty}^{\infty} \diff[q]_{\alpha} \mathrm{exp}\left(-\beta\frac{m\omega_{\alpha}^2}{2}q_{\alpha}^2\right)
\end{equation}
Allora $H=\sum_{\alpha} H_{\alpha}=\sum_{\alpha}p_{\alpha}^2/2m+H_{\alpha}'$, dove $H_{\alpha}'$ può assumere solo un certo numero discreto di valori. Supponiamo che tale numero sia due; chiamati $E_1,E_2$ suddetti possibili valori, poniamo $\Delta=E_2-E_1$. Dato che i punti materiali sono identici, avremo $Z=(Z_1)^N$, dove:
\begin{equation}
Z_1=\int \diff[p]\diff[q] \mathrm{exp}\left(-\beta\frac{p^2}{2m}\right)\sum \mathrm{exp}\left(-\beta H'\right)=Z_1^{cl}\times Z'\;,
\end{equation}
avendo posto $\int \diff[p]\diff[q] \sum \mathrm{exp}(-\beta\ham')\equiv Z'$, in cui la somma è fatta sugli stati discreti possibili, nel nostro caso si ha:
\begin{equation}
Z'=\mathrm{exp}(-\beta E_1)+\mathrm{exp}(-\beta E_2)\;.
\end{equation}
Calcoliamo adesso l'energia media:
\begin{equation}
U= -\frac{\partial}{\partial\beta} \log Z=\frac{E_1e^{-\beta E_1}+E_2e^{-\beta E_2}}{e^{-\beta E_1}+e^{-\beta E_2}}=E_1P_1+E_2P_2\;.
\end{equation}
In generale:
\begin{equation*}
P_i=\frac{e^{-\beta E_i}}{\sum e^{-\beta E_i}}\qquad  \Longrightarrow\qquad  N_i=N\frac{e^{-\beta E_i}}{Z}\;,
\end{equation*}
e dunque:
\begin{equation}
\frac{N_2}{N_1}=\frac{e^{-\beta E_2}}{e^{-\beta E_1}}=e^{-(E_2-E_1)/kT}=e^{-\Delta/kT}\;.
\end{equation}
Inoltre:
\begin{align*}
\pdev{U}{T}=\pdev{\beta}{T}\pdev{U}{\beta}&= \frac{1}{kT^2}\frac{\Delta E_2e^{-\beta\Delta}(1+e^{-\beta\Delta})-\Delta e^{-\beta\Delta}(E_1+E_2e^{-\beta\Delta}}{(1+e^{-\beta\Delta})^2} \\
&= \frac{N_A}{kT^2}\frac{\Delta^2 e^{-\beta\Delta}}{(1+e^{-\beta\Delta})^2} \\
&= N_Ak \left(\frac{\Delta}{kT}\right)^2 \frac{e^{-\beta\Delta}}{(1+e^{-\beta\Delta})^2} \\
&= R \frac{e^{-\beta\Delta}}{(1+e^{-\beta\Delta})^2}\;.
\end{align*}
\end{exm}
\end{document}